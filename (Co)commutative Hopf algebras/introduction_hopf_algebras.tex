\section{Introduction}
    \subsection{Preliminaries}
        For now, we refer the reader to \cite{maclane} for generalities about category theory, and to \cite{EGNO_tensor_categories} for details on the notion of monoidal categories and the various strengthening thereof (symmetric monoidal categories, braided monoidal categories, etc.)\footnote{I will try to write a concise appendix about monoidal categories at some point, just to make these notes self-contained.}. For our intents and purposes, it suffices to only be familiar with tensor products of modules over rings, though it is important that one knows how to work with noncommutative rings; a good reference is \cite{lam_first_course_in_noncommutative_rings}. We assume that the reader is comfortable with modules and commutative algebras over commutative rings.

    \subsection{What are Hopf algebras, and where do we see them ?}
        Hopf algebras are highly symmetric and rigid algebraic entities that appear in many mathematical disciplines. Even though they have roots in algebraic topology, originating from the works of Heinz Hopf on certain kinds generalised topological groups\footnote{The keyword being \say{H-spaces}.}, nowadays they tend to be thought of as belonging to the realm of algebra, most notably in representation theory. 
    
        These notes focus primarily on Hopf algebras that are either commutative or cocommutative (or both!), with the reason being that examples of non-cocommutative (and often also noncommutative) Hopf algebras are extremely non-trivial. These are the so-called \say{quantum groups}, and discussing them requires us to assume a not-so-small amount of background in representation theory, symplectic geometry, and Hochschild deformation theory (and even some theoretical physics) on the part of the reader. As such, we have chosen to limit the scope of these notes somewhat, focusing mostly on the examples of Hopf algebras that come from group theory; a good reference is \cite{jantzen_representations_of_algebraic_groups}. Readers interested in thee non-cocommutative side of the story can consult \cite{kassel_quantum_groups}, \cite{etingof_schiffmann_lectures_on_quantum_groups}, \cite{chari_pressley_quantum_groups}, as well as \cite{etingof_kazhdan_quantisation_1} and its sequels.

    \subsection{How to read these notes}
        There are many exercises sprinkled all throughout these notes. Admittedly, this is mostly due to a lack of time (and laziness) on my part, but I have also made effort at being deliberate with their placements. The goal of these exercises are two-fold. Firstly, it is to remind you, dear reader, of some facts from classical ring theory that I think are relevant and therefore worth keeping in mind. Secondly, it is to help ease the transition towards abstractions of classical concepts.