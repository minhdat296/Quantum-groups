\section{Quantum Kac-Moody algebras}
    Here, we recall the notion of quantum Kac-Moody algebras from \cite{etingof_kazhdan_quantisation_6}.

    \subsection{Kac-Moody algebras}
        To begin, let us recall some basic features of Kac-Moody algebras. The canonical reference is the book \cite{kac_infinite_dimensional_lie_algebras}.
        
        Following \cite[Chapter 1]{kac_infinite_dimensional_lie_algebras}, let $C := ( C_{i, j} )_{1 \leq i, j \leq n}$ be a (generalised) Cartan matrix and choose for it a realisation $(\h, \simpleroots, \simpleroots^{\vee})$, i.e. a vector space $\h$ of dimension $l := 2n - \rank C$ along with linearly independent subsets $\simpleroots^{\vee} := \{ \alpha_i^{\vee} \}_{1 \leq i \leq n} \subset \h$ and $\simpleroots := \{ \alpha_i \}_{1 \leq i \leq n} \subset \h^*$, whose elements are known as \say{simple coroots} and \say{simple roots}, and are such that:
            \begin{equation} \label{equation: cartan_matrix_entries}
                \alpha_i( \alpha_j^{\vee} ) = C_{i, j}
            \end{equation}

        We can then define a Lie algebra $\tilde{\g}$ to be the one generated by the set:
            \begin{equation} \label{equation: kac_moody_generators}
                \h \cup \{ e_i^{\pm} \}_{1 \leq i \leq n}
            \end{equation}
        whose elements are subjected to the following relations:
            \begin{equation} \label{equation: extended_kac_moody_relations}
                \begin{gathered}
                    [h, h'] = 0 \quad, \quad h, h' \in \h
                    \\
                    [h, e_i^{\pm}] = \pm \alpha_i(h) e_i^{\pm} \quad, \quad [e_i^+, e_j^-] = \delta_{i, j} \alpha_i^{\vee} \quad, \quad h \in \h, 1 \leq i \leq n
                \end{gathered}
            \end{equation}
        It can be shown (see e.g. \cite[Theorem 1.2]{kac_infinite_dimensional_lie_algebras}) that $\tilde{\g}$ admits a \say{triangular decomposition}:
            $$\tilde{\g} \cong \tilde{\n}^- \oplus \h \oplus \tilde{\n}^+$$
        wherein $\tilde{\n}^{\pm}$ are the free Lie algebras generated by the sets $\{ e_i^{\pm} \}_{1 \leq i \leq n}$.
            
        Next, if we let $\r \subset \tilde{\g}$ be the Lie ideal that is the sum of all ideals with zero intersection with the Lie ideal $\h \subset \tilde{\g}$, then we shall obtain the \say{Kac-Moody algebra} $\g$ associated to the previously fixed Cartan matrix as the quotient:
            $$\g := \tilde{\g}/\r$$
        \textit{A priori} - and this is a somewhat non-trivial fact (see \cite[Theorem 9.11]{kac_infinite_dimensional_lie_algebras}) - the Lie algebra $\g$ is generated by the same set \eqref{equation: kac_moody_generators}\footnote{Even though it is technically an abuse of notations, we shall use the same symbols to denote the elements of \eqref{equation: kac_moody_generators} and their images under the quotient map $\tilde{\g} \to \g$.}, and in addition to the relations \eqref{equation: extended_kac_moody_relations}, the generators now satisfy also the so-called \say{Serre relations}, which take the following form in the adjoint representation of $\g$:
            \begin{equation} \label{equation: kac_moody_serre_relations}
                ( \ad( e_i^{\pm} ) )^{1 - C_{i, j}} \cdot e_j \quad, \quad 0 \leq i \not = j \leq l
            \end{equation}
        or in other words, $\r$ is generated by such relations. This leads to a triangular decomposition:
            $$\g \cong \n^- \oplus \h \oplus \n^+$$
        wherein $\n^{\pm} := \tilde{\n}^{\pm}/( \tilde{\n}^{\pm} \cap \r )$

        Due to the relations $[h, e_i^{\pm}] = \pm \alpha_i(h) e_i^{\pm}$, the Lie algebra $\tilde{\g}$ has a canonical \say{root grading} by the abelian group $\rootlattice := \Z \simpleroots$ (commonly called the \say{root lattice}), taking the form of a \say{root space decomposition}:
            $$\tilde{\g} \cong \bigoplus_{\alpha \in \rootlattice} \tilde{\g}_{\alpha}$$
        wherein $\tilde{\g}_{\alpha} := \{ x \in \tilde{\g} \mid \forall h \in \h: [h, x] = \alpha(h) x \}$ are the \say{root spaces}. Any lattice element $\alpha := \sum_{1 \leq i \leq n} a_i \alpha_i \in \rootlattice$ has a \say{height} $\height(\alpha) := \sum_{1 \leq i \leq n} a_i$, which allows us to define $\deg x_{\alpha} := \height(\alpha)$ for all $x_{\alpha} \in \tilde{\g}_{\alpha}$. In particular, the degrees of the generators \eqref{equation: kac_moody_generators} are $\deg e_i^{\pm} = \pm 1$ and $\deg h = 0$ (for all $h \in \h$). Through the relations \eqref{equation: kac_moody_serre_relations}, one can also see that the graded components are all finite-dimensional.

        Henceforth, let us assume moreover that the Cartan matrix $C$ is symmetrisable, i.e. that there exists an invertible diagonal $n \x n$ matrix $D$ and a symmetric $n \x n$ matrix $A$ (called the symmetrisation of $C$) such that:
            $$C := DA$$
        This allows us to define a symmetric and non-degenerate bilinear form $(\cdot, \cdot)_{\h} \in \Hom( \Sym^2(\h), \bbC )$ given by:
            \begin{equation} \label{equation: kac_moody_pairing_on_cartan_subalgebras}
                ( \alpha_i^{\vee}, h ) := \delta_{i, j} D_{i, i}^{-1} \alpha_i(h) \quad, \quad h \in \h, 1 \leq i \leq n
            \end{equation}
        As an aside, we note that equations \eqref{equation: kac_moody_pairing_on_cartan_subalgebras} and \eqref{equation: cartan_matrix_entries} together imply that:
            $$A_{i, j} = (\alpha_i^{\vee}, \alpha_j^{\vee}) = \delta_{i, j} D_{i, i}^{-1} \alpha_i( \alpha_j^{\vee} ) = \delta_{i, j} D_{i, i}^{-1} C_{i, j}$$
        Anyhow, the bilinear form $(\cdot, \cdot)_{\h}$ constructed above induces, via an induction process\footnote{For a less \textit{ad hoc} construction of $(\cdot, \cdot)_{\g}$, we refer the reader to \cite{neher_pianzola_prelat_sepp_invariant_bilinear_forms_via_fppf_descent}.}, a symmetric, non-degenerate, and \textit{invariant} bilinear form $(\cdot, \cdot)_{\g} \in \Hom( \Sym^2(\g)^{\g}, \bbC )$ given by:
            \begin{equation} \label{equation: kac_moody_pairing}
                ( e_i^-, e_j^+ )_{\g} = \delta_{i, j} D_{i, i}^{-1}
            \end{equation}
        and is uniquely determined by $(\cdot, \cdot)_{\h}$. Moreover, the bilinear form $(\cdot, \cdot)_{\g}$ extends to $\tilde{\g}$; this extension is a degenerate bilinear form whose radical is precisely $\r \subset \tilde{\g}$. Additionally, and this is a rather important fact for us, the bilinear form $(\cdot, \cdot)_{\g}$ is of total degree $0$ with respect to the $\rootlattice$-grading on $\Hom( \Sym^2(\g)^{\g}, \bbC )$ induced by the one on $\g$ (and this is one reason why the root spaces $\g_{\alpha} \subset \g$ being finite-dimensional is important).
        \begin{convention}
            For brevity, let us refer to symmetric (and often also non-degenerate) and invariant bilinear forms as \say{invariant pairings}.
        \end{convention}

        Now, assuming that the Cartan matrix $C$ is symmetrisable, let us now direct our focus towards the cases wherein $\g$ is of an affine type in Kac's classification as in \cite[Chapter 4]{kac_infinite_dimensional_lie_algebras}. From \cite[Chapter 7]{kac_infinite_dimensional_lie_algebras}, we know that in such cases, there exists a \say{loop realisation}:
            $$\g \cong ( \Loop \bar{\g} \oplus \bbC \level ) \rtimes \bbC \del$$
        \todo[inline]{Loop realisations for affine Kac-Moody algebras}

    \subsection{The standard Lie bialgebra structure and its quantisation}
        Within the context of Lie theory, invariant pairings are important for all manners of reasons. For us, they are indispensable for constructing so-called \say{Manin triples}, which are tools for constructing Lie bialgebra structures. Let us recall the following definition from \cite[Subsection 2.6]{appel_laredo_2_categorical_etingof_kazhdan_quantisation} (see also \cite[Subsection 7.4]{etingof_kazhdan_quantisation_1}).
        \begin{definition}[Manin triples] \label{def: manin_triples}
            A \textbf{Manin triple} is a triple of Lie algebras $(\a, \a^+, \a^-)$ such that $\a = \a^- \oplus \a^+$, along with a non-degenerate invariant pairing $(\cdot, \cdot) \in \Hom( \Sym^2(\a)^{\a}, \bbC )$, which are to satisfy the following conditions.
            \begin{itemize}
                \item With respect to $(\cdot, \cdot)$, the Lie subalgebras $\a^{\pm} \subset \a$ are to be isotropic. 
                \item The non-degenerate pairing $(\cdot, \cdot)$ induces an isomorphism of topological vector spaces $\a^- \xrightarrow[]{\cong} (\a^+)^*$, with $\a^+$ equipped with the discrete topology and $(\a^+)^*$ equipped with the weak topology.
                \item The commutator on $\a = \a^- \oplus \a^+$ is continuous with respect to the topologies chosen above. 
            \end{itemize}
        \end{definition}
        Now, if $(\a, \a^+, \a^-)$ is a Manin triple, then we can construct a Lie bialgebra structure on $\a = \a^+ \oplus \a^-$ in the following manner. Let $\calr \in \a^+ \tensor \a^- \subset \a \tensor \a$ is the canonical element, corresponding to $\id_{\a}$ via the non-degenerate pairing on $\a$. Then, one can check that the following construction defines a topological Lie cobracket $\delta: \a \to \a \wedge \a$:
            \begin{equation}
                \delta(x) := [\Box(x), \calr] \quad, \quad x \in \a
            \end{equation}
        wherein $\Box(x) := x \tensor 1 + 1 \tensor x$. This is compatible with the Lie bracket on $\a$ in such a manner that $\a$ is a Lie bialgebra. Also, it can be shown that:
            $$\pm \delta^{\pm} := \delta|_{\a^{\pm}}: \a^{\pm} \to \a^{\pm} \tensor \a^{\pm}$$
        are well-defined Lie sub-bialgebra structures, which are such that each of the Lie cobrackets $\delta^{\pm}$ is dual to the Lie bracket on $\a^{\pm}$, respectively. Additionally, $(\a, \delta)$ is isomorphic to Drinfeld's classical double of either $(\a^{\pm}, \delta^{\pm})$ (more on this shortly); indeed, these doubles are isomorphic to one another.
        \begin{remark}
            It can be checked that $\calr$ satisfies the \say{classical Yang-Baxter equation}:
                \begin{equation} \label{equation: CYBE}
                    [\calr_{1, 2}, \calr_{1, 3}] + [\calr_{1, 2}, \calr_{2, 3}] + [\calr_{1, 3}, \calr_{2, 3}] = 0
                \end{equation}
            and for this reason, Lie bialgebras that arise in this manner are known as being \say{quasi-triangular}.
        \end{remark}
        
        Conversely, given a Lie bialgebra structure $\delta^+: \a^+ \to \a^+ \wedge \a^+$, one can construct a Manin triple $(\a, \a^+, \a^-)$ with $\a^- := (\a^+)^*, \a := \a^+ \oplus \a^-$, and the non-degenerate and invariant pairing on $\a$ is the canonical one between $\a^+$ and $\a^-$, given by:
            $$(x, \varphi) := \varphi(x) \quad, \quad x \in \a^+, \varphi \in \a^-$$
        Moreover, $\a^-$ automatically carries an induced Lie bialgebra structure $\delta^-$ given by dualising the Lie bracket on $\a^+$; there is thus also a Lie bialgebra structure on $\a$ given by $\delta := \delta^+ \oplus (-\delta^-)$. As such, the Manin triple constructed above is in fact a triple of Lie bialgebras; the procedure above that outputs the Lie bialgebra $(\a, \delta)$ from the Lie sub-bialgebra $(\a^+, \delta^+)$ is commonly known as Drinfeld's \textbf{classical double} construction, and we write:
            $$\a \cong \Dr(\a^+)$$
        It is also easy to see that $\a \cong \Dr(\a^-)$.

        In short, the procedure described above yields us a bijective correspondence:
            \begin{equation} \label{equation: manin_triple_lie_bialgebra_correspondence}
                \left\{ \text{Manin triples $(\a, \a^+, \a^-)$} \right\} \leftrightarrows \left\{ \text{Lie bialgebra structures $(\a^+, \delta^+)$} \right\}
            \end{equation}
        wherein the forward map is given by $\delta^+ := [\Box, \calr]|_{\a^+}$ while the backward map sends $(\a^+, \delta^+)$ to $(\Dr(\a^+) := \a^+ \oplus (\a^+)^*, \a^+, (\a^+)^*)$.

        Now, let $\b^{\pm} := \n^{\pm} \oplus \h$ be the \say{Borel subalgebras} of the Kac-Moody algebra $\g$. It is clear that these Lie subalgebras of $\g$ are not isotropic with respect to the Kac-Moody pairing $(\cdot, \cdot)_{\g}$ given by equation \eqref{equation: kac_moody_pairing}. However, one can consider instead the larger Lie algebra $\a := \b^+ \oplus \b^-$, into which $\b^{\pm}$ embed by means of the maps $\eta^{\pm}: \b^{\pm} \hookrightarrow \a$ given by:
            \begin{equation} \label{equation: borel_lie_sub_bialgebra_embeddings}
                \eta^{\pm}(x) := x \oplus ( \pm x_{\h} ) \quad, \quad x \in \b^{\pm}
            \end{equation}
        wherein $x_{\h}$ is the image of $x \in \b^{\pm}$ under the canonical quotient map $\b^{\pm} \to \b^{\pm}/\n^{\pm} \cong \h$. This larger Lie algebra shall be equipped with the non-degenerate and invariant pairing given by:
            $$(\cdot, \cdot)_{\a} := (\cdot, \cdot)_{\g} - (\cdot, \cdot)_{\h}$$
        in which the Lie subalgebras $\eta^{\pm}(\b^{\pm})$ are clearly isotropic with respect to $(\cdot, \cdot)_{\a}$. As such, there is a Manin triple:
            $$(\a, \eta^+(\b^+), \eta^-(\b^-))$$
        from which arises the Lie bialgebra structures $\delta^{\pm}: \eta^{\pm}(\b^{\pm}) \to \eta^{\pm}(\b^{\pm}) \wedge \eta^{\pm}(\b^{\pm})$ given by:
            $$\delta^{\pm} = [\Box, \calr]$$
        wherein $\calr$ is the Casimir tensor. On generators, these Lie cobrackets are given by:
            \begin{equation} \label{equation: standard_kac_moody_lie_bialgebra_structure}
                \begin{gathered}
                    \delta^{\pm}(h) = 0 \quad, \quad h \in \h
                    \\
                    \delta^{\pm}(e_i^{\pm}) = \frac12 D_{i, i} e_i^{\pm} \wedge \alpha_i^{\vee} \quad, \quad 1 \leq i \leq n
                \end{gathered}
            \end{equation}
        \begin{remark}
            Note also, that by construction, we have that:
                $$\a \cong \Dr( \eta^{\pm}( \b^{\pm} ) )$$
            as Lie bialgebras.
        \end{remark}
            
        As a consequence of this construction, the Lie subalgebra $\h \subset \a$ is a Lie coideal on top of being a Lie ideal (this is trivial, for it is abelian), and hence the quotient:
            $$\g \cong \a/\h$$
        carries a Lie bialgebra structure given by the same formulae as in \eqref{equation: standard_kac_moody_lie_bialgebra_structure}.
        \begin{remark}
            In fact, it is well-known that the formulae in \eqref{equation: standard_kac_moody_lie_bialgebra_structure} lift to Lie bialgebra structures $\tilde{\delta}^{\pm}: \tilde{\b}^{\pm} \to \tilde{\b}^{\pm} \wedge \tilde{\b}^{\pm}$ on the Borel subalgebras $\tilde{\b}^{\pm} := \h \oplus \tilde{\n}^{\pm}$, and hence also on the extended Kac-Moody algebra $\tilde{\g}$. This is a \textit{post hoc} construction, inspired by the construction of the Lie bialgebra structures on $\b^{\pm}$ described above, for $\tilde{\g}$ possesses no non-degenerate pairings. In particular, this means that the Lie cobrackets $\tilde{\delta}^{\pm}$ do \textit{not} arise from any Manin  triple. It can also be shown, without much difficulty, that the Lie algebra $\tilde{\a} := \tilde{\b}^+ \oplus \tilde{\b}^-$ with the Lie cobracket given by $\tilde{\delta} := \tilde{\delta}^+ \oplus (-\tilde{\delta}^-)$ is isomorphic to the classical doubles $\Dr(\tilde{\b}^{\pm})$; note that the embeddings of $\tilde{\b}^{\pm}$ into this larger Lie bialgebra are given by the same formulae as in \eqref{equation: borel_lie_sub_bialgebra_embeddings}. 

            That said, the existence of such a Lie bialgebra structure on the extended Kac-Moody algebra $\tilde{\g}$ is very useful for formulating a quantisation of $\g$, as we shall explain below. 
        \end{remark}

        Let us now construct the extended quantum Kac-Moody algebra $\calU_{\hbar}(\tilde{\g})$ as the quantum double of a quantisation $\calU_{\hbar}(\tilde{\b}^{\pm})$ of the extended Borel subalgebras $\tilde{\b}^{\pm}$. From now on, let:
            $$q := e^{\frac12 \hbar}$$
        The following is known from \cite[Propositions 3.1 and 3.2]{etingof_kazhdan_quantisation_6}, in which:
            $$\calU_{\hbar}: \bbC\-\LBA \to \bbC[\![\hbar]\!]\-\QUE$$
        is the Etingof-Kazhdan quantisation functor going from the category of Lie biagebras over $\bbC$ to the category of quantised universal enveloping algebras (QUEs) over $\bbC[\![\hbar]\!]$ (i.e. topologically free Hopf $\bbC[\![\hbar]\!]$-algebras which are cocommutative modulo $\hbar$).
        \begin{proposition} \label{prop: extended_quantum_borel_subalgebras}
            The QUEs $\calU_{\hbar}(\tilde{\b}^{\pm})$ are, respectively, isomorphic to the QUEs $\calU_{\hbar}^{\pm}$ topologically generated by the sets:
                $$\h \cup \{ E_i^{\pm} \}_{1 \leq i \leq n}$$
            whose elements are subjected to the following relations:
                \begin{equation} \label{equation: extended_quantum_borel_subalgebras_commutation_relations}
                    \begin{gathered}
                        [h, h'] = 0 \quad, \quad h, h' \in \h
                        \\
                        [h, E_i^{\pm}] = \alpha_i(h) E_i^{\pm} \quad, \quad h \in \h, 1 \leq i \leq n 
                    \end{gathered} 
                \end{equation}
            and when evaluated on the generators, the coproducts $\Delta^{\pm}: \calU_{\hbar}^{\pm} \to \calU_{\hbar}^{\pm} \hattensor \calU_{\hbar}^{\pm}$ read:
                \begin{equation} \label{equation: extended_quantum_borel_subalgebras_coproducts}
                    \begin{gathered}
                        \Delta^{\pm}(h) = \Box(h) \quad, \quad h \in \h
                        \\
                        \Delta^+(E_i^+) = E_i^+ \tensor q^{D_{i, i} \alpha_i^{\vee}} + 1 \tensor E_i^+ \quad, \quad \Delta^-(E_i^-) = E_i^- \tensor 1 + q^{-D_{i, i} \alpha_i^{\vee}} \tensor E_i^- \quad, \quad 1 \leq i \leq n 
                    \end{gathered}
                \end{equation}
        \end{proposition}
        \begin{remark}
            The idea behind this result stems from the fact that, when regarded merely as an algebra, the QUE $\calU_{\hbar}(\a)$ coming from a Lie bialgebra $(\a, \delta)$ is isomorphic to $\calU(\a)[\![\hbar]\!]$, and only the coproduct is deformed to a non-commutative expression. In particular, this is why the relations \eqref{equation: extended_quantum_borel_subalgebras_commutation_relations} that determine the algebra structure of $\calU_{\hbar}(\tilde{\b}^{\pm})$ are identical to those that determine their classical limits $\tilde{\b}^{\pm}$ (cf. equation \eqref{equation: extended_kac_moody_relations}).
        \end{remark}

    \subsection{Presentation by Drinfeld currents for quantum affine Kac-Moody algebras}
        Let us now focus on the cases when $\g$ is of an affine type.
        
        Thanks to the loop realisation, we can now write the Casimir tensor of $\g$ more succinctly in the following manner:
            \begin{equation} \label{equation: affine_casimir_tensor}
                \begin{aligned}
                    \calr(z, w) & = \sum_{\alpha \in \bar{\rootsystem}} x_{\alpha} z^n \tensor x_{-\alpha} w^{-n} + ( \level \tensor \del + \del \tensor \level )
                    \\
                    & = \bar{\calr} \1(zw^{-1}) + ( \level \tensor \del + \del \tensor \level )
                \end{aligned}
            \end{equation}
        wherein $x_{\pm \alpha} \in \g_{\alpha}$ are orthonormal real root vectors, $\bar{\calr}$ is the Casimir tensor of the underlying finite-type Lie algebra $\bar{\g}$, and $\1(zw^{-1}) := \sum_{n \in \Z} z^n w^{-n}$ is the formal Dirac distribution\footnote{We use this alternate notation in order to avoid confusion with Lie cobrackets.}. Using formula \eqref{equation: affine_casimir_tensor}, we can rewrite the formulae for the Lie cobrackets from \eqref{equation: standard_kac_moody_lie_bialgebra_structure} into the following:
            \begin{equation}
                \begin{gathered}
                    \delta(\level) = \delta(\del) = 0
                    \\
                    \begin{aligned}
                        \delta( x t^m ) & = [x z^m \tensor 1 + 1 \tensor x w^m, \calr(z, w)]
                        \\
                        & = [ x z^m \tensor 1 + 1 \tensor x w^m, \bar{\calr} \1(zw^{-1}) + ( \level \tensor \del + \del \tensor \level ) ]
                        \\
                        & = \left( [x \tensor 1, \bar{\calr}] z^m + [1 \tensor x, \bar{\calr}] w^m \right) \1(zw^{-1}) + m \left( x z^m \tensor \level + \level \tensor x w^m \right)
                        \\
                        & = m \left( x z^m \tensor \level + \level \tensor x w^m \right)
                    \end{aligned}
                \end{gathered}  
            \end{equation}
        for all $x \in \bar{\g}$ and $m \in \Z$, wherein the last equality holds because $\bar{\calr}$ is central in $\calU(\bar{\g})$ \textit{a priori}; more succinctly still, we can write down the following action of the Lie cobracket $\delta: \g \to \g \wedge \g$ on currents $x(t) := \sum_{m \in \Z} x[m] t^{-m}$ (with $x[m] \in \bar{\g}$ for all $m \in \Z$) as:
            $$\delta( x(t) ) = \del x(z) \tensor \level + \level \tensor \del x(w)$$
        This gives us a Lie bialgebra structure on the loop algebra $\Loop \bar{\g}$, now regarded as the quotient $\Loop \bar{\g} \cong \g/(\bbC \level \oplus \bbC \del)$. 