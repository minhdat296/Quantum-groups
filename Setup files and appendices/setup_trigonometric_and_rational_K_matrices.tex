\section{Universal K-matrices}
    \subsection{Abstract K-matrices and reflection algebras}
        \begin{definition}[Spectral bQYBEs (additive form)] \label{def: additive_spectral_bQYBE}
            Given a solution $\calR(u)$ to the QYBE \eqref{equation: additive_spectral_QYBE}, the resulting \textbf{(spectral) boundary quantum Yang-Baxter equation (bQYBE)} reads:
                \begin{equation} \label{equation: additive_spectral_bQYBE}
                    \calR(u - v)_{1, 2} \calK(u)_1 \calR(u + v)_{1, 2} \calK(v)_2 = \calK(v)_2 \calR(u + v)_{1, 2} \calK(u)_1 \calR(u - v)_{1, 2} 
                \end{equation}
            Solutions:
                $$\calK(u)$$
            to such a bQYBE shall be referred to as \textbf{(spectral) K-matrices}.
        \end{definition}
        \begin{remark}
            In fact, one can even input different solutions of the same type (trigonometric or rational, in particular) to the QYBE into equation \eqref{equation: additive_spectral_bQYBE}, but the resulting bQYBEs are beyond the scope of our interest.
        \end{remark}
    
        \begin{definition}[Boundary monodromy matrices] \label{def: boundary_monodromy_matrices}
            Suppose that $\calR(u)$ is a solution to the QYBE \eqref{equation: additive_spectral_QYBE}, as well as a solution $\calK(u)$ to the bQYBE \eqref{equation: additive_spectral_bQYBE}. By evaluating $\calK(u)$ on the vector representation $\bbV_N$, one obtains a \textbf{boundary monodromy matrix}, commonly denoted by:
                $$S(u)$$
            which automatically satisfies the \textbf{reflection equation}:
                \begin{equation} \label{equation: reflection_equations}
                    \calR(u - v)_{1, 2} S(u)_1 \calR(u + v)_{1, 2} S(v)_2 = S(v)_2 \calR(u + v)_{1, 2} S(u)_1 \calR(u - v)_{1, 2} 
                \end{equation}
        \end{definition}
        \begin{definition}[Reflection algebras] \label{def: reflection_algebras}
            Suppose that $\calR(u)$ is a solution to the QYBE \eqref{equation: additive_spectral_QYBE}, as well as a solution $\calK(u)$ to the bQYBE \eqref{equation: additive_spectral_bQYBE}. The \textbf{formal reflection algebra} associated to $\calK$, denoted by:
                $$\calB_{\hbar}(\calK)$$
            is the associative $\bbC[\![\hbar]\!]$-algebra generated by the coefficients of the matrix:
                $$S(u) \in $$
        \end{definition}

    \subsection{Twisted Yangians and rational K-matrices}
        \begin{definition}[(Extended) twisted Yangians associated to symmetric pairs] \label{def: (extended)_twisted_yangians}
            
        \end{definition}
    
        \begin{lemma}[Twisted quantum contractions] \label{lemma: twisted_quantum_contractions}
            The boundary monodromy matrix:
                $$S(u) \in \Mat_N( \calX^{\tw}(\calK) )[\![u^{-1}]\!]$$
            satisfies the following so-called \textbf{twisted quantum contraction} formula:
                \begin{equation} \label{equation: twisted_quantum_contraction}
                    S(u) S(-u) = z^{\tw}(u)
                \end{equation}
            for some invertible \textit{even} scalar series $z^{\tw}(u) \in \calX^{\tw}(\calK)$.
        \end{lemma}
            \begin{proof}
                
            \end{proof}
        \begin{lemma}[Centres of extended twisted Yangians] \label{lemma: centres_of_extended_twisted_yangians}
            \begin{enumerate}
                \item The coefficients of the twisted quantum contraction $z^{\tw}(u)$ from equation \eqref{equation: twisted_quantum_contraction} generated the centre $\calZ^{\tw}(\calK) := \rmZ( \calX^{\tw}(\calK) )$.
                \item Moreover, the even coefficients of $z^{\tw}(u)$ are algebraically independent from one another, and thus:
                    $$\calX^{\tw}(\calK) \cong \calY^{\tw}(\calK) \tensor \calZ^{\tw}(\calK)$$
            \end{enumerate}
        \end{lemma}
            \begin{proof}
                \begin{enumerate}
                    \item See \cite[Corollary 3.5]{guay_regelskis_twisted_yangians_for_symmetric_pairs_of_types_BCD}.
                    \item See \cite[Corollary 3.6]{guay_regelskis_twisted_yangians_for_symmetric_pairs_of_types_BCD}.
                \end{enumerate}
            \end{proof}