\section{Quantum vertex algebras} \label{section: quantum_vertex_algebras}
    \subsection{Non-local, braided, and quantum vertex algebras}
        We begin by recalling the definition of quantum vertex algebras over $\bbC[\![\hbar]\!]$ in the sense of \cite{etingof_kazhdan_quantisation_5}. Start with a topologically free $\bbC[\![\hbar]\!]$-module $\scrV$, in which a distinguished element $\ket{0} \in \scrV$, the so-called \textbf{vacuum vector}, is identified. Next, we equip it with a \textbf{state-field correspondence}:
            \begin{equation} \label{equation: state_field_correspondence}
                Y(-, z): \scrV \to \End_{\bbC[\![\hbar]\!]}(\scrV)(\!(z)\!)
            \end{equation}
        which is just a continuous $\bbC[\![\hbar]\!]$-linear map for now. If the state-field correspondence \eqref{equation: state_field_correspondence} satisfies \:
        \begin{itemize}
            \item the \textbf{vacuum condition}, which reads:
            \item as well as \textbf{weak associativity}, which reads:
        \end{itemize}
        then we will say that $\scrV$ is a \textbf{non-local vertex algebra}. If the weak associativity condition in the previous definition is strengthened to \textbf{braided associativity}\footnote{This is also called \say{S-locality}, but using this terminology will cause clashes later on.}, then we will obtain the notion of \textbf{braided vertex algebras}\footnote{Also called \say{weak quantum vertex algebras.}}. The condition reads:

        \begin{definition}[Quantum vertex algebras (qVAs)] \label{def: quantum_vertex_algebras}
            
        \end{definition}

    \subsection{Modules over quantum vertex algebras}
        \begin{definition}[qVAs (quasi-)modules] \label{def: formal_quantum_vertex_algebra_(quasi)_modules}
            \begin{enumerate}
                \item 
                \item 
                \item 
            \end{enumerate}
        \end{definition}
    
        \begin{definition}[qVAs deformed (quasi-)modules] \label{def: formal_quantum_vertex_algebra_deformed_(quasi)_modules}
            \begin{enumerate}
                \item 
                \item 
            \end{enumerate}
        \end{definition}