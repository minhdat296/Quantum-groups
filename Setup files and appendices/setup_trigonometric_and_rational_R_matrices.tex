\section{Trigonometric and rational R-matrices}
    \subsection{Trigonometric R-matrices}

    \subsection{Rational R-matrices}

    \subsection{Quantum loop algebras associated to trigonometric and rational R-matrices} \label{subsection: trigonometric_and_rational_loop_QUEs}
        Following \cite{guay_regelskis_wendlandt_R_matrix_presentation_of_loop_QUEs}, we begin with quantum (twisted) loop algebras given by RLL-type generators and relations. The defining relations in the following R-matrix presentation of (extended) quantum loop algebras are natural from the point of view of the FRT formalism, and let us also mention - for later reference - that it is useful to think of the (extended) quantum loop algebras defined below as trigonometric analogues of (extended) double Yangians, a kind of quantum double for the usual Yangian; see example \ref{example: trigonometric_and_rational_loop_QUEs} below. This is why the monodromy matrices are \say{doubled}, and later on, we will see that there is a natural trigonometric analogue of the Yangian (the so-called \say{$q$-Yangian}) that is realisable as the subalgebra of a quantum loop algebra generated by the non-negative Fourier modes of the aforementioned monodromy matrices.

        Because we will be employing quantum vertex algebras, it is more convenient to start with the formal versions of the (extended) quantum loop algebras, i.e. work over $\bbC[\![\hbar]\!]$, and then specialise the deformation parameter $q := e^{\hbar}$ later on when necessary. Moreover, it will be convenient to define these algebras relative to a level parameter $\level \in \bbC$. Definition \ref{def: extended_loop_QUEs_R_matrix_presentation} is, strictly speaking, a slight generalisation of the notion of the same name from \cite{guay_regelskis_wendlandt_R_matrix_presentation_of_loop_QUEs}; there, the authors were concerned only with the level $\level = 0$ case. Working at an arbitrary level $\level \in \bbC$ introduces an extra dependence on $q^{\pm \level}$ of the quantum R-matrix in relation \eqref{equation: extended_loop_QUEs_mixed_relation}, which disappears as $\level \to 0$ (see \cite[Remark 3.6]{guay_regelskis_wendlandt_R_matrix_presentation_of_loop_QUEs}).

        We fix once and for all a solution:
            $$\calR(u, v)$$
        to the spectral QYBE \eqref{equation: additive_spectral_QYBE}, which shall always be either trigonometric or rational.

        \begin{definition}[Abstract extended quantum loop algebras] \label{def: extended_loop_QUEs_R_matrix_presentation}
            The \textbf{(formal) extended quantum loop algebra} associated $\calR$ at a \textbf{level} $\level \in \bbC$ is the associative algebra over $\bbC[\![\hbar]\!]$ that we denote by:
                $$\DX_{\hbar}(\calR)_{\level}$$
            It is generated by the coefficients of the matrix entries of the elements:
                \begin{equation} \label{equation: RLL_creation_annihilation_operators}
                    \begin{cases}
                        L^+(u) := \sum_{r \geq 0} L^+[r] u^{r + 1} \in \Mat_N( \DX_{\hbar}(\calR)[\![u]\!] )
                        \\
                        L^-(u) := \sum_{r \geq 0} L^-[r] u^{-r - 1} \in \Mat_N( \DX_{\hbar}(\calR)[\![u^{-1}]\!] )
                    \end{cases}
                \end{equation}
            along with distinguished central elements:
                \begin{equation} \label{equation: extended_loop_QUE_level}
                    \level_{\calR}^{\pm} =
                    \begin{cases}
                        \text{$e^{\pm \hbar \level}$ if $\calR$ is trigonometric}
                        \\
                        \text{$\pm \level$ if $\calR$ is rational}
                    \end{cases}
                \end{equation}
            and these generators subjected to the following relations:
                $$[ \level_{\calR}, L^{\pm}_{i, j}[r] ] = 0 \quad, \quad 1 \leq i, j \leq N, r \geq 0$$
                \begin{equation}
                    \begin{cases}
                        L_{i, j}^-[0] = L_{j, i}^+[0] = 0
                        \\
                        L_{i, i}^+[0] L_{i, i}^-[0] = L_{i, i}^-[0] L_{i, i}^+[0]
                    \end{cases}
                    , \quad 1 \leq i < j \leq N
                \end{equation}
                \begin{equation} \label{equation: extended_loop_QUE_RLL_relation}
                    \calR(u, v) L_1^{\pm}(u) L_2^{\pm}(v) = L_2^{\pm}(u) L_1^{\pm}(v) \calR(u, v)
                \end{equation}
                \begin{equation} \label{equation: extended_loop_QUEs_mixed_relation}
                    \calR(u + \level, v) L_1^+(u) L_2^-(v) = L_2^-(u) L_1^+(v) \calR(u - \level, v)
                \end{equation}
        \end{definition}
        \begin{remark}
            It is not necessary to specify the underlying (twisted) loop algebra $\Loop^{\sigma} \g_N$, as this information is already carried by the R-matrix $\calR$.
        \end{remark}

        \begin{lemma}[Centres of extended quantum loop algebras] \label{lemma: centres_of_extended_loop_QUEs}
            \todo[inline]{Quantum determinants}
        \end{lemma}
            \begin{proof}
                
            \end{proof}
        \begin{definition}[Quantum loop algebras] \label{def: loop_QUEs_R_matrix_presentation}
            At any level $\level \in \bbC$, the \textbf{(formal) quantum loop algebra} associated $\calR$ is the associative algebra over $\bbC[\![\hbar]\!]$ given by:
                $$\calU_{\hbar}(\calR) := \DX_{\hbar}(\calR)_{\level}/\<\qdet L^{\pm}(u) = 1\>$$
        \end{definition}

        \begin{proposition}[PBW bases for quantum loop algebras] \label{prop: PBW_for_loop_QUEs}
            
        \end{proposition}
            \begin{proof}
                
            \end{proof}

        \begin{lemma}[Hopf structures on extended quantum loop algebras] \label{lemma: hopf_structure_on_extended_loop_QUEs}
            
        \end{lemma}
            \begin{proof}
                
            \end{proof}
        \begin{corollary}[Hopf structures on quantum loop algebras] \label{coro: hopf_structure_on_loop_QUEs}
            
        \end{corollary}
            \begin{proof}
                
            \end{proof}

        \begin{example}[Trigonometric and rational quantum loop algebras] \label{example: trigonometric_and_rational_loop_QUEs}
            Specialising the abstract QYBE solution $\calR$ in definition \ref{def: extended_loop_QUEs_R_matrix_presentation} to R-matrices of trigonometric and rational types, one recovers some well-known quantum algebras.
            \begin{itemize}
                \item When $\calR$ is the universal R-matrix of the quantum Kac-Moody algebra $\calU_{\hbar}(\sfX)$, i.e. of trigonometric type, the quantum loop algebra construction in definition \ref{def: extended_loop_QUEs_R_matrix_presentation} returns the usual quantum loop algebra of affine type $\sfX$, for which there is a Hopf algebra embedding:
                    $$\calU_{\hbar}( \calR )_{\level} \subset \calU_{\hbar}(\sfX)$$
                \item When $\calR$ is the universal R-matrix of the Yangian, i.e. of rational type, one obtains a Hopf algebra isomorphism:
                    $$\calU_{\hbar}( \calR )_{\level} \cong \DY_{\hbar}(\calR)_{\level}$$
                with the usual \textbf{(formal) double Yangian} at level $\level \in \bbC$ over $\bbC[\![\hbar]\!]$ (cf. e.g. \cite{molev_sugawara_operators_for_classical_lie_algebras}). Somewhat surprisingly, this is irrespective of what the diagram automorphism $\sigma$ is.
            \end{itemize}
        \end{example}

        \begin{lemma}[Drinfeld's current presentation for quantum loop algebras] \label{lemma: drinfeld_current_presentation_for_loop_QUEs}
            \todo[inline]{Equivalence between Drinfeld's current presentation and R-matrix presentation}
        \end{lemma}
            \begin{proof}
                
            \end{proof}
        \begin{corollary}[Triangular decomposition] \label{coro: triangular_decomposition_for_loop_QUEs}
            
        \end{corollary}
            \begin{proof}
                
            \end{proof}
        \begin{corollary}[Kac-Moody coproduct fomrulae for Drinfeld generators] \label{coro: coproduct_formulae_for_drinfeld_generators}
            
        \end{corollary}
            \begin{proof}
                
            \end{proof}
        \begin{remark}
            The formulae from corollary \ref{coro: coproduct_formulae_for_drinfeld_generators} should not be confused with those of the Drinfeld coproduct, which gives rise to an alternate Hopf algebra structure on quantum loop algebras. This is conjecturally related to the standard Kac-Moody coproduct by a Drinfeld twisting process, but we will not touch upon this point. We make this comment instead to put emphasis on the fact that the coideal subalgebras arising from the K-matrices of Appel-Regelskis-Vlaar are with respect to the Kac-Moody coproduct. 
        \end{remark}

        \todo[inline]{I'm not sure if the category O will end up being necessary to mention, but I'll keep this stuff here for the moment. If anything, I'll just need it in order to be able to write down explicit formulae for trigonometric K-matrices evaluated on objects of the category O.}
        \begin{definition}[The category O for quantum loop algebras] \label{def: category_O_loop_QUEs}
            
        \end{definition}
        \begin{lemma}[Tensor structures and semi-simplicity of the category O] \label{lemma: tensor_structure_and_semi_simplicity_of_category_O_loop_QUEs}
            \begin{enumerate}
                \item The weight module category $\calU_{\hbar}(\calR)\mod^{\rootlattice}$ is a tensor category with respect to $\tensor := \tensor_{\bbC}$.
                \item The category $\calO( \calU_{\hbar}(\calR) )$ is meromorphically braided via the universal R-matrix $\calR(w, z)$.
                \item The category $\calO^{\integrable}( \calU_{\hbar}(\calR) )$ a semi-simple meromorphically braided subcategory of $\calO( \calU_{\hbar}(\calR) )$.
            \end{enumerate}
        \end{lemma}
            \begin{proof}
                
            \end{proof}