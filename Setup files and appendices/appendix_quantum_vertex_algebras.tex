\section{Quantum vertex algebras}
    \subsection{Quantum vertex algebras from R-matrices}
        If $\scrV$ is a $\bbC[\![\hbar]\!]$-module then the submodule consisting of topologically nilpotent elements (i.e., elements on which the pseudo-uniformiser $\hbar$ of the underlying adic ring acts topologically nilpotently) will be denoted by $\scrV^{\circ}$. 
    
        We begin by recalling the definition of $\hbar$-adic quantum vertex algebras in the sense of \cite{etingof_kazhdan_quantisation_5}. 

        \begin{definition}[Formal quantum vertex algebras (qVAs)] \label{def: formal_quantum_vertex_algebras}
            Consider a topologically free $\bbC[\![\hbar]\!]$-module:
                $$\scrV$$
            with a distinguished element:
                $$\ket{0} \in \scrV$$
            identified (this is the so-called \textbf{vacuum vector}), and a \textbf{state-field correspondence}:
            \begin{equation} \label{equation: state_field_correspondence}
                Y(-, z): \scrV \hattensor_{\bbC[\![\hbar]\!]} \scrV \to \scrV^{\circ}(\!(z)\!)
            \end{equation}
            which is just a continuous $\bbC[\![\hbar]\!]$-linear map for now.
            \begin{enumerate}
                \item If the state-field correspondence \eqref{equation: state_field_correspondence} satisfies \:
                \begin{itemize}
                    \item the \textbf{vacuum condition}, which reads:
                    \item as well as \textbf{weak associativity}, which reads:
                \end{itemize}
                then we will say that $\scrV$ is a \textbf{non-local vertex algebra}.
                \item If the weak associativity condition in the previous definition is strengthened to \textbf{braided associativity}\footnote{This is also called \say{S-locality}, but using this terminology will cause clashes later on.}, then we will obtain the notion of \textbf{weak quantum vertex algebras}. The condition reads:
                \item Finally, if the braiding satisfies the QYBE, then the resulting structure will be referred to as a \textbf{quantum vertex algebra}.
            \end{enumerate}
        \end{definition}

    \subsection{Modules}
        \begin{definition}[qVAs (quasi-)modules] \label{def: formal_quantum_vertex_algebra_(quasi)_modules}
            \begin{enumerate}
                \item 
                \item 
                \item 
            \end{enumerate}
        \end{definition}
    
        \begin{definition}[qVAs deformed (quasi-)modules] \label{def: formal_quantum_vertex_algebra_deformed_(quasi)_modules}
            \begin{enumerate}
                \item 
                \item 
            \end{enumerate}
        \end{definition}