\section{(Twisted) Yangians of classical types}
    \subsection{(Extended) untwisted Yangians in the RTT presentation}
        All throughout:
            $$\g_N \subseteq \gl_N$$
        will be used to denote a finite-dimensional simple Lie algebra over $\bbC$ that is of one of the classical types in the Cartan-Killing Classification; equivalently, suppose that:
            $$
                \g_N =
                \begin{cases}
                    \sl_N, \quad N \geq 2
                    \\
                    \o_{2n + 1}, \quad N = 2n + 1, \quad n \geq 1
                    \\
                    \sp_{2n}, \quad N = 2n, \quad n \geq 1
                    \\
                    \o_{2n}, \quad N = 2n, \quad n \geq 1
                \end{cases}
            $$
        and respectively, we say that $\g_N$ is linear, odd-orthogonal, symplectic, or even-orthogonal. This Lie algebra has a natural action on $\calV_N := \bbC^{\oplus N}$ (the so-called \say{vector representation}), which shall always be equipped with the standard basis:
            $$\{e_i\}_{1 \leq i \leq N}$$
        consisting of vectors with $1$ in the $i^{th}$ entry and $0$ elsewhere. By finite-dimensionality, we identify:
            $$\calV_N^{\tensor 2} \xrightarrow[]{\cong} \End(\calV_N)$$
        and then write:
            $$E_{i, j}$$
        for the image of $e_i \tensor e_j$ under the identification above; explicitly, $E_{i, j}$ is the $N \x N$ matrix with $1$ in the $(i, j)^{th}$ entry and $0$ elsewhere. Moreover, recall that $\calV_N$ comes equipped with a natural $\g_N$-invariant and non-degenerate bilinear form that we shall denote by:
            $$\eta$$
        This form is symmetric for the linear type $\sfA$ and orthogonal types $\sfB, \sfD$, while skew-symmetric for the symplectic type $\sfC$.
    
        Let us begin by quickly recalling some general features of the RTT formalism, mostly so that we can set up and justify some key notations. We remark that this is only applicable to the classical types, due to a lack of faithful vector representations in the exceptional types.
            
        Begin with the \textbf{formal loop algebra}:
            $$\g_N(\!(t^{-1})\!) := \g_N \tensor \bbC(\!(t^{-1})\!)$$
        which shall always be equipped with the canonical $\Z$-grading given by $\deg xt^m := m$ for all $x \in \g_N$ and all $m \in \Z$. This Lie algebra can be endowed with the non-degenerate, invariant, and symmetric bilinear form given for all $x, y \in \g_N$ and $m, n \in \Z$ by:
            $$(xt^m, yt^n) := (x, y)_{\g_N} \delta_{m + n, -1}$$
        wherein $(-, -)_{\g_N}$ is a non-degenerate, invariant, and symmetric bilinear form on $\g_N$ (\textit{a priori}, a $\bbC^{\x}$-multiple of the Killing form). Using this bilinear form, one can then construct the following $\Z$-graded Manin triple:
            $$\left( \g_N(\!(t^{-1})\!), \g_N[t], \g_N[\![t^{-1}]\!] \right)$$
        This then defines a Lie cobracket $\delta: \g_N[t] \to \g_N[t_1] \tensor \g_N[t_2]$ given by:
            $$\delta( X ) := [\Box(X), \calr(t_1 - t_2)]$$
        for all $X \in \g_N[t]$ and for:
            $$\calr(u) := \frac{\calr_{\g_N}}{u} + O(1)$$
        (with $u$ a formal variable), which we call the Yangian \textbf{classical r-matrix}. Here, $\Box(X) := X \tensor 1 + 1 \tensor X$ is the \say{standard} classical coproduct, and $\calr_{\g_N} \in \g_N \tensor \g_N$ is the Casimir tensor\footnote{To be more precise, we have $\calr_{\g_N} \in \Sym^2(\g_N)^{\g_N}$. By viewing $(-, -)_{\g_N}$ as an element of $\Hom( \Sym^2(\g_N)^{\g_N}, \bbC )$, we can then regard $\calr_{\g_N}$ as the pre-image of $(-, -)_{\g_N}$ under the canonical map $\g_N \tensor \g_N \to \Hom(\g_N \tensor \g_N, \bbC)$. Since $(-, -)_{\g_N}$ is non-degenerate, symmetric, and invariant, the pre-image is unique and lies in the subspace $\Sym^2(\g_N)^{\g_N} \subset \g_N \tensor \g_N$.}.
        
        \textit{A priori}, the r-matrix $\calr$ is a solution to the \textbf{classical Yang-Baxter equation (CYBE)}:
            \begin{equation} \label{equation: spectral_CYBE}
                [\calr_{1, 2}(u), \calr_{1, 3}(u + v)] + [\calr_{1, 2}(u), \calr_{2, 3}(v)] + [\calr_{1, 3}(u + v), \calr_{2, 3}(v)] = 0
            \end{equation}
        Note also that:
            \begin{equation} \label{equation: spectral_classical_unitarity}
                \calr(u) = -\calr(-u)
            \end{equation}
        and because of this, we characterise the Yangian classical r-matrix as being \textbf{unitary}.
        
        A general result of Etingof-Kazhdan tells us that the graded Lie bialgebra $(\g_N[t], \delta)$ admits a \say{graded quantisation}:
            $$\calY_{\hbar}(\g_N)$$
        of the aforementioned Lie bialgebra structure on $\g_N[t]$; in particular, this means that there is isomorphism of $\Z$-graded \textit{Hopf algebras}:
            $$\calU(\g_N[t]) \cong \calY_{\hbar}(\g_N)/\hbar$$
        with the comultiplication on $\calU(\g_N[t])$ being given by $\Box(X)$ for all $X \in \g_N[t]$. This graded quantisation is we call the \textbf{formal Yangian}, and by specialising $\hbar$ to any $\hbar_0 \in \bbC \setminus \{0\}$, one obtains the \textbf{Yangian}:
            $$\calY(\g_N)$$
        It turns out that, up to isomorphisms, the Yangian does not depend on the choice of $\hbar_0$ (hence why we do not specify this choice in the notation), and this is because the Yangian $\calY(\g_N)$ carries a natural $\Z$-filtration such that $\calY_{\hbar}(\g_N) \cong \bigoplus_{n \geq 1} \calY(\g_N) \hbar^n$ and $\g_Nr \calY(\g_N) \cong \calU(\g_N[t]) \cong \calY_{\hbar}(\g_N)/\hbar$ (cf. \cite{drinfeld_original_yangian_paper}); as such, we refer to the Yangian as a \say{filtered quantisation} of $\calU(\g_N[t])$.

        Let:
            $$\calY_{\hbar}(\g_N)'$$
        be the $\bbC[\![\hbar]\!]$-submodule of $\Hom_{\bbC[\![\hbar]\!]}( \calY_{\hbar}(\g_N), \bbC[\![\hbar]\!] )$ spanned by so-called \say{tempered} $\bbC[\![\hbar]\!]$-linear functionals, which are elements $f := \sum_{m \geq 0} f_m$ such that $\lim_{m \to +\infty} \deg f_m = +\infty$.
            
        Such an element satisfies the \textbf{quantum Yang-Baxter equation (QYBE)}:
            \begin{equation} \label{equation: additive_spectral_QYBE}
                \calR_{1, 2}(u - v) \calR_{1, 3}(u) \calR_{2, 3}(v) = \calR_{2, 3}(v) \calR_{1, 3}(u) \calR_{1, 2}(u - v)
            \end{equation}
        wherein the subscript pairs indicate the pair of tensor copies that $\calR(u)$ is acting on.

        Finally, by letting 
            \begin{equation} \label{equation: spectral_RTT_relation}
                \calR(u - v) T_1(u) T_2(v) = T_2(v) T_1(u) \calR(u - v)
            \end{equation}
        obtaining the so-called \textbf{monodromy matrix}.
            $$T(u) \in \Mat_N( \calY(\g_N) )[\![u^{-1}]\!]$$
    
        Following \cite[Definition 2.1]{guay_regelskis_twisted_yangians_for_symmetric_pairs_of_types_BCD}, we work with the following definition of extended untwisted Yangians associated to a rational solution $\calR(u)$ to the QYBE on $\calV_N^{\tensor 3}$.
        \begin{definition}[Extended untwisted Yangians] \label{def: extended_untwisted_yangians}
            The \textbf{extended untwisted Yangian} associated to the classical Lie algebra $\g_N$ and a rational solution $\calR(u)$ to the QYBE \eqref{equation: additive_spectral_QYBE} is the associative algebra:
                $$\calX(\g_N, \calR(u))$$
            generated by the coefficients of the matrix entries of the elements:
                $$T(u) \in \Mat_N( \calX(\g_N)[\![u^{-1}]\!] )$$
            subjected to the RTT relation \eqref{equation: spectral_RTT_relation}.
        \end{definition}

        Let $P \in \calV_N^{\tensor 2}$ be the operator permuting the two tensor factors\footnote{Note that this is an involutive symmetry.}. We will be fixing once and for all the following solution:
            $$\calR_{\calV_N}(u) = 1 + \frac{P}{u} + \frac{Q}{u - \kappa}$$
        for the QYBE on $\calV_N^{\tensor 3}$, wherein $\kappa := \sgn(\g_N) + \frac{N}{2}$, with $Q = -2P$ for type $\sfA$, and:
            \begin{equation} \label{equation: ABCD_signature}
                \sgn(\g_N) =
                \begin{cases}
                    \text{$0$ if $\g_N$ is of type $\sfA$}
                    \\
                    \text{$-1$ if $\g_N$ is of either type $\sfB$ or $\sfD$}
                    \\
                    \text{$1$ if $\g_N$ is of type $\sfC$}
                \end{cases}
            \end{equation}
        As such, we can abbreviate:
            $$\calX(\g_N) := \calX(\g_N, \calR)$$

        \begin{lemma}[Automorphisms of extended untwisted Yangians] \label{lemma: automorphisms_of_extended_untwisted_yangians}
            \begin{enumerate}
                \item Let $f(u) \in \bbC[\![u^{-1}]\!]^{\x}$ be an invertible formal power series in $u^{-1}$; recall that, because $\bbC[\![u^{-1}]\!]$ is a local commutative ring with (unique) maximal ideal $u^{-1}\bbC[\![u^{-1}]\!]$, $f(u)$ must be of the form $1 + \sum_{r \geq 0} f^{(r)} u^{-r - 1}$. Then, the map:
                    $$\mu_f: \calX(\g_N)[\![u^{-1}]\!] \to \Mat_N(\calX(\g_N)[\![u^{-1}]\!])$$
                given by:
                    $$\mu_f( T(u) ) := f(u) T(u)$$
                defines an algebra automorphism of $\calX(\g_N)$.
                \item Let $a \in \bbC$. Then, the map:
                    $$\tau_a: \Mat_N(\calX(\g_N)[\![u^{-1}]\!]) \to \Mat_N(\calX(\g_N)[\![u^{-1}]\!])$$
                given by:
                    $$\tau_a( T(u) ) := T(u - a)$$
                defines an algebra automorphism of $\calX(\g_N)$.
            \end{enumerate}
        \end{lemma}
            \begin{proof}
                See \cite[Section 2]{guay_regelskis_twisted_yangians_for_symmetric_pairs_of_types_BCD}.
            \end{proof}

        \begin{lemma}[Quantum contractions] \label{lemma: quantum_contractions}
            The monodromy matrix:
                $$T(u) \in \Mat_N( \calX(\g_N) )[\![u^{-1}]\!]$$
            satisfies the \textbf{quantum contraction} formula:
                \begin{equation} \label{equation: quantum_contraction}
                    T(u + \kappa)^t T(u) = T(u) T(u + \kappa)^t = z(u)
                \end{equation}
            wherein $z(u) \in \calX(\g_N)[\![u^{-1}]\!]^{\x}$ is some invertible scalar series.
        \end{lemma}
            \begin{proof}
                See \cite[Section 2]{guay_regelskis_twisted_yangians_for_symmetric_pairs_of_types_BCD}, Equation 2.11 in particular.
            \end{proof}
            
        This leads us to the following definition.
        \begin{definition}[Untwisted Yangians] \label{def: untwisted_yangians}
            The \textbf{untwisted Yangian} is given by:
                $$\calY(\g_N) := \calX(\g_N)/\<z(u) - 1\>$$
        \end{definition}
        \begin{remark}[Unitarity]
            Through equation \eqref{equation: quantum_contraction}, we see that $\calY(\g_N)$ is equivalently the quotient of $\calX(\g_N)$ by the two-sided ideal generated by the relation $T(u + \kappa)^t T(u) - 1$ (or equivalently $T(u) T(u + \kappa)^t - 1$). Often, we will refer to these relations collectively as the \textbf{unitarity relation} or the \textbf{unitary condition}.
        \end{remark}
        \begin{lemma}[Centres of extended untwisted Yangians] \label{lemma: centres_of_extended_untwisted_yangians}
            \begin{enumerate}
                \item The coefficients of the quantum contraction $z(u)$ from equation \eqref{equation: quantum_contraction} generated the centre $\calZ(\g_N) := \rmZ( \calX(\g_N) )$.
                \item Moreover, the coefficients of $z(u)$ are algebraically independent from one another, and thus:
                    $$\calX(\g_N) \cong \calY(\g_N) \tensor \calZ(\g_N)$$
            \end{enumerate}
        \end{lemma}
            \begin{proof}
                \begin{enumerate}
                    \item 
                    \item 
                \end{enumerate}
            \end{proof}

    \subsection{(Extended) twisted Yangians of classical types}
        \begin{convention}
            Given a vector space $V$ and an operator $K(u) \in \End(V)(\!(u^{-1})\!)$, we shall write:
                $$K_1(u) := 1 \tensor K(u), K_2(u) := K(u) \tensor 1$$
            as shorthands.
        \end{convention}
        \begin{definition}[Spectral bQYBEs (additive form)] \label{def: additive_spectral_bQYBE}
            Given a rational solution $\calR(u)$ to the QYBE (written in additive form; cf. equation \eqref{equation: additive_spectral_QYBE}), the resulting \textbf{boundary quantum Yang-Baxter equation (bQYBE)} reads:
                \begin{equation} \label{equation: additive_spectral_bQYBE}
                    \calR_{1, 2}(u - v) \calK_1(u) \calR_{1, 2}(u + v) \calK_2(v) = \calK_2(v) \calR_{1, 2}(u + v) \calK_1(u) \calR_{1, 2}(u - v) 
                \end{equation}
            Solutions:
                $$\calK(u)$$
            to such a bQYBE shall be referred to as \textbf{rational-type universal K-matrices}.
        \end{definition}
        \begin{remark}
            In fact, one can even input different rational-type solutions to the QYBE into equation \eqref{equation: additive_spectral_bQYBE}, but the resulting bQYBEs are beyond the scope of our interest.
        \end{remark}
    
        \begin{definition}[Boundary monodromy matrices] \label{def: twisted_yangian_boundary_monodromy_matrices}
            
        \end{definition}
        \begin{lemma}[Reflection equations] \label{lemma: twisted_yangian_boundary_monodromy_matrices_satisfy_reflection_equations}
            The boundary monodromy matrix $S(u)$ as in definition \ref{def: twisted_yangian_boundary_monodromy_matrices} satisfy the following so-called \textbf{reflection equation}\footnote{Also called the quaternary relation, especially in the context of twisted Yangians. See e.g. \cite[Proposition 2.2.1]{molev_yangians_and_classical_lie_algebras}.}:
                \begin{equation} \label{equation: reflection_equations}
                    \calR_{1, 2}(u - v) S_1(u) \calR_{1, 2}(u + v) S_2(v) = S_2(v) \calR_{1, 2}(u + v) S_1(u) \calR_{1, 2}(u - v) 
                \end{equation}
        \end{lemma}
            \begin{proof}
                
            \end{proof}
    
        \begin{definition}[(Extended) twisted Yangians associated to symmetric pairs] \label{def: (extended)_twisted_yangians}
            
        \end{definition}

        \begin{lemma}[Twisted quantum contractions] \label{lemma: twisted_quantum_contractions}
            The boundary monodromy matrix:
                $$S(u) \in \Mat_N( \calX^{\tw}(\g_N, \vartheta) )[\![u^{-1}]\!]$$
            satisfies the following so-called \textbf{twisted quantum contraction} formula:
                \begin{equation} \label{equation: twisted_quantum_contraction}
                    S(u) S(-u) = z^{\tw}(u)
                \end{equation}
            for some invertible \textit{even} scalar series $z^{\tw}(u) \in \calX^{\tw}(\g_N, \vartheta)$.
        \end{lemma}
            \begin{proof}
                
            \end{proof}
        \begin{lemma}[Centres of extended twisted Yangians] \label{lemma: centres_of_extended_twisted_yangians}
            \begin{enumerate}
                \item The coefficients of the twisted quantum contraction $z^{\tw}(u)$ from equation \eqref{equation: twisted_quantum_contraction} generated the centre $\calZ^{\tw}(\g_N, \vartheta) := \rmZ( \calX^{\tw}(\g_N, \vartheta) )$.
                \item Moreover, the even coefficients of $z^{\tw}(u)$ are algebraically independent from one another, and thus:
                    $$\calX^{\tw}(\g_N, \vartheta) \cong \calY^{\tw}(\g_N, \vartheta) \tensor \calZ^{\tw}(\g_N, \vartheta)$$
            \end{enumerate}
        \end{lemma}
            \begin{proof}
                \begin{enumerate}
                    \item See \cite[Corollary 3.5]{guay_regelskis_twisted_yangians_for_symmetric_pairs_of_types_BCD}.
                    \item See \cite[Corollary 3.6]{guay_regelskis_twisted_yangians_for_symmetric_pairs_of_types_BCD}.
                \end{enumerate}
            \end{proof}