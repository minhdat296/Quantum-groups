\section{\texorpdfstring{Comparing presentations for QSPs in trigonometric and rational quantum affine $\sl_2$ algebras}{}}
    \subsection{\texorpdfstring{Highest-weight modules for QSPs in trigonometric and rational quantum affine $\sl_2$}{}}
        \todo[inline]{
            Recall the classification of finite-dimensional simple modules over QSPs inside $\calU_{\hbar}(\hat{\sl}_2)$ and $\DY_{\hbar}(\sl_2)$. For trigonometric QSPs, very little is known for reflection algebras outside of the types $\sfA.1$ and $\sfA.2$ (see \cite{gow_molev_representations_of_type_A_twisted_q_yangians}), as far as I am aware (I will make a suggestion for how we can circumvent this difficulty later); as for $\i$quantum groups, I believe that more is known, but I am still reading up on the details. For rational QSPs, the classifications of finite-dimensional simple modules over the reflection algebras can be found in \cite[Chapter 4]{molev_yangians_and_classical_lie_algebras} and \cite{molev_ragoucy_representations_of_reflection_algebras} (for type $\sfA$), as well as \cite{guay_regelskis_wendlandt_representations_of_twisted_yangians_for_symmetric_pairs_of_types_BCD_1} and \cite{guay_regelskis_wendlandt_representations_of_twisted_yangians_for_symmetric_pairs_of_types_BCD_2} (for the types $\sfB, \sfC, \sfD$; there are some subcases for which the classification remains unknown, but it's more or less a complete story). 
        }

    \subsection{\texorpdfstring{Embedding highest-weight modules into Fock spaces; comparison of presentations}{}}
        \todo[inline]{
            This is the most important part of the paper. In what follows, the "quantum affine algebra" will be used for referring to either $\calU_{\hbar}(\hat{\sl}_2)$ or $\DY_{\hbar}(\sl_2)$. Recall from the previous section, that the Fock space of the deformed oscillator algebra $\calA_{\hbar}(\calR)_{\level}$ is the cyclic left-$\calA_{\hbar}(\calR)_{\level}$-module with cyclic vector $\ket{0}$, the same underlying $\bbC[\![\hbar]\!]$-module as that of the subalgebra of $\calA_{\hbar}(\calR)_{\level}$ generated by $a^+(z)$ and $\level$, and is acted upon by:
                $$a^-(z) \ket{0} = 0 \quad, \quad \level \ket{0} = \level \ket{0}$$
            This gives rise naturally to an action of the quantum affine algebra by:
                $$L^-(z) \ket{0} = 0 \quad, \quad \level \ket{0} = \level \ket{0}$$
            and then an action of the reflection algebra by:
                $$B(z) \ket{0} = \calK(z) \ket{0}$$
            Now, the crucial part is this: in general, we do \textit{not} know how the universal K-matrix $\calK(z)$ looks explicitly, but we do know how it acts on integrable modules of the quantum affine algbra (and I think even more explicitly on finite-dimensional modules) by works of Appel and Vlaar, \cite{appel_vlaar_trigonometric_K_matrices_for_finite_dimensional_representations_of_affine_QUEs} and \cite{appel_vlaar_tensor_K_matrices} in particular; it is also worth specifying, that Appel and Vlaar worked with $\i$quantum groups, so in their setting, the fact that $B(z) \ket{0} = \calK(z) \ket{0}$ was invisible, and they knew only $\calK(z) \ket{0}$ when $\ket{0}$ is identified with the cyclic vector of a finite-dimensional simple module the $\i$quantum group at play. Suppose that $V$ is any such finite-dimensional simple module, say with highest weight $\vec{\lambda}(z) := ( \lambda_1(z), ..., \lambda_N(z) )$. Via the fact that $B(z) \ket{0} = \calK(z) \ket{0}$, we then infer that $\calK_{i, i}(z) \ket{0} = \lambda_i(z) \ket{0}$.
        }

    \subsection{\texorpdfstring{Presentations by Drinfeld currents and Kac-Moody generators for QSPs in $\DY_{\hbar}(\sl_2)$}{}}