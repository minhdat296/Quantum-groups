\input{article preambles}

\setcounter{section}{0}

\input{commands}
\renewcommand{\1}{\boldsymbol{\delta}}

\begin{document}

    \title{
        Quantum symmetric pairs in $\calU_q(\Loop\sl_2)$ and $\DY(\sl_2)$
        \\
        ---
        \\
        Plan
    }
    
    \author{Dat Minh Ha}
    \maketitle

    \begin{abstract}
        This is a proof of concept for an intended solution to the original question of defining and studying twisted $q$-Yangians associated to (pseudo-)symmetric pairs of types $\sfB, \sfC, \sfD$. I also realised that many details are still missing in the type $\sfA$ case as well, so I thought that it was worth spending some time with this case, even if it is to serve just as a prototype for our eventual goal.
        
        This document is meant only as an outline of the approach that I intend to pursue, so there is some trade-off between technical precision and conciseness. 
    \end{abstract}
    
    {
    \hypersetup{} 
    %\dominitoc
    \tableofcontents %sort sections alphabetically
    }

    \section{Introduction}
        \subsection{Notations}
            Given any formal distribution:
                $$a(z) := \sum_{n \in \Z} a_n z^{-n}$$
            let us write:
                $$a(z)^- := \sum_{n > 0} a_m z^{-n} \quad, \quad a(z)^+ := \sum_{n \leq 0} a_n z^{-n}$$
            for the negative and positive halves, respectively, with the indexing sets for the two sub-series being chosen so that we would have:
                $$(\del_z a(z))^{\pm} = \del_z( a(z)^{\pm} ) \quad, \quad \del_z := \frac{d}{dz}$$
            These notations will usually be used for generating series. Also, for the formal Dirac distribution around $z = w$, we write:
                $$\1(zw^{-1}) := \sum_{n \in \Z} z^n w^{-n} \in \bbC[\![z^{\pm 1}, w^{\pm 1}]\!]$$
            (cf. \cite[p. 65]{jimbo_miwa_algebraic_analysis_of_solvable_lattice_models}), and it is useful to note that in the region $\abs{z} < \abs{w}$, we have $\1(zw^{-1}) := \sum_{n > 0} z^n w^{-n} = -1 + \frac{1}{1 - zw^{-1}} = \frac{z}{z - w}$, while in the region $\abs{z} > \abs{w}$, we have $\1(zw^{-1})^+ := \sum_{n \leq 0} z^n w^{-n} = \frac{1}{1 - z^{-1}w} = \frac{z}{z - w}$.

        \subsection{Quantum algebras}
            Quantum algebras that arise as quantum doubles or (reasonably minimal) quotients thereof, such as quantum Kac-Moody algebras, quantum affine algebras, quantum loop algebras, (extended) double Yangians, will sometimes be referred to collectively as \say{affine quantum algebras} or something of the sort; this is because there are constructions that will work identically between the trigonometric and rational cases and will only depend qualitatively on the R-matrix; I hope context will make things clear down below.

            Trigonometric R-matrices and K-matrices will usually depend multiplicatively on spectral parameters $z, w$, while rational R-matrices will depend additively on spectral parameters $u := \log(z), v := \log(w)$.
        
            For us, $\hbar$ shall be a formal variable for the moment. We write $q := e^{\hbar}$, and when $\hbar$ is specialised to numerical values, we assume that $\hbar \in \bbC \setminus 2\pi i \Q$ so that $q$ is not a root of unity. We will also be needing the following $q$-number expressions:
                $$[n]_q := \frac{q^n - q^{-n}}{q - q^{-1}} \quad, \quad [n]_q! := \prod_{k = 1}^n [k]_q$$
                $$\exp_q(z) := \sum_{n \geq 0} \frac{1}{[n]_q!} z^n$$
            They are to be interpreted as elements of $\bbC[q^{\pm 1}]$.
        
            Suppose that $\calR_{\trigonometric}(zw^{-1})$ is a trigonometric solution to the quantum Yang-Baxter equation (QYBE) with rational limit $\calR_{\rational}(u - v)$ with classical limits $\calr_{\trigonometric}(zw^{-1})$ and $\calr_{\rational}(u - v)$ respectively; denote the corresponding quasi-triangular (topological) Hopf algebras as in \cite{etingof_kazhdan_quantisation_1} (see also \cite{etingof_kazhdan_quantisation_1} and \cite{etingof_kazhdan_quantisation_6}) by:
                $$\calU_{\hbar}(\calR)$$
            and their classical limits by:
                $$\g := \g(\calr)$$
            \textit{A priori}, $\g(\calr_{\trigonometric})$ is a Kac-Moody algebra (with derivation), but let us assume further that it is of a untwisted affine type $\sfX$ in the classification of \cite[Chapter 4]{kac_infinite_dimensional_lie_algebras}. Consequently, there exists a finite-dimensional simple Lie algebra $\bar{\g}$ such that:
                $$\g \cong ( \Loop \bar{\g} \oplus \bbC c ) \rtimes \bbC \del$$
            wherein $\Loop \bar{\g} := \g[t^{\pm 1}]$ and $\del$ acts by $\id \tensor t\frac{d}{dt}$. In particular, this means that the loop algebra $\Loop \bar{\g}$ is a Lie sub-bialgebra of $\g(\calr_{\trigonometric})$, and hence the quantum Kac-Moody algebra $\calU_{\hbar}^{\calR}(\g) := \calU_{\hbar}(\calR_{\trigonometric})$ admits the quantum loop algebra $\calU_{\hbar}^{\calR}(\Loop \bar{\g})$ (both in the R-matrix presentation) as a Hopf subalgebra.

            $\calU_{\hbar}^{\calR}(\g)$ comes equipped with representations $\pi_z^{\pm}: \calU_{\hbar}^{\calR}(\g) \to \End(\bbV_N)[\![z^{\pm 1}]\!]$, using which we can write down the defining relations for $\calU_{\hbar}^{\calR}(\g)$, using coefficients of the matrix entries of the following so-called \say{monodromy matrices}:
                $$L^+(z) := (\id \tensor \pi_z^+)( \calR^{-1} )_{2, 1} \in \End(\bbV_N \tensor \bbV_N) \tensor \calU_{\hbar}^{\calR}(\g)[\![z]\!]$$
                $$L^-(z) := (\pi_z^- \tensor \id)( \calR )_{1, 2} \in \End(\bbV_N \tensor \bbV_N) \tensor \calU_{\hbar}^{\calR}(\g)[\![z^{-1}]\!]$$
            along with a central generator $\level$ and a derivation $\del$. The aforementioned relations take place in $\End(\bbV_N \tensor \bbV_N) \tensor \calU_{\hbar}^{\calR}(\g)[\![z^{\pm 1}, w^{\pm 1}]\!]$ and take the following form:
                $$[\level, \calU_{\hbar}^{\calR}(\g)] = 0 \quad, \quad [\del, L^{\pm}(z)] = \frac{d}{dz} L^{\pm}(z)$$
                \begin{equation} \label{equation: RLL_relations}
                    \calR(zw^{-1}) L^{\pm}(z)_1 L^{\pm}(w)_2 = L^{\pm}(w)_2 L^{\pm}(z)_1 \calR(zw^{-1}) \quad, \quad \calR(zw^{-1}) L^+(z)_1 L^-(w)_2 = L^-(w)_2 L^+(z)_1 \calR(zw^{-1})
                \end{equation}
            
            Recall also, that there are two other presentations for the quantum Kac-Moody and quantum loop algebras, namely by Drinfeld currents and Kac-Moody generators; let us denote these algebras by $\calU_{\hbar}^{\current}(\g), \calU_{\hbar}^{\current}(\Loop \bar{\g})$ and $\calU_{\hbar}^{\KM}(\g), \calU_{\hbar}^{\KM}(\Loop \bar{\g})$ respectively. For our purposes here, it is sufficient to know only that $\calU_{\hbar}^{\current}(\g)$ is generated a central generator $q^{\level}$ and a derivation $\del$, along with the coefficients of generating series:
                $$E_i(z) := \sum_{n \in \Z} E_i[n] z^{-n} \quad, \quad F_i(z) := \sum_{n \in \Z} F_i[n] z^{-n} \quad, \quad K^{\pm}_i(z) := \sum_{n \geq 0} K^{\pm}_i[\mp n] z^{\pm n} \quad, \quad i \in \simpleroots$$
            while $\calU_{\hbar}^{\KM}(\g)$ is generated by the same central and derivation elements, along with generators:
                $$E_i, F_i, q^{\pm H_i} \quad, \quad i \in \simpleroots$$
            with $\simpleroots$ denoting a set of simple root for the Kac-Moody algebra $\g$; the quantum loop subalgebras are generated likewise. For more information on the commutation relations between these generators, see e.g. \cite[Section 2]{ding_pakuliak_khoroshkin_integral_formula_for_R_matrices_of_affine_QUEs} or \cite[Section 3]{frenkel_reshetikhin_affine_QUEs_and_deformed_virasoro_and_finite_W_algebras}. 
            
            The three presentations turn out to be equivalent to one another. By theorems of Ding-I. Frenkel and Beck (later on clarified by E. Frenkel and Mukhin), we know that there are algebra isomorphisms:
                $$\calU_{\hbar}^{\calR}(\g) \xrightarrow[]{\DF} \calU_{\hbar}^{\current}(\g) \xrightarrow[]{\Beck} \calU_{\hbar}^{\KM}(\g)$$
            which then get restricted down to isomorphisms:
                $$\calU_{\hbar}^{\calR}(\Loop \bar{\g}) \xrightarrow[]{\cong} \calU_{\hbar}^{\current}(\Loop \bar{\g}) \xrightarrow[]{\cong} \calU_{\hbar}^{\KM}(\Loop \bar{\g})$$
            \begin{remark} \label{remark: gaussian_decompositions_of_monodromy_matrices}
                The existence (and uniqueness) of the first isomorphism $\DF$ relies on the existence (and again, uniqueness) of the Gaussian decomposition of the monodromy matrices $L^{\pm}(z)$, so for the purpose of writing down an explicit formula for $\DF$, \textit{having an explicit formula for the aforementioned Gaussian decomposition is a prerequisite}. For instance, in the case wherein $\g = \hat{\sl}_2$, the following Gaussian decompositions exist in the algebra $\calU_{\hbar}^{\calR}(\hat{\sl}_2)$:
                    \begin{equation} \label{equation: quantum_affine_sl_2_monodromy_matrices_gaussian_decompositions}
                        L^{\pm}(z)
                        =
                        \begin{pmatrix}
                            1 & 0
                            \\
                            e^{\pm}(z) & 1
                        \end{pmatrix}
                        \begin{pmatrix}
                            k_0^{\pm}(z) & 0
                            \\
                            0 & k_1^{\pm}(z)
                        \end{pmatrix}
                        \begin{pmatrix}
                            1 & f^{\pm}(z)
                            \\
                            0 & 1
                        \end{pmatrix}
                        =
                        \begin{pmatrix}
                            k_0^{\pm}(z) & k_0^{\pm}(z) f^{\pm}(z)
                            \\
                            e^{\pm}(z) k_0^{\pm}(z) & k_1^{\pm}(z) + e^{\pm}(z) k_0^{\pm}(z) f^{\pm}(z)
                        \end{pmatrix}
                    \end{equation}
                and consequently, there is an algebra isomorphism:
                \begin{equation} \label{equation: ding_frenkel_isomorphism_quantum_affine_sl_2}
                    \DF: \calU_{\hbar}^{\calR}(\hat{\sl}_2)_{\level} \xrightarrow[]{\cong} \calU_{\hbar}^{\current}(\hat{\sl}_2)_{\level}
                \end{equation}
                determined by:
                    \begin{equation}
                        E(z) = \DF( e^+(z q^{\frac12 \level}) - e^-(z q^{-\frac12 \level}) ) \quad, \quad F(z) = \DF( f^+(z q^{-\frac12 \level}) - f^-(z q^{\frac12 \level}) )
                    \end{equation}
                    \begin{equation}
                        K_i^{\pm}(z) = \DF( k_i^{\pm}(z) )
                    \end{equation}
            \end{remark}

        \subsection{Quantum symmetric pairs}
            First of all, to fix terminologies, let me refer to both quantum symmetric pairs and quantum pseudo-symmetric pairs (in the sense of \cite{regelskis_vlaar_kac_moody_pseudo_symmetric_pairs}) as \say{quantum symmetric pairs} and abbreviate the term by QSP. When there is a need to specify the presentation of the QSPs at play, the QSPs that arise from reflection equations\footnote{... either untransposed or transposed.} (and hence ultimately from K-matrices; e.g. as in \cite{molev_ragoucy_sorba_twisted_q_yangians_type_A} or \cite{guay_regelskis_twisted_yangians_for_symmetric_pairs_of_types_BCD}) shall be referred to as \say{reflection algebras}, while those given by Kac-Moody-style generators satisfying Chevalley-Serre-type relations and those given by means of Drinfeld currents (such as the QSPs in \cite{regelskis_vlaar_reflection_matrices_coideal_subalgebras}, and in \cite{lu_wang_drinfeld_current_presentation_for_affine_iQUEs_1} and \cite{zhang_drinfeld_current_presentation_for_affine_iQUEs_2}) shall be referred to as \say{$\i$quantum groups}. This distinction is needed because the latter type does not seem to have an agreed-upon name in the literature, though the term \say{$\i$quantum group} seems to be the next best thing. Otherwise, I will keep to the standard terminologies, e.g. \say{twisted $q$-Yangians}, \say{twisted Yangians}, etc.
        
            Next, suppose that $\calK_{\trigonometric}(z)$ and $\calK_{\rational}(u)$ respectively are solutions to the boundary quantum Yang-Baxter equation (bQYBE) given using the R-matrices fixed previously, and then denote the reflection algebras arising from these K-matrices by:
                $$\calB_{\hbar}(\calK)$$
            Suppose that $\vartheta$ is the (pseudo-)involutive automorphism on $\g(\calr)$ such that the classical limit of the reflection algebra above is the fixed point subalgebra:
                $$(\Loop \bar{\g})^{\vartheta}$$
            Moreover, $\calB_{\hbar}(\calK)$ is \textit{a priori} a right-coideal subalgebra of the quantum loop algebra $\calU_{\hbar}^{\calR}(\Loop \bar{\g})$, hence also a right-coideal subalgebra of $\calU_{\hbar}^{\KM}(\Loop \bar{\g})$ in particular.

            At the same time, the automorphism $\vartheta$ induces an automorphism on the Dynkin diagram of type $\sfX$, and hence . Thus, there exists an induced automorphism on the set of Kac-Moody-style generators of $\calU_{\hbar}^{\KM}(\Loop \bar{\g})$, and therefore there exists an $\i$quantum group associated to the aforementioned automorphism $\vartheta$, which we denote by:
                $$\calU_{\hbar}^{\KM}(\Loop \bar{\g}, \vartheta)$$
            
        \subsection{Goal}
            Ideally, we would like to be able to show that the algebras:
                $$\calB_{\hbar}(\calK) \quad \text{and} \quad \calU_{\hbar}^{\KM}(\Loop \bar{\g}, \vartheta)$$
            are isomorphic for \textit{all} affine types $\sfX$, even the twisted types (in which case the loop algebra $\Loop \bar{\g}$ will have to be replaced by an twisted version, but we will worry about this later). However, we will focus on the case when:
                $$\bar{\g} = \sl_2$$
            for now, as a proof of concept. In particular, we will be focusing on the cases when the K-matrix is of type $\sfA.1, \sfA.2, \sfA.3$ and thus are diagonal, and the K-matrix of the $q$-Onsager algebra (corresponding to when $\vartheta$ is the Chevalley involution), which is the simplest non-diagonal K-matrix.
            \begin{remark}
                On \cite[p. 60]{regelskis_vlaar_reflection_matrices_coideal_subalgebras}, it has been conjectured that when $\sfX$ is an \textit{untwisted} affine type, all solutions $\calK_{\trigonometric}(z)$ of the untransposed and transposed bQYBEs are diagonalisable. The strategy outlined below depends heavily on this conjecture being true. At the moment, I do not have any idea of how to deal with non-diagonalisable K-matrices.

                Additionally, the aforementioned conjecture on the diagonalisability of K-matrices depends on \cite[Theorem 9.3]{regelskis_vlaar_reflection_matrices_coideal_subalgebras}, which listed explicit formulae for K-matrices in all affine types. While it is stated as a theorem, this result actually remains a conjecture for spin chains with $N > 15$ particles, and was verified numerically using \url{Mathematica} for $1 \leq N \leq 15$ (cf. Remark 9.4 of \textit{loc. cit.}). For this reason, we have deliberately avoided relying on the explicit formulae for the K-matrices in our approach. Instead, we are concerned only with the property of being diagonalisable, which we believe to be a weaker conjecture, and should be true for a fairly large class of K-matrices anyway.
            \end{remark}

    \section{Strategy}
        From now on, we focus on the case of the untwisted affine Kac-Moody algebra $\g = \hat{\sl}_2$ with underlying finite-type Lie algebra $\bar{\g} = \sl_2$.

        \subsection{Free field realisations for affine quantum algebras in Drinfeld's current presentation}
            It is well-known that the generators of $\calU_{\hbar}^{\current}(\hat{\sl}_2)$ - and hence $\calU_{\hbar}^{\current}(\Loop \sl_2)$ too - admits a realisation in terms of so-called free fields; see \cite{awata_odake_shiraishi_free_boson_realisation_of_quantum_affine_sl_N} and \cite[Section 7]{frenkel_reshetikhin_affine_QUEs_and_deformed_virasoro_and_finite_W_algebras} (and also \cite[Chapter 5]{jimbo_miwa_algebraic_analysis_of_solvable_lattice_models}), and in essence, this means that we are constructing an embedding:
                $$\freeboson^{\current}_{\hbar, \level}: \calU_{\hbar}^{\current}(\hat{\sl}_2)_{\level} \to \calA_{\hbar}^{\current}(\hat{\sl}_2)_{\level}$$
            of the quantum Kac-Moody algebra into a \say{deformed oscillator algebra} (in a presentation by Drinfeld-style currents), both at a \say{level} $\level \in \bbC$; the subscript $\level$ indicates that on the Fock spaces of the two algebras, the central generator of $\calU_{\hbar}^{\current}(\hat{\sl}_2)$ and its image inside the deformed oscillator algebra act as the scalar $q^{\level}$; more on this later. This deformed oscillator algebra $\calA_{\hbar}^{\current}(\hat{\sl}_2)_{\level}$ is the completion with respect to the natural $\Z$-grading of the the associative $\bbC[\![\hbar]\!]$-algebra generated by the elements:
                $$\lambda[n], b[n], c[n] \quad, \quad n \in \Z$$
                $$\beta, \gamma$$
            which are subjected to the following relations, given for all $m, n \in \Z \setminus \{0\}$:
                \begin{equation} \label{equation: q_deformed_affine_sl_2_oscillator_lambda_commutators}
                    [ \lambda[m], \lambda[n] ] = \frac1n \frac{ [(\level + \dualcoxeter) n]_q [n]_q^2 }{ [2n]_q } \delta_{m + n, 0}
                \end{equation}
                
                \begin{equation} \label{equation: q_deformed_affine_sl_2_oscillator_bc_commutators}
                    [ b[m], b[n] ] = -\frac1n [n]_q^2 \delta_{m + n, 0} \quad, \quad [ c[m], c[n] ] = \frac1n [n]_q^2 \delta_{m + n, 0}
                \end{equation}
                
                \begin{equation} \label{equation: q_deformed_affine_sl_2_oscillator_beta_gamma_commutators}
                    [ b[0], \beta ] = -\frac{q - q^{-1}}{2\hbar} \quad, \quad [ c[0], \gamma ] = \frac{q - q^{-1}}{2\hbar}
                \end{equation}
            in which $\dualcoxeter = -2$ is the dual Coxeter number, and:
                $$\lambda(z) = \sum_{n \in \Z} \lambda[n] z^{-n}$$
                $$
                    b(z) = \sum_{n \in \Z} b[n] z^{-n} \quad, \quad c(z) = \sum_{n \in \Z} c[n] z^{-n}
                $$
            \begin{remark}[Regarding notations from \cite{frenkel_reshetikhin_affine_QUEs_and_deformed_virasoro_and_finite_W_algebras} and \cite{awata_odake_shiraishi_free_boson_realisation_of_quantum_affine_sl_N}]
                In \cite{frenkel_reshetikhin_affine_QUEs_and_deformed_virasoro_and_finite_W_algebras}, the elements $\beta, \gamma$ are denoted by $p_b, p_c$ respectively, and the series $B^{\pm}(z), C^{\pm}(z), \Beta(z)$, and $\Gamma(z)$ are denoted by $b^{\pm}(z), c^{\pm}(z), b(z)$, and $c(z)$ respectively. Also, in comparison with the generators $a[n]$ from \cite{awata_odake_shiraishi_free_boson_realisation_of_quantum_affine_sl_N}, the generators $\lambda[n]$ differ by $\lambda[n] = \frac{q - q^{-1}}{q^n + q^{-n}} a[n]$. This leads to $\Lambda^{\pm}(z) \Lambda^{\pm}(z q^{\pm 2}) = \exp\left( a^{\pm}( z q^{\pm 1} ) \right)$. 
            \end{remark}
                
            We can also form the following alternate generating series for the algebra $\calA_{\hbar}^{\current}(\hat{\sl}_2)_{\level}$, which are more convenient for writing down an explicit formula for $\freeboson^{\current}_{\hbar, \level}$:
                \begin{equation} \label{equation: q_deformed_affine_sl_2_oscillator_Lambda_generating_series}
                    \Lambda^{\pm}(z) := e^{ \pm \frac{\lambda_0}{2} } e^{ \pm \lambda(z)^{\pm} }
                \end{equation}
                
                \begin{equation} \label{equation: q_deformed_affine_sl_2_oscillator_BC_generating_series}
                    B^{\pm}(z) := \pm (q - q^{-1}) \left( \frac12 b[0] + b(z)^{\pm} \right)
                    \quad, \quad
                    C^{\pm}(z) := \pm (q - q^{-1}) \left( \frac12 c[0] + c(z)^{\pm} \right)
                \end{equation}
                
                \begin{equation} \label{equation: quatum_affine_sl_2_oscillator_Theta_Gamma_generating_series}
                    \Beta(z) := \beta + \frac{(q - q^{-1}) b_0}{2\hbar} \log(z) + \sum_{n \in \Z \setminus \{0\}} \frac{b[n]}{[n]_q} z^{-n}
                    \quad, \quad
                    \Gamma(z) := \gamma + \frac{(q - q^{-1}) c_0}{2\hbar} \log(z) + \sum_{n \in \Z \setminus \{0\}} \frac{c[n]}{[n]_q} z^{-n}
                \end{equation}
            It was then noted in \cite{frenkel_reshetikhin_affine_QUEs_and_deformed_virasoro_and_finite_W_algebras} that the following relations between the series in \eqref{equation: q_deformed_affine_sl_2_oscillator_Lambda_generating_series}, \eqref{equation: q_deformed_affine_sl_2_oscillator_BC_generating_series}, and \eqref{equation: quatum_affine_sl_2_oscillator_Theta_Gamma_generating_series} satisfy the following commutation relation\footnote{... wherein $g$ is the normalising factor for the R-matrix $\calR$.}:
                \begin{equation} \label{equation: q_deformed_affine_sl_2_oscillator_Lambda_generating_series_commutators}
                    \Lambda^+(z) \Lambda^-(w) = \frac{ g\left( q^{-\level - 2} \frac{w}{z} \right) }{ g\left( q^{\level + 2} \frac{w}{z} \right) } \Lambda^-(w) \Lambda^+(z)
                \end{equation}
                
                \begin{equation} \label{equation: q_deformed_affine_sl_2_oscillator_BC_generating_series_commutators}
                    e^{ B^+(z) } \order{ e^{\Beta(w)} } = \frac{ 1 - \frac{w}{z} q }{ q - \frac{w}{z} } \order{ e^{\Beta(w)} } e^{ B^+(z) }
                    \quad, \quad
                    \order{ e^{ \Beta(z) } } e^{ B^-(w) } = \frac{ 1 - \frac{w}{z} q }{ q - \frac{w}{z} } e^{B^-(w)} \order{ e^{ \Beta(z) } }
                \end{equation}
                
                \begin{equation} \label{equation: quatum_affine_sl_2_oscillator_Theta_Gamma_generating_series_commutators}
                    e^{B^+(z)} e^{B^-(w)} = \frac{ \left(1 - \frac{w}{z}\right)^2 }{ \left( 1 - \frac{w}{z} q^2 \right) \left( 1 - \frac{w}{z} q^{-2} \right) } e^{B^-(w)} e^{B^+(z)}
                    \quad, \quad
                    e^{C^+(z)} e^{C^-(w)} = \frac{ \left( 1 - \frac{w}{z} q^2 \right) \left( 1 - \frac{w}{z} q^{-2} \right) }{ \left(1 - \frac{w}{z}\right)^2 } e^{C^-(w)} e^{C^+(z)}
                \end{equation}

            \begin{remark}
                At this point, it is interesting - and indeed, important - to note that while the relations between the series $e^{\Lambda^{\pm}(z)}, e^{B^{\pm}(z)}, e^{C^{\pm}(z)}, e^{\Beta(z)}, e^{\Gamma(z)}$ as described above indicate that they do not commute, what determine this non-commutativity are precisely the entries of the (normalised) universal R-matrix of $\calU_{\hbar}(\hat{\sl}_2)_{\level}$. We will revisit this observation in subsection \ref{subsection: free_field_realisations_for_R_matrix_affine_QUEs}.
            \end{remark}
            
            In the classical limit $\hbar \to 0$, the deformed oscillator algebra $\calA_{\hbar}(\hat{\sl}_2)_{\level}$ becomes the \say{classical oscillator algebra} (at the same level $\level \in \bbC$), which is the associative $\bbC$-algebra generated by the coefficients of the series:
                $$\chi(z) := \sum_{n \in \Z} \chi[n] z^{-n}$$
                $$a(z) := \sum_{n \in \Z} a[n] z^{-n} \quad, \quad a^{\dagger}(z) := \sum_{n \in \Z} a^{\dagger}[n] z^{-n}$$
            which obeys the classical Heisenberg OPE (cf. \cite[Subsection 2.3.2]{frenkel_ben_zvi_vertex_algebras_and_algebraic_curves}):
                \begin{equation} \label{equation: classical_heisenberg_OPE}
                    [\chi(z), \chi(w)] \sim 2 \level \1(zw^{-1})
                \end{equation}
            while the series $b(z)$ and $c(z)$ will become $a(z)$ and $z^{\dagger}(z)$ respectively, between which there are the following OPE relations:
                \begin{equation} \label{equation: classical_weyl_OPE}
                    [a(z), a(w)] = [a^{\dagger}(z), a^{\dagger}(w)] = 0, \quad, \quad [a(z), a^{\dagger}(w)] \sim \level \1(zw^{-1})
                \end{equation}
            Thus, $\lambda(z)^{\pm}$ should appear in the free boson realisations of the Cartan-like currents $K^{\pm}(z)$ of $\calU_{\hbar}(\hat{\sl}_2)_{\level}$, while $b(z)$ and $c(z)$ should appear in the free boson realisations of the \say{upper and lower triangular} currents $E(z), F(z)$. The idea behind the construction of the free boson realisation $\freeboson^{\current}_{\hbar, \level}$ is thus, that $\calU_{\hbar}(\hat{\sl}_2)_{\level}$ can be embedded into $\calA_{\hbar}^{\current}(\hat{\sl}_2)_{\level}$ because the latter is generated by the same number of similarly behaving generating series, but those are subjected to fewer relations than the generating series of the former, and thus $\calA_{\hbar}^{\current}(\hat{\sl}_2)_{\level}$ is \say{sufficiently large}. Indeed, we have the following.
            \begin{proposition}[Free boson realisation of $\calU_{\hbar}^{\current}(\hat{\sl}_2)$] \label{prop: free_boson_realisation_quantum_affine_sl_2_drinfeld_current_presentation}
                There exists an \textit{injective} algebra homomorphism:
                    \begin{equation} \label{equation: free_boson_realisation_quantum_affine_sl_2_drinfeld_current_presentation}
                        \freeboson^{\current}_{\hbar, \level}: \calU_{\hbar}^{\current}(\hat{\sl}_2)_{\level} \to \calA_{\hbar}^{\current}(\hat{\sl}_2)_{\level}
                    \end{equation}
                that is given on Drinfeld currents by:
                    \begin{equation}
                        \freeboson^{\current}_{\hbar, \level}( E(z) ) = -\order{  }
                    \end{equation}
                    \begin{equation}
                        \freeboson^{\current}_{\hbar, \level}( F(z) ) =
                    \end{equation}
                    \begin{equation}
                        \freeboson^{\current}_{\hbar, \level}( K_i^+(z) ) = \quad, \quad \freeboson^{\current}_{\hbar, \level}( K_i^-(z) ) =
                    \end{equation}
            \end{proposition}

            \begin{remark}
                There are two Cartan-like currents $K^{\pm}(z)$ amongst the generating series for $\calU_{\hbar}(\hat{\sl}_2)_{\level}$, and hence two similarly behaving generating series $\lambda(z)^{\pm}$ for the deformed oscillator algebra $\calA_{\hbar}^{\current}(\hat{\sl}_2)_{\level}$, because $\calU_{\hbar}(\hat{\sl}_2)$ arises as a quantum double. In particular, in order to construct the underlying Lie bialgebra structure on the affine Kac-Moody algebra $\hat{\sl}_2$, it is necessary to \say{double} the Cartan subalgebra so that a non-degenerate invariant bilinear form could be constructed on the resulting enlargement of $\hat{\sl}_2$, and thus a Manin triple can be constructed. See \cite{etingof_kazhdan_quantisation_6} for more details. 
            \end{remark}

            \begin{remark}
                The factor of $2$ appears in equation \eqref{equation: classical_heisenberg_OPE} because we can choose a symmetric, non-degenerate, and invariant bilinear form $(\cdot, \cdot)_{\sl_2}$ on $\sl_2$ so that $( \chi[m], \chi[n] )_{\sl_2} = 2 \delta_{m + n, 0}$ for all $m, n \in \Z$. In general, it is important to keep track of such a bilinear form on $\bar{\g}$, since it determines the Cartan matrix of $\bar{\g}$, and hence of its affinisastion $\g$ (see \cite[Chapter 7]{kac_infinite_dimensional_lie_algebras}).
            \end{remark}

        \subsection{Free field realisations for affine quantum algebras in the R-matrix presentation} \label{subsection: free_field_realisations_for_R_matrix_affine_QUEs}
            In principle, it should be possible to compute the images of the monodromy matrices $L^{\pm}(z)$ under the composition:
                $$\calU_{\hbar}^{\calR}(\hat{\sl}_2)_{\level} \xrightarrow[]{\DF} \calU_{\hbar}^{\current}(\hat{\sl}_2)_{\level} \xrightarrow[]{\freeboson_{\hbar, \level}} \calA_{\hbar}(\hat{\sl}_2)_{\level}$$
            However, there is a practical issue, namely that $\DF$ (as in equation \eqref{equation: ding_frenkel_isomorphism_quantum_affine_sl_2}) is given by specifying the pre-images\footnote{The advantage of this formulation is that it makes it easy to write down an explicit formula for the inverse map $\DF^{-1}$.} of the Drinfeld currents $E(z), F(z), K^{\pm}(z)$. Therefore, we must first write down a free field realisation for $\calU_{\hbar}^{\calR}(\hat{\sl}_2)_{\level}$, entirely in terms of the universal R-matrix, and then compare the two free field realisations.

            The first construction that we shall require is that of deformed oscillator algebras, depending only on R-matrices; we note right away, that any solution to the QYBE equation, constant or dynamic, can be used as an input for this construction. Following \cite{ragoucy_vertex_representations_of_R_matrix_quantum_groups}, we define the deformed oscillator algebra\footnote{In \cite{ragoucy_vertex_representations_of_R_matrix_quantum_groups}, $\calA_{\hbar}(\calR)_{\level}$ is called a \say{Zamolodchikov-Faddeev (ZF) algebra}.} associated to $\calR$ at level $\level \in \bbC$:
                $$\calA_{\hbar}(\calR)_{\level}$$
            to be the associative algebra equipped with representations $\rho_z^{\pm}: \calA_{\hbar}(\calR) \to \End(\bbV_N)[\![z^{\pm 1}]\!]$, using which we can write down the defining relations for $\calA_{\hbar}(\calR)$ in terms of the coefficients of the matrix entries of the following so-called \say{creation and annihilation operators}:
                $$a^+(z) := (\id \tensor \rho_z^+)( \calR^{-1} )_{2, 1} \in \End(\bbV_N \tensor \bbV_N) \tensor \calA_{\hbar}(\calR)[\![z]\!]$$
                $$a^-(z) := (\rho_z^- \tensor \id)( \calR )_{1, 2} \in \End(\bbV_N \tensor \bbV_N) \tensor \calA_{\hbar}(\calR)[\![z^{-1}]\!]$$
            along with a central generator $\level$. The aforementioned relations take the following form inside $\End(\bbV_N \tensor \bbV_N) \tensor \calA_{\hbar}(\calR)[\![z^{\pm 1}, w^{\pm 1}]\!]$:
                $$[\level, \calA_{\hbar}(\calR)_{\level}] = 0$$
                \begin{equation} \label{equation: deformed_oscillator_algebra_RLL_relations}
                    a^{\pm}(z) a^{\pm}(w) - \calR(z, w)_{2, 1} a^{\pm}(z) a^{\pm}(w) \quad, \quad a^+(z) L^-(w) - \calR(z, w)_{2, 1} a^-(z) a^+(w) - \1(zw^{-1})
                \end{equation}
    
        \subsection{Free field realisations for QSPs}

        \subsection{\texorpdfstring{Comparing reflection algebras and $\i$quantum groups}{}}

    \section{Future works}
    
    \addcontentsline{toc}{section}{References}
    \printbibliography

\end{document}