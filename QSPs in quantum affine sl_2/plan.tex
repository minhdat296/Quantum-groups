\documentclass[a4paper, 11pt]{article}

%\usepackage[center]{titlesec}

\usepackage{amsfonts, amssymb, amsmath, amsthm, amsxtra}

%\usepackage{foekfont}

\usepackage{MnSymbol}

\usepackage{pdfrender, xcolor}
%\pdfrender{StrokeColor=black,LineWidth=.4pt,TextRenderingMode=2}

%\usepackage{minitoc}
%\setcounter{section}{-1}
%\setcounter{tocdepth}{}
%\setcounter{minitocdepth}{}
%\setcounter{secnumdepth}{}

\usepackage{graphicx}

\usepackage[english]{babel}
\usepackage[utf8]{inputenc}
%\usepackage{mathpazo}
\usepackage{dutchcal}
%\usepackage{eucal}
\usepackage{eufrak}
\usepackage{bbm}
\usepackage{bm}
\usepackage{csquotes}
\usepackage[nottoc]{tocbibind}
\usepackage{appendix}
\usepackage{float}
\usepackage[T1]{fontenc}
\usepackage[
    left = \flqq{},% 
    right = \frqq{},% 
    leftsub = \flq{},% 
    rightsub = \frq{} %
]{dirtytalk}

\usepackage{imakeidx}
\makeindex

%\usepackage[dvipsnames]{xcolor}
\usepackage{hyperref}
    \hypersetup{
        colorlinks=true,
        linkcolor=teal,
        filecolor=pink,      
        urlcolor=teal,
        citecolor=magenta
    }
\usepackage{comment}

% You would set the PDF title, author, etc. with package options or
% \hypersetup.

\usepackage[backend=biber, style=alphabetic, sorting=nty]{biblatex}
    \addbibresource{bibliography.bib}
\renewbibmacro{in:}{}

\raggedbottom

\usepackage{mathrsfs}
\usepackage{mathtools} 
\mathtoolsset{showonlyrefs} 
%\usepackage{amsthm}
\renewcommand\qedsymbol{$\blacksquare$}

\usepackage{tikz-cd}
\tikzcdset{scale cd/.style={every label/.append style={scale=#1},
    cells={nodes={scale=#1}}}}
\usepackage{tikz}
\usepackage{setspace}
\usepackage[version=3]{mhchem}
\parskip=0.1in
\usepackage[margin=25mm]{geometry}

\usepackage{listings, lstautogobble}
\lstset{
	language=matlab,
	basicstyle=\scriptsize\ttfamily,
	commentstyle=\ttfamily\itshape\color{gray},
	stringstyle=\ttfamily,
	showstringspaces=false,
	breaklines=true,
	frameround=ffff,
	frame=single,
	rulecolor=\color{black},
	autogobble=true
}

\usepackage{todonotes,tocloft,xpatch,hyperref}

% This is based on classicthesis chapter definition
\let\oldsec=\section
\renewcommand*{\section}{\secdef{\Sec}{\SecS}}
\newcommand\SecS[1]{\oldsec*{#1}}%
\newcommand\Sec[2][]{\oldsec[\texorpdfstring{#1}{#1}]{#2}}%

\newcounter{istodo}[section]

% http://tex.stackexchange.com/a/61267/11984
\makeatletter
%\xapptocmd{\Sec}{\addtocontents{tdo}{\protect\todoline{\thesection}{#1}{}}}{}{}
\newcommand{\todoline}[1]{\@ifnextchar\Endoftdo{}{\@todoline{#1}}}
\newcommand{\@todoline}[3]{%
	\@ifnextchar\todoline{}
	{\contentsline{section}{\numberline{#1}#2}{#3}{}{}}%
}
\let\l@todo\l@subsection
\newcommand{\Endoftdo}{}

\AtEndDocument{\addtocontents{tdo}{\string\Endoftdo}}
\makeatother

\usepackage{lipsum}

%   Reduce the margin of the summary:
\def\changemargin#1#2{\list{}{\rightmargin#2\leftmargin#1}\item[]}
\let\endchangemargin=\endlist 

%   Generate the environment for the abstract:
%\newcommand\summaryname{Abstract}
%\newenvironment{abstract}%
    %{\small\begin{center}%
    %\bfseries{\summaryname} \end{center}}

\newtheorem{theorem}{Theorem}[section]
    \numberwithin{theorem}{subsection}
\newtheorem{proposition}{Proposition}[section]
    \numberwithin{proposition}{subsection}
\newtheorem{lemma}{Lemma}[section]
    \numberwithin{lemma}{subsection}
\newtheorem{claim}{Claim}[section]
    \numberwithin{claim}{subsection}
\newtheorem{question}{Question}[section]
    \numberwithin{question}{subsection}

\theoremstyle{definition}
    \newtheorem{definition}{Definition}[section]
        \numberwithin{definition}{subsection}

\theoremstyle{remark}
    \newtheorem{remark}{Remark}[section]
        \numberwithin{remark}{subsection}
    \newtheorem{example}{Example}[section]
        \numberwithin{example}{subsection}    
    \newtheorem{convention}{Convention}[section]
        \numberwithin{convention}{subsection}
    \newtheorem{corollary}{Corollary}[section]
        \numberwithin{corollary}{subsection}

\numberwithin{equation}{section}

\setcounter{section}{0}

\renewcommand{\implies}{\Rightarrow}
\renewcommand{\cong}{\simeq}
\newcommand{\codim}{\operatorname{codim}}
\newcommand{\ladjoint}{\dashv}
\newcommand{\radjoint}{\vdash}
\newcommand{\<}{\langle}
\renewcommand{\>}{\rangle}
\newcommand\bra[2][]{#1\langle {#2} #1\rvert}
\newcommand\ket[2][]{#1\lvert {#2} #1\rangle}
\newcommand\abs[2][]{#1\left\lvert {#2} #1\right\rvert}
\newcommand\norm[2][]{#1\left\| {#2} #1\right\|}
\newcommand\order[2][]{#1\: \mathbf{:} \: {#2} #1 \: \mathbf{:} \:}
\newcommand{\ndiv}{\hspace{-2pt}\not|\hspace{5pt}}
\newcommand{\cond}{\blacktriangle}
\newcommand{\decond}{\triangle}
\newcommand{\solid}{\blacksquare}
\newcommand{\ot}{\leftarrow}
\renewcommand{\-}{\text{-}}
\renewcommand{\mapsto}{\leadsto}
\renewcommand{\leq}{\leqslant}
\renewcommand{\geq}{\geqslant}
\renewcommand{\setminus}{\smallsetminus}
\newcommand{\punc}{\overset{\circ}}
\renewcommand{\div}{\operatorname{div}}
\newcommand{\grad}{\operatorname{grad}}
\newcommand{\curl}{\operatorname{curl}}
\makeatletter
\DeclareRobustCommand{\cev}[1]{%
  {\mathpalette\do@cev{#1}}%
}
\newcommand{\do@cev}[2]{%
  \vbox{\offinterlineskip
    \sbox\z@{$\m@th#1 x$}%
    \ialign{##\cr
      \hidewidth\reflectbox{$\m@th#1\vec{}\mkern4mu$}\hidewidth\cr
      \noalign{\kern-\ht\z@}
      $\m@th#1#2$\cr
    }%
  }%
}
\makeatother

\newcommand{\N}{\mathbb{N}}
\newcommand{\Z}{\mathbb{Z}}
\newcommand{\Q}{\mathbb{Q}}
\newcommand{\R}{\mathbb{R}}
\newcommand{\bbC}{\mathbb{C}}
\newcommand{\bbK}{\mathbb{K}}
\NewDocumentCommand{\x}{e{_^}}{%
  \mathbin{\mathop{\times}\displaylimits
    \IfValueT{#1}{_{#1}}
    \IfValueT{#2}{^{#2}}
  }%
}
\NewDocumentCommand{\pushout}{e{_^}}{%
  \mathbin{\mathop{\sqcup}\displaylimits
    \IfValueT{#1}{_{#1}}
    \IfValueT{#2}{^{#2}}
  }%
}
\newcommand{\simpleroots}{\mathsf{\Delta}}
\newcommand{\rootsystem}{\mathsf{\Phi}}
\newcommand{\rootlattice}{\mathsf{Q}}
\newcommand{\weight}{\mathsf{\Pi}}
\newcommand{\weightlattice}{\mathsf{\Lambda}}
\newcommand{\Weylgroup}{\mathsf{W}}
\newcommand{\supp}{\operatorname{supp}}
\newcommand{\domain}{\operatorname{dom}}
\newcommand{\codomain}{\operatorname{codom}}
\newcommand{\im}{\operatorname{im}}
\newcommand{\coim}{\operatorname{coim}}
\newcommand{\coker}{\operatorname{coker}}
\newcommand{\id}{\mathrm{id}}
\newcommand{\chara}{\operatorname{char}}
\newcommand{\trdeg}{\operatorname{trdeg}}
\newcommand{\rank}{\operatorname{rank}}
\newcommand{\trace}{\operatorname{tr}}
\newcommand{\qdet}{\operatorname{qdet}}
\newcommand{\sklyanindet}{\operatorname{sdet}} %sklyanin determinant
\newcommand{\quasidet}{\operatorname{quasidet}}
\newcommand{\pfaff}{\operatorname{Pf}} %pfaffian
\newcommand{\qpfaff}{\operatorname{qPf}} %quantum pfaffian
\newcommand{\sklyaninpfaff}{\operatorname{sPf}} %twisted quantum pfaffian
\newcommand{\length}{\operatorname{length}}
\newcommand{\height}{\operatorname{ht}}
\renewcommand{\span}{\operatorname{span}}
\newcommand{\e}{\epsilon}
\newcommand{\p}{\mathfrak{p}}
\newcommand{\q}{\mathfrak{q}}
\newcommand{\m}{\mathfrak{m}}
\newcommand{\n}{\mathfrak{n}}
\newcommand{\calF}{\mathcal{F}}
\newcommand{\calG}{\mathcal{G}}
\newcommand{\calO}{\mathcal{O}}
\newcommand{\F}{\mathbb{F}}
\DeclareMathOperator{\lcm}{lcm}
\newcommand{\gr}{\operatorname{gr}}
\newcommand{\vol}{\mathrm{vol}}
\newcommand{\ord}{\operatorname{ord}}
\newcommand{\projdim}{\operatorname{proj.dim}}
\newcommand{\injdim}{\operatorname{inj.dim}}
\newcommand{\flatdim}{\operatorname{flat.dim}}
\newcommand{\globdim}{\operatorname{glob.dim}}
\renewcommand{\Re}{\operatorname{Re}}
\renewcommand{\Im}{\operatorname{Im}}
\newcommand{\sgn}{\operatorname{sgn}}
\newcommand{\coad}{\operatorname{coad}}
\newcommand{\ch}{\operatorname{ch}} %characters of representations
\newcommand{\dist}{\operatorname{dist}} %distance
\newcommand{\level}{\boldsymbol{c}} %level
\newcommand{\critlevel}{\boldsymbol{h}^{\vee}} %dual coxeter number

\newcommand{\Ad}{\mathrm{Ad}}
\newcommand{\GL}{\mathrm{GL}}
\newcommand{\SL}{\mathrm{SL}}
\newcommand{\PGL}{\mathrm{PGL}}
\newcommand{\PSL}{\mathrm{PSL}}
\newcommand{\Sp}{\mathrm{Sp}}
\newcommand{\GSp}{\mathrm{GSp}}
\newcommand{\GSpin}{\mathrm{GSpin}}
\newcommand{\rmO}{\mathrm{O}}
\newcommand{\SO}{\mathrm{SO}}
\newcommand{\SU}{\mathrm{SU}}
\newcommand{\rmU}{\mathrm{U}}
\newcommand{\rmH}{\mathrm{H}}
\newcommand{\rmY}{\mathrm{Y}} %yangian
\newcommand{\rmX}{\mathrm{X}} %extended yangian
\newcommand{\rmu}{\mathrm{u}}
\newcommand{\rmV}{\mathrm{V}}
\newcommand{\gl}{\mathfrak{gl}}
\renewcommand{\sl}{\mathfrak{sl}}
\newcommand{\diag}{\mathfrak{diag}}
\newcommand{\pgl}{\mathfrak{pgl}}
\newcommand{\psl}{\mathfrak{psl}}
\newcommand{\fraksp}{\mathfrak{sp}}
\newcommand{\gsp}{\mathfrak{gsp}}
\newcommand{\gspin}{\mathfrak{gspin}}
\newcommand{\frako}{\mathfrak{o}}
\newcommand{\so}{\mathfrak{so}}
\newcommand{\su}{\mathfrak{su}}
\newcommand{\Spec}{\operatorname{Spec}}
\newcommand{\Spf}{\operatorname{Spf}}
\newcommand{\Spm}{\operatorname{Spm}}
\newcommand{\Spv}{\operatorname{Spv}}
\newcommand{\Spa}{\operatorname{Spa}}
\newcommand{\Spd}{\operatorname{Spd}}
\newcommand{\Proj}{\operatorname{Proj}}
\newcommand{\Gr}{\mathrm{Gr}}
\newcommand{\Hecke}{\mathrm{Hecke}}
\newcommand{\Sht}{\mathrm{Sht}}
\newcommand{\Quot}{\mathrm{Quot}}
\newcommand{\Hilb}{\mathrm{Hilb}}
\newcommand{\Pic}{\mathrm{Pic}}
\newcommand{\Div}{\mathrm{Div}}
\newcommand{\Jac}{\mathrm{Jac}}
\newcommand{\Alb}{\mathrm{Alb}} %albanese variety
\newcommand{\Bun}{\mathrm{Bun}}
\newcommand{\loopspace}{\boldsymbol{\Omega}}
\newcommand{\suspension}{\boldsymbol{\Sigma}}
\newcommand{\tangent}{\mathrm{T}} %tangent space
\newcommand{\Eig}{\mathrm{Eig}}
\newcommand{\Cox}{\mathrm{Cox}} %coxeter functors
\newcommand{\rmK}{\mathrm{K}} %Killing form
\newcommand{\km}{\mathfrak{km}} %kac-moody algebras
\newcommand{\Dyn}{\mathrm{Dyn}} %associated Dynkin quivers
\newcommand{\Car}{\mathrm{Car}} %cartan matrices of finite quivers
\newcommand{\uce}{\mathfrak{uce}} %universal central extension of lie algebras

\newcommand{\Ring}{\mathrm{Ring}}
\newcommand{\Cring}{\mathrm{CRing}}
\newcommand{\Bool}{\mathrm{Bool}} %boolean algebras
\newcommand{\Alg}{\mathrm{Alg}}
\newcommand{\Leib}{\mathrm{Leib}} %leibniz algebras
\newcommand{\Fld}{\mathrm{Fld}}
\newcommand{\Sets}{\mathrm{Sets}}
\newcommand{\Equiv}{\mathrm{Equiv}} %equivalence relations
\newcommand{\Cat}{\mathrm{Cat}}
\newcommand{\Grp}{\mathrm{Grp}}
\newcommand{\Ab}{\mathrm{Ab}}
\newcommand{\Sch}{\mathrm{Sch}}
\newcommand{\Coh}{\mathrm{Coh}}
\newcommand{\QCoh}{\mathrm{QCoh}}
\newcommand{\Perf}{\mathrm{Perf}} %perfect complexes
\newcommand{\Sing}{\mathrm{Sing}} %singularity categories
\newcommand{\Desc}{\mathrm{Desc}}
\newcommand{\Sh}{\mathrm{Sh}}
\newcommand{\Psh}{\mathrm{PSh}}
\newcommand{\Fib}{\mathrm{Fib}}
\renewcommand{\mod}{\-\mathrm{mod}}
\newcommand{\comod}{\-\mathrm{comod}}
\newcommand{\bimod}{\-\mathrm{bimod}}
\newcommand{\Vect}{\mathrm{Vect}}
\newcommand{\Rep}{\mathrm{Rep}}
\newcommand{\Grpd}{\mathrm{Grpd}}
\newcommand{\Arr}{\mathrm{Arr}}
\newcommand{\Esp}{\mathrm{Esp}}
\newcommand{\Ob}{\mathrm{Ob}}
\newcommand{\Mor}{\mathrm{Mor}}
\newcommand{\Mfd}{\mathrm{Mfd}}
\newcommand{\Riem}{\mathrm{Riem}}
\newcommand{\RS}{\mathrm{RS}}
\newcommand{\LRS}{\mathrm{LRS}}
\newcommand{\TRS}{\mathrm{TRS}}
\newcommand{\TLRS}{\mathrm{TLRS}}
\newcommand{\LVRS}{\mathrm{LVRS}}
\newcommand{\LBRS}{\mathrm{LBRS}}
\newcommand{\Spc}{\mathrm{Spc}}
\newcommand{\Top}{\mathrm{Top}}
\newcommand{\Topos}{\mathrm{Topos}}
\newcommand{\Nil}{\operatorname{Nil}}
\newcommand{\Rad}{\operatorname{Rad}}
\newcommand{\Stk}{\mathrm{Stk}}
\newcommand{\Pre}{\mathrm{Pre}}
\newcommand{\simp}{\mathbf{\Delta}}
\newcommand{\Res}{\mathrm{Res}}
\newcommand{\Ind}{\mathrm{Ind}}
\newcommand{\Pro}{\mathrm{Pro}}
\newcommand{\Mon}{\mathrm{Mon}}
\newcommand{\Comm}{\mathrm{Comm}}
\newcommand{\Fin}{\mathrm{Fin}}
\newcommand{\Assoc}{\mathrm{Assoc}}
\newcommand{\Semi}{\mathrm{Semi}}
\newcommand{\Co}{\mathrm{Co}}
\newcommand{\Loc}{\mathrm{Loc}}
\newcommand{\Ringed}{\mathrm{Ringed}}
\newcommand{\Haus}{\mathrm{Haus}} %hausdorff spaces
\newcommand{\Comp}{\mathrm{Comp}} %compact hausdorff spaces
\newcommand{\Stone}{\mathrm{Stone}} %stone spaces
\newcommand{\Extr}{\mathrm{Extr}} %extremely disconnected spaces
\newcommand{\Ouv}{\mathrm{Ouv}}
\newcommand{\Str}{\mathrm{Str}}
\newcommand{\Func}{\mathrm{Func}}
\newcommand{\Crys}{\mathrm{Crys}}
\newcommand{\LocSys}{\mathrm{LocSys}}
\newcommand{\Sieves}{\mathrm{Sieves}}
\newcommand{\pt}{\mathrm{pt}}
\newcommand{\Graphs}{\mathrm{Graphs}}
\newcommand{\Lie}{\mathrm{Lie}}
\newcommand{\Env}{\mathrm{Env}}
\newcommand{\Ho}{\mathrm{Ho}}
\newcommand{\rmD}{\mathrm{D}}
\newcommand{\Cov}{\mathrm{Cov}}
\newcommand{\Frames}{\mathrm{Frames}}
\newcommand{\Locales}{\mathrm{Locales}}
\newcommand{\Span}{\mathrm{Span}}
\newcommand{\Corr}{\mathrm{Corr}}
\newcommand{\Monad}{\mathrm{Monad}}
\newcommand{\Var}{\mathrm{Var}}
\newcommand{\sfN}{\mathrm{N}} %nerve
\newcommand{\Diam}{\mathrm{Diam}} %diamonds
\newcommand{\co}{\mathrm{co}}
\newcommand{\ev}{\mathrm{ev}}
\newcommand{\bi}{\mathrm{bi}}
\newcommand{\Nat}{\mathrm{Nat}}
\newcommand{\Hopf}{\mathrm{Hopf}}
\newcommand{\Dmod}{\mathrm{D}\mod}
\newcommand{\Perv}{\mathrm{Perv}}
\newcommand{\Sph}{\mathrm{Sph}}
\newcommand{\Moduli}{\mathrm{Moduli}}
\newcommand{\Pseudo}{\mathrm{Pseudo}}
\newcommand{\Lax}{\mathrm{Lax}}
\newcommand{\Strict}{\mathrm{Strict}}
\newcommand{\Opd}{\mathrm{Opd}} %operads
\newcommand{\Shv}{\mathrm{Shv}}
\newcommand{\Char}{\mathrm{Char}} %CharShv = character sheaves
\newcommand{\Huber}{\mathrm{Huber}}
\newcommand{\Tate}{\mathrm{Tate}}
\newcommand{\Affd}{\mathrm{Affd}} %affinoid algebras
\newcommand{\Adic}{\mathrm{Adic}} %adic spaces
\newcommand{\Rig}{\mathrm{Rig}}
\newcommand{\An}{\mathrm{An}}
\newcommand{\Perfd}{\mathrm{Perfd}} %perfectoid spaces
\newcommand{\Sub}{\mathrm{Sub}} %subobjects
\newcommand{\Ideals}{\mathrm{Ideals}}
\newcommand{\Isoc}{\mathrm{Isoc}} %isocrystals
\newcommand{\Ban}{\mathrm{Ban}} %Banach spaces
\newcommand{\Fre}{\mathrm{Fr\acute{e}}} %Frechet spaces
\newcommand{\Ch}{\mathrm{Ch}} %chain complexes
\newcommand{\Pure}{\mathrm{Pure}}
\newcommand{\Mixed}{\mathrm{Mixed}}
\newcommand{\Hodge}{\mathrm{Hodge}} %Hodge structures
\newcommand{\Mot}{\mathrm{Mot}} %motives
\newcommand{\KL}{\mathrm{KL}} %category of Kazhdan-Lusztig modules
\newcommand{\Pres}{\mathrm{Pr}} %presentable categories
\newcommand{\Noohi}{\mathrm{Noohi}} %category of Noohi groups
\newcommand{\Inf}{\mathrm{Inf}}
\newcommand{\LPar}{\mathrm{LPar}} %Langlands parameters
\newcommand{\ORig}{\mathrm{ORig}} %overconvergent sites
\newcommand{\Quiv}{\mathrm{Quiv}} %quivers
\newcommand{\Def}{\mathrm{Def}} %deformation functors
\newcommand{\Higgs}{\mathrm{Higgs}}
\newcommand{\BGG}{\mathrm{BGG}}
\newcommand{\Poiss}{\mathrm{Poiss}}
\newcommand{\Fact}{\mathrm{Fact}} %factorisation
\newcommand{\Chr}{\mathrm{Chr}} %chiral
\newcommand{\Smooth}{\mathrm{Sm}}

\newcommand{\Aut}{\operatorname{Aut}}
\newcommand{\Inn}{\operatorname{Inn}}
\newcommand{\Out}{\operatorname{Out}}
\newcommand{\der}{\mathfrak{der}} %derivations on Lie algebras
\newcommand{\frakend}{\mathfrak{end}}
\newcommand{\aut}{\mathfrak{aut}}
\newcommand{\inn}{\mathfrak{inn}} %inner derivations
\newcommand{\out}{\mathfrak{out}} %outer derivations
\newcommand{\Stab}{\operatorname{Stab}}
\newcommand{\Cent}{\operatorname{Cent}}
\newcommand{\Norm}{\operatorname{Norm}}
\newcommand{\Core}{\operatorname{Core}}
\newcommand{\cent}{\mathfrak{cent}}
\newcommand{\core}{\mathfrak{core}}
\newcommand{\Transporter}{\operatorname{Transp}} %transporter between two subsets of a group
\newcommand{\Conj}{\operatorname{Conj}}
\newcommand{\Diag}{\operatorname{Diag}}
\newcommand{\Gal}{\operatorname{Gal}}
\newcommand{\bfG}{\mathbf{G}} %absolute Galois group
\newcommand{\Frac}{\operatorname{Frac}}
\newcommand{\Ann}{\operatorname{Ann}}
\newcommand{\Val}{\operatorname{Val}}
\newcommand{\Chow}{\operatorname{Chow}}
\newcommand{\Sym}{\operatorname{Sym}}
\newcommand{\End}{\operatorname{End}}
\newcommand{\Mat}{\operatorname{Mat}}
\newcommand{\Diff}{\operatorname{Diff}}
\newcommand{\Autom}{\operatorname{Autom}}
\newcommand{\Artin}{\operatorname{Artin}} %artin maps
\newcommand{\sk}{\operatorname{sk}} %skeleton of a category
\newcommand{\eqv}{\operatorname{eqv}} %functor that maps groups $G$ to $G$-sets
\newcommand{\Inert}{\operatorname{Inert}}
\newcommand{\Fil}{\operatorname{Fil}}
\newcommand{\prim}{\mathfrak{prim}}
\newcommand{\Nerve}{\operatorname{N}}
\newcommand{\Hol}{\operatorname{Hol}} %holomorphic functions %holonomy groups
\newcommand{\Bi}{\operatorname{Bi}} %Bi for biholomorphic functions
\newcommand{\chev}{\operatorname{chev}}
\newcommand{\bfLie}{\mathbf{Lie}} %non-reduced lie algebra associated to generalised cartan matrices
\newcommand{\frakLie}{\mathfrak{Lie}} %reduced lie algebra associated to generalised cartan matrices
\newcommand{\frakChev}{\mathfrak{Chev}} 
\newcommand{\Rees}{\operatorname{Rees}}
\newcommand{\Dr}{\operatorname{Dr}} %Drinfeld's quantum double 
\newcommand{\frakDr}{\mathfrak{Dr}} %classical double of lie bialgebras
\newcommand{\Tot}{\operatorname{Tot}} %total complexes

\renewcommand{\projlim}{\varprojlim}
\newcommand{\indlim}{\varinjlim}
%\newcommand{\colim}{\operatorname{colim}}
%\renewcommand{\lim}{\operatorname{lim}}
\newcommand{\toto}{\rightrightarrows}
%\newcommand{\tensor}{\otimes}
\NewDocumentCommand{\tensor}{e{_^}}{%
  \mathbin{\mathop{\otimes}\displaylimits
    \IfValueT{#1}{_{#1}}
    \IfValueT{#2}{^{#2}}
  }%
}
\NewDocumentCommand{\singtensor}{e{_^}}{%
  \mathbin{\mathop{\odot}\displaylimits
    \IfValueT{#1}{_{#1}}
    \IfValueT{#2}{^{#2}}
  }%
}
\NewDocumentCommand{\hattensor}{e{_^}}{%
  \mathbin{\mathop{\hat{\otimes}}\displaylimits
    \IfValueT{#1}{_{#1}}
    \IfValueT{#2}{^{#2}}
  }%
}
\NewDocumentCommand{\brevetensor}{e{_^}}{%
  \mathbin{\mathop{\breve{\otimes}}\displaylimits
    \IfValueT{#1}{_{#1}}
    \IfValueT{#2}{^{#2}}
  }%
}
\NewDocumentCommand{\semidirect}{e{_^}}{%
  \mathbin{\mathop{\rtimes}\displaylimits
    \IfValueT{#1}{_{#1}}
    \IfValueT{#2}{^{#2}}
  }%
}
\newcommand{\eq}{\operatorname{eq}}
\newcommand{\coeq}{\operatorname{coeq}}
\newcommand{\Hom}{\operatorname{Hom}}
\newcommand{\Bil}{\operatorname{Bil}} %bilinear maps
\newcommand{\Maps}{\operatorname{Maps}}
\newcommand{\Tor}{\operatorname{Tor}}
\newcommand{\Ext}{\operatorname{Ext}}
\newcommand{\Isom}{\operatorname{Isom}}
\newcommand{\stalk}{\operatorname{stalk}}
\newcommand{\RKE}{\operatorname{RKE}}
\newcommand{\LKE}{\operatorname{LKE}}
\newcommand{\oblv}{\operatorname{oblv}}
\newcommand{\const}{\operatorname{const}}
\newcommand{\free}{\operatorname{free}}
\newcommand{\adrep}{\operatorname{ad}} %adjoint representation
\newcommand{\NL}{\mathbb{NL}} %naive cotangent complex
\newcommand{\pr}{\operatorname{pr}}
\newcommand{\Der}{\operatorname{Der}}
\newcommand{\Frob}{\operatorname{Fr}} %Frobenius
\newcommand{\frob}{\operatorname{f}} %trace of Frobenius
\newcommand{\bfpt}{\mathbf{pt}}
\newcommand{\bfloc}{\mathbf{loc}}
\DeclareMathAlphabet{\mymathbb}{U}{BOONDOX-ds}{m}{n}
\newcommand{\0}{\mymathbb{0}}
\newcommand{\1}{\mathbbm{1}}
\newcommand{\2}{\mathbbm{2}}
\newcommand{\Jet}{\operatorname{Jet}}
\newcommand{\Split}{\mathrm{Split}}
\newcommand{\Sq}{\mathrm{Sq}}
\newcommand{\Zero}{\mathrm{Z}}
\newcommand{\SqZ}{\Sq\Zero}
\newcommand{\lie}{\mathfrak{lie}}
\newcommand{\y}{\operatorname{y}} %yoneda
\newcommand{\Sm}{\mathrm{Sm}}
\newcommand{\AJ}{\phi} %abel-jacobi map
\newcommand{\act}{\mathrm{act}}
\newcommand{\ram}{\mathrm{ram}} %ramification index
\newcommand{\inv}{\mathrm{inv}}
\newcommand{\Spr}{\mathrm{Spr}} %the Springer map/sheaf
\newcommand{\Refl}{\mathrm{Refl}} %reflection functor]
\newcommand{\HH}{\mathrm{HH}} %Hochschild (co)homology
\newcommand{\HC}{\mathrm{HC}} %cyclic (co)homology
\newcommand{\Poinc}{\mathrm{Poinc}}
\newcommand{\Simpson}{\mathrm{Simpson}}
\newcommand{\Section}{\operatorname{Sect}}
\newcommand{\Ran}{\operatorname{Ran}} %Ran space
\newcommand{\quantumfields}{\operatorname{QFld}}

\newcommand{\bbU}{\mathbb{U}}
\newcommand{\bbV}{\mathbb{V}}
\newcommand{\W}{\mathbb{W}}
\newcommand{\calU}{\mathcal{U}}
\newcommand{\calW}{\mathcal{W}}
\newcommand{\rmI}{\mathrm{I}} %augmentation ideal
\newcommand{\bfV}{\mathbf{V}}
\newcommand{\C}{\mathcal{C}}
\newcommand{\D}{\mathcal{D}}
\newcommand{\scrD}{\mathscr{D}}
\newcommand{\T}{\mathscr{T}} %Tate modules
\newcommand{\calM}{\mathcal{M}}
\newcommand{\calN}{\mathcal{N}}
\newcommand{\calP}{\mathcal{P}}
\newcommand{\calQ}{\mathcal{Q}}
\newcommand{\A}{\mathbb{A}}
\renewcommand{\P}{\mathbb{P}}
\newcommand{\calL}{\mathcal{L}}
\newcommand{\scrL}{\mathscr{L}}
\newcommand{\E}{\mathcal{E}}
\renewcommand{\H}{\mathbf{H}}
\newcommand{\scrS}{\mathscr{S}}
\newcommand{\calX}{\mathcal{X}}
\newcommand{\calY}{\mathcal{Y}}
\newcommand{\calZ}{\mathcal{Z}}
\newcommand{\calS}{\mathcal{S}}
\newcommand{\calR}{\mathcal{R}}
\newcommand{\calr}{\mathcal{r}}
\newcommand{\scrX}{\mathscr{X}}
\newcommand{\scrY}{\mathscr{Y}}
\newcommand{\scrZ}{\mathscr{Z}}
\newcommand{\calA}{\mathcal{A}}
\newcommand{\calB}{\mathcal{B}}
\renewcommand{\S}{\mathcal{S}}
\newcommand{\B}{\mathbb{B}}
\newcommand{\bbD}{\mathbb{D}}
\newcommand{\G}{\mathbb{G}}
\newcommand{\horn}{\mathbf{\Lambda}}
\renewcommand{\L}{\mathbb{L}}
\renewcommand{\a}{\mathfrak{a}}
\renewcommand{\b}{\mathfrak{b}}
\renewcommand{\c}{\mathfrak{c}}
\renewcommand{\d}{\mathfrak{d}}
\renewcommand{\t}{\mathfrak{t}}
\renewcommand{\r}{\mathfrak{r}}
\renewcommand{\o}{\mathfrak{o}}
\renewcommand{\sp}{\mathfrak{sp}}
\newcommand{\fraku}{\mathfrak{u}}
\newcommand{\frakl}{\mathfrak{l}}
\newcommand{\fraky}{\mathfrak{y}}
\newcommand{\frakv}{\mathfrak{v}}
\newcommand{\frakw}{\mathfrak{w}}
\newcommand{\frake}{\mathfrak{e}}
\newcommand{\bbX}{\mathbb{X}}
\newcommand{\frakG}{\mathfrak{G}}
\newcommand{\frakH}{\mathfrak{H}}
\newcommand{\frakE}{\mathfrak{E}}
\newcommand{\frakF}{\mathfrak{F}}
\newcommand{\g}{\mathfrak{g}}
\newcommand{\h}{\mathfrak{h}}
\renewcommand{\k}{\mathfrak{k}}
\newcommand{\z}{\mathfrak{z}}
\newcommand{\fraki}{\mathfrak{i}}
\newcommand{\frakj}{\mathfrak{j}}
\newcommand{\del}{\partial}
\newcommand{\bbE}{\mathbb{E}}
\newcommand{\scrO}{\mathscr{O}}
\newcommand{\bbO}{\mathbb{O}}
\newcommand{\scrA}{\mathscr{A}}
\newcommand{\scrB}{\mathscr{B}}
\newcommand{\scrE}{\mathscr{E}}
\newcommand{\scrF}{\mathscr{F}}
\newcommand{\scrG}{\mathscr{G}}
\newcommand{\scrM}{\mathscr{M}}
\newcommand{\scrN}{\mathscr{N}}
\newcommand{\scrP}{\mathscr{P}}
\newcommand{\frakS}{\mathfrak{S}}
\newcommand{\frakT}{\mathfrak{T}}
\newcommand{\calI}{\mathcal{I}}
\newcommand{\calJ}{\mathcal{J}}
\newcommand{\scrI}{\mathscr{I}}
\newcommand{\scrJ}{\mathscr{J}}
\newcommand{\scrH}{\mathscr{H}}
\newcommand{\calH}{\mathcal{H}}
\newcommand{\scrK}{\mathscr{K}}
\newcommand{\calK}{\mathcal{K}}
\newcommand{\scrV}{\mathscr{V}}
\newcommand{\scrW}{\mathscr{W}}
\newcommand{\bbS}{\mathbb{S}}
\newcommand{\bfA}{\mathbf{A}}
\newcommand{\bfB}{\mathbf{B}}
\newcommand{\bfC}{\mathbf{C}}
\renewcommand{\O}{\mathbb{O}}
\newcommand{\calV}{\mathcal{V}}
\newcommand{\scrR}{\mathscr{R}} %radical
\newcommand{\sfR}{\mathsf{R}} %quantum R-matrices
\newcommand{\sfr}{\mathsf{r}} %classical R-matrices
\newcommand{\rmZ}{\mathrm{Z}} %centre of algebra
\newcommand{\rmC}{\mathrm{C}} %centralisers in algebras
\newcommand{\bfGamma}{\mathbf{\Gamma}}
\newcommand{\scrU}{\mathscr{U}}
\newcommand{\rmW}{\mathrm{W}} %Weil group
\newcommand{\frakM}{\mathfrak{M}}
\newcommand{\frakN}{\mathfrak{N}}
\newcommand{\frakB}{\mathfrak{B}}
\newcommand{\frakX}{\mathfrak{X}}
\newcommand{\frakY}{\mathfrak{Y}}
\newcommand{\frakZ}{\mathfrak{Z}}
\newcommand{\frakU}{\mathfrak{U}}
\newcommand{\frakR}{\mathfrak{R}}
\newcommand{\frakP}{\mathfrak{P}}
\newcommand{\frakQ}{\mathfrak{Q}}
\newcommand{\sfX}{\mathsf{X}}
\newcommand{\sfY}{\mathsf{Y}}
\newcommand{\sfZ}{\mathsf{Z}}
\newcommand{\sfS}{\mathsf{S}}
\newcommand{\sfT}{\mathsf{T}}
\newcommand{\sfOmega}{\mathsf{\Omega}} %drinfeld p-adic upper-half plane
\newcommand{\rmA}{\mathrm{A}}
\newcommand{\rmB}{\mathrm{B}}
\newcommand{\calT}{\mathcal{T}}
\newcommand{\sfA}{\mathsf{A}}
\newcommand{\sfB}{\mathsf{B}}
\newcommand{\sfC}{\mathsf{C}}
\newcommand{\sfD}{\mathsf{D}}
\newcommand{\sfE}{\mathsf{E}}
\newcommand{\sfF}{\mathsf{F}}
\newcommand{\sfG}{\mathsf{G}}
\newcommand{\sfh}{\mathsf{h}} %coxeter number
\newcommand{\frakL}{\mathfrak{L}}
\newcommand{\K}{\mathrm{K}}
\newcommand{\rmT}{\mathrm{T}}
\newcommand{\bfv}{\mathbf{v}}
\newcommand{\bfg}{\mathbf{g}}
\newcommand{\frakV}{\mathfrak{V}}
\newcommand{\bfn}{\mathbf{n}}

%special lie modules
\newcommand{\standard}{\boldsymbol{M}}
\newcommand{\simple}{\boldsymbol{L}}
\newcommand{\vacuum}{ \boldsymbol{ \mathcal{Vac} } }
\newcommand{\weyl}{\mathcal{W}}
\newcommand{\boson}{ \boldsymbol{ \mathcal{Bos} } }
\newcommand{\fermion}{ \boldsymbol{ \mathcal{Fer} } }
\newcommand{\KR}{ \boldsymbol{ \mathcal{KR} } }

\newcommand{\frakC}{\mathfrak{C}}
\newcommand{\frakD}{\mathfrak{D}}
\newcommand{\rmi}{\mathrm{i}}
\newcommand{\bfH}{\mathbf{H}}
\newcommand{\bfX}{\mathbf{X}}
\newcommand{\coxeter}{\mathrm{h}}

\newcommand{\aff}{\mathrm{aff}}
\newcommand{\ft}{\mathrm{ft}} %finite type
\newcommand{\fp}{\mathrm{fp}} %finite presentation
\newcommand{\fr}{\mathrm{fr}} %free
\newcommand{\tft}{\mathrm{tft}} %topologically finite type
\newcommand{\tfp}{\mathrm{tfp}} %topologically finite presentation
\newcommand{\tfr}{\mathrm{tfr}} %topologically free
\newcommand{\aft}{\mathrm{aft}}
\newcommand{\lft}{\mathrm{lft}}
\newcommand{\laft}{\mathrm{laft}}
\newcommand{\cpt}{\mathrm{cpt}}
\newcommand{\cproj}{\mathrm{cproj}}
\newcommand{\qc}{\mathrm{qc}}
\newcommand{\qs}{\mathrm{qs}}
\newcommand{\lcmpt}{\mathrm{lcmpt}}
\newcommand{\red}{\mathrm{red}}
\newcommand{\fin}{\mathrm{fin}}
\newcommand{\fd}{\mathrm{fd}} %finite-dimensional
\newcommand{\gen}{\mathrm{gen}}
\newcommand{\petit}{\mathrm{petit}}
\newcommand{\gros}{\mathrm{gros}}
\newcommand{\loc}{\mathrm{loc}}
\newcommand{\glob}{\mathrm{glob}}
%\newcommand{\ringed}{\mathrm{ringed}}
%\newcommand{\qcoh}{\mathrm{qcoh}}
\newcommand{\cl}{\mathrm{cl}}
\newcommand{\et}{\mathrm{\acute{e}t}}
\newcommand{\fet}{\mathrm{f\acute{e}t}}
\newcommand{\profet}{\mathrm{prof\acute{e}t}}
\newcommand{\proet}{\mathrm{pro\acute{e}t}}
\newcommand{\Zar}{\mathrm{Zar}}
\newcommand{\fppf}{\mathrm{fppf}}
\newcommand{\fpqc}{\mathrm{fpqc}}
\newcommand{\orig}{\mathrm{orig}} %overconvergent topology
\newcommand{\smooth}{\mathrm{sm}}
\newcommand{\sh}{\mathrm{sh}}
\newcommand{\op}{\mathrm{op}}
\newcommand{\cop}{\mathrm{cop}}
\newcommand{\open}{\mathrm{open}}
\newcommand{\closed}{\mathrm{closed}}
\newcommand{\geom}{\mathrm{geom}}
\newcommand{\alg}{\mathrm{alg}}
\newcommand{\sober}{\mathrm{sober}}
\newcommand{\dR}{\mathrm{dR}}
\newcommand{\rad}{\mathfrak{rad}}
\newcommand{\discrete}{\mathrm{discrete}}
%\newcommand{\add}{\mathrm{add}}
%\newcommand{\lin}{\mathrm{lin}}
\newcommand{\Krull}{\mathrm{Krull}}
\newcommand{\qis}{\mathrm{qis}} %quasi-isomorphism
\newcommand{\ho}{\mathrm{ho}} %homotopy equivalence
\newcommand{\sep}{\mathrm{sep}}
\newcommand{\insep}{\mathrm{insep}}
\newcommand{\unr}{\mathrm{unr}}
\newcommand{\tame}{\mathrm{tame}}
\newcommand{\wild}{\mathrm{wild}}
\newcommand{\nil}{\mathrm{nil}}
\newcommand{\defm}{\mathrm{defm}}
\newcommand{\Art}{\mathrm{Art}}
\newcommand{\Noeth}{\mathrm{Noeth}}
\newcommand{\affd}{\mathrm{affd}}
%\newcommand{\adic}{\mathrm{adic}}
\newcommand{\pre}{\mathrm{pre}}
\newcommand{\coperf}{\mathrm{coperf}}
\newcommand{\perf}{\mathrm{perf}}
\newcommand{\perfd}{\mathrm{perfd}}
\newcommand{\rat}{\mathrm{rat}}
\newcommand{\cont}{\mathrm{cont}}
\newcommand{\dg}{\mathrm{dg}}
\newcommand{\almost}{\mathrm{a}}
%\newcommand{\stab}{\mathrm{stab}}
\newcommand{\heart}{\heartsuit}
\newcommand{\proj}{\mathrm{proj}}
\newcommand{\rot}{\mathrm{rot}}
\newcommand{\qproj}{\mathrm{qproj}} %quasi-projective
\newcommand{\pd}{\mathrm{pd}}
\newcommand{\crys}{\mathrm{crys}}
\newcommand{\prisma}{\mathrm{prisma}}
\newcommand{\FF}{\mathrm{FF}}
\newcommand{\sph}{\mathrm{sph}}
\newcommand{\lax}{\mathrm{lax}}
\newcommand{\weak}{\mathrm{weak}}
\newcommand{\strict}{\mathrm{strict}}
\newcommand{\mon}{\mathrm{mon}}
\newcommand{\sym}{\mathrm{sym}}
\newcommand{\lisse}{\mathrm{lisse}}
\newcommand{\an}{\mathrm{an}}
\newcommand{\ad}{\mathrm{ad}}
\newcommand{\sch}{\mathrm{sch}}
\newcommand{\rig}{\mathrm{rig}}
\newcommand{\pol}{\mathrm{pol}}
\newcommand{\plat}{\mathrm{flat}}
\newcommand{\proper}{\mathrm{proper}}
\newcommand{\compl}{\mathrm{compl}}
\newcommand{\non}{\mathrm{non}}
\newcommand{\access}{\mathrm{access}}
\newcommand{\comp}{\mathrm{comp}}
\newcommand{\tstructure}{\mathrm{t}} %t-structures
\newcommand{\pure}{\mathrm{pure}} %pure motives
\newcommand{\mixed}{\mathrm{mixed}} %mixed motives
\newcommand{\num}{\mathrm{num}} %numerical motives
\newcommand{\ess}{\mathrm{ess}}
\newcommand{\topological}{\mathrm{top}}
\newcommand{\convex}{\mathrm{cvx}}
\newcommand{\locconvex}{\mathrm{lcvx}}
\newcommand{\ab}{\mathrm{ab}} %abelian extensions
\newcommand{\inj}{\mathrm{inj}}
\newcommand{\surj}{\mathrm{surj}} %coverage on sets generated by surjections
\newcommand{\eff}{\mathrm{eff}} %effective Cartier divisors
\newcommand{\Weil}{\mathrm{Weil}} %weil divisors
\newcommand{\lex}{\mathrm{lex}}
\newcommand{\rex}{\mathrm{rex}}
\newcommand{\AR}{\mathrm{A\-R}}
\newcommand{\cons}{\mathrm{c}} %constructible sheaves
\newcommand{\tor}{\mathrm{tor}} %tor dimension
\newcommand{\connected}{\mathrm{connected}}
\newcommand{\cg}{\mathrm{cg}} %compactly generated
\newcommand{\nilp}{\mathrm{nilp}}
\newcommand{\isg}{\mathrm{isg}} %isogenous
\newcommand{\qisg}{\mathrm{qisg}} %quasi-isogenous
\newcommand{\irr}{\mathrm{irr}} %irreducible represenations
\newcommand{\indecomp}{\mathrm{indecomp}}
\newcommand{\preproj}{\mathrm{preproj}}
\newcommand{\preinj}{\mathrm{preinj}}
\newcommand{\reg}{\mathrm{reg}}
\newcommand{\sing}{\mathrm{sing}}
\newcommand{\crit}{\mathrm{crit}}
\newcommand{\semisimple}{\mathrm{ss}}
\newcommand{\integrable}{\mathrm{int}}
\newcommand{\s}{\mathfrak{s}}
\newcommand{\elliptic}{\mathrm{ell}}
\newcommand{\stab}{\mathrm{stab}}
\newcommand{\disj}{\mathrm{disj}}
\newcommand{\positive}{\mathrm{pos}} 
\newcommand{\negative}{\mathrm{neg}} 
\newcommand{\up}{\mathrm{up}} %upper
\newcommand{\low}{\mathrm{low}} %lower
\newcommand{\locallyconstant}{\mathrm{lconst}}
\newcommand{\complete}{\mathrm{compl}}
\newcommand{\chiral}{\mathrm{ch}}
\newcommand{\fact}{\mathrm{fact}}
\newcommand{\tw}{\mathrm{tw}} %twisted
\newcommand{\ext}{\mathrm{ext}} %extended

%prism custom command
\usepackage{relsize}
\usepackage[bbgreekl]{mathbbol}
\usepackage{amsfonts}
\DeclareSymbolFontAlphabet{\mathbb}{AMSb} %to ensure that the meaning of \mathbb does not change
\DeclareSymbolFontAlphabet{\mathbbl}{bbold}
\newcommand{\prism}{{\mathlarger{\mathbbl{\Delta}}}}

%symmetric pairs; twisted yangians and reflection algebras
\newcommand{\romanzero}{\mathsf{0}}
\newcommand{\romanone}{\mathsf{I}}
\newcommand{\romantwo}{\mathsf{II}}
\newcommand{\romanthree}{\mathsf{III}}
\newcommand{\Aone}{\sfA\romanone}
\newcommand{\BCD}{\sfB, \sfC, \sfD}
\newcommand{\BCDzero}{\sfB\sfC\sfD\romanzero}
\newcommand{\BCDone}{\sfB\sfC\sfD\romanone}
\newcommand{\XB}{\mathcal{XB}} %reflection equation
\newcommand{\UXB}{\mathcal{UXB}} %reflection equation and unitary condition
\newcommand{\UB}{\mathcal{UB}} %reflection equation, unitary condition, and symmetry relation
\newcommand{\ZB}{\mathcal{ZB}} 
\newcommand{\DB}{\mathcal{DB}} %doubled reflection algebra 
\newcommand{\DY}{\mathcal{DY}} %double yangian
\newcommand{\DX}{\mathcal{DX}} %double extended yangian

\newcommand{\Loop}{\mathrm{L}}

\renewcommand{\i}{\imath}

%superalgebras
\newcommand{\even}{\bar{0}}
\newcommand{\odd}{\bar{1}}
\newcommand{\super}{\mathrm{s}}

%schur weyl duality
\newcommand{\heckealgebra}{\mathsf{He}}
\newcommand{\degenerateheckealgebra}{\overline{\heckealgebra}}
\newcommand{\braueralgebra}{\mathsf{Br}}
\newcommand{\walledbraueralgebra}{\overline{\braueralgebra}}
\newcommand{\VW}{\mathsf{VW}}

\newcommand{\Heis}{\mathcal{H}}

\newcommand{\KM}{\mathrm{KM}} %kac-moody presentation/drinfeld-jimbo presentation
\newcommand{\current}{\mathrm{Dr}} %drinfeld current presentation
\renewcommand{\1}{\boldsymbol{\delta}}

\begin{document}

    \title{
        Quantum symmetric pairs in $\calU_q(\Loop\sl_2)$ and $\DY(\sl_2)$
        \\
        ---
        \\
        Plan
    }
    
    \author{Dat Minh Ha}
    \maketitle

    \begin{abstract}
        This is a proof of concept for an intended solution to the original question of defining and studying twisted $q$-Yangians associated to (pseudo-)symmetric pairs of types $\sfB, \sfC, \sfD$. I also realised that many details are still missing in the type $\sfA$ case as well, so I thought that it was worth spending some time with this case, even if it is to serve just as a prototype for our eventual goal.
        
        This document is meant only as an outline of the approach that I intend to pursue, so there is some trade-off between technical precision and conciseness. 
    \end{abstract}
    
    {
    \hypersetup{} 
    %\dominitoc
    \tableofcontents %sort sections alphabetically
    }

    \section{The plan}
        \subsection{Terminologies and notations}
            \begin{convention}[Quantum algebras]
                First of all, to fix terminologies, let me refer to both quantum symmetric pairs and quantum pseudo-symmetric pairs (in the sense of \cite{regelskis_vlaar_kac_moody_pseudo_symmetric_pairs}) as \say{quantum symmetric pairs} and abbreviate the term by QSP. When there is a need to specify the presentation of the QSPs at play, the QSPs that arise from reflection equations (and hence ultimately from K-matrices; e.g. as in \cite{molev_ragoucy_sorba_twisted_q_yangians_type_A} or \cite{guay_regelskis_twisted_yangians_for_symmetric_pairs_of_types_BCD}) shall be referred to as \say{reflection algebras}, while those given by Kac-Moody-style generators satisfying Chevalley-Serre-type relations (such as the QSPs in \cite{regelskis_vlaar_reflection_matrices_coideal_subalgebras}) shall be referred to as \say{$\i$quantum groups}; I need this distinction also because the latter type does not seem to have an agreed-upon name in the literature, though the term \say{$\i$quantum group} seems to be the next best thing. Otherwise, I will keep to the standard terminologies, e.g. \say{twisted $q$-Yangians}, \say{twisted Yangians}, etc.
            
                Quantum algebras that arise as quantum doubles or (reasonably minimal) quotients thereof, such as quantum Kac-Moody algebras, quantum affine algebras, quantum loop algebras, (extended) double Yangians, will sometimes be referred to collectively as \say{affine quantum algebras} or something of the sort; this is because there are constructions that will work identically between the trigonometric and rational cases and will only depend qualitatively on the R-matrix; I hope context will make things clear down below.
            \end{convention}

            \begin{convention}[Spectral parameters]
                Trigonometric R-matrices and K-matrices will usually depend multiplicatively on spectral parameters $z, w$, while rational R-matrices will depend additively on spectral parameters $u := \log(z), v := \log(w)$.
            \end{convention}

            \begin{convention}[Formal power series]
                In keeping with modern conventions (particularly within the vertex algebra literature), given any formal distribution:
                    $$a(z) := \sum_{m \in \Z} a_m z^{-m - 1}$$
                let us write:
                    $$a(z)^- := \sum_{m \in \geq 0} a_m z^{-m - 1} \quad, \quad a(z)^+ := \sum_{m < 0} a_m z^{-m - 1}$$
                for the negative and positive halves, respectively, with the point being that we would then have:
                    $$(\del_z a(z))^{\pm} = \del_z( a(z)^{\pm} ) \quad, \quad \del_z := \frac{d}{dz}$$
                These notations will usually be used for generating series. Also, for the formal Dirac distribution around $z = w$ (cf. e.g. \cite[Subsection 1.1.3]{frenkel_ben_zvi_vertex_algebras_and_algebraic_curves}), we write:
                    $$\1(z - w) := \sum_{n \in \Z} z^n w^{-n - 1}$$
                and it is useful to recall that:
                    $$(z - w)^{n + 1} \del_w^n \1(z - w) = 0$$
            \end{convention}

        \subsection{Setup}
            For us, $\hbar$ shall be a formal variable for the moment. We write $q := e^{\hbar}$, and when $\hbar$ is specialised to numerical values, we assume that $\hbar \in \bbC \setminus 2\pi i \Q$ so that $q$ is not a root of unity.
        
            Suppose that $\calR_{\trigonometric}(zw^{-1})$ is a trigonometric solution to the quantum Yang-Baxter equation (QYBE) with rational limit $\calR_{\rational}(u - v)$ with classical limits $\calr_{\trigonometric}(zw^{-1})$ and $\calr_{\rational}(u - v)$ respectively; denote the corresponding quasi-triangular (topological) Hopf algebras as in \cite{etingof_kazhdan_quantisation_1} (see also \cite{etingof_kazhdan_quantisation_1} and \cite{etingof_kazhdan_quantisation_6}) by:
                $$\calU_{\hbar}(\calR)$$
            and their classical limits by:
                $$\g := \g(\calr)$$
            \textit{A priori}, $\g(\calr_{\trigonometric})$ is a Kac-Moody algebra (with derivation), but let us assume further that it is of a untwisted affine type $\sfX$ in the classification of \cite[Chapter 4]{kac_infinite_dimensional_lie_algebras}. Consequently, there exists a finite-dimensional simple Lie algebra $\bar{\g}$ such that:
                $$\g \cong ( \Loop \bar{\g} \oplus \bbC c ) \rtimes \bbC \del$$
            wherein $\Loop \bar{\g} := \g[t^{\pm 1}]$ and $\del$ acts by $\id \tensor t\frac{d}{dt}$. In particular, this means that the loop algebra $\Loop \bar{\g}$ is a Lie sub-bialgebra of $\g(\calr_{\trigonometric})$, and hence the quantum Kac-Moody algebra $\calU_{\hbar}^{\calR}(\g) := \calU_{\hbar}(\calR_{\trigonometric})$ admits the quantum loop algebra $\calU_{\hbar}^{\calR}(\Loop \bar{\g})$ (both in the R-matrix presentation) as a Hopf subalgebra. Recall also, that there are two other presentations for the quantum Kac-Moody and quantum loop algebras, namely by Drinfeld currents and Kac-Moody generators; let us denote these algebras by $\calU_{\hbar}^{\current}(\g), \calU_{\hbar}^{\current}(\Loop \bar{\g})$ and $\calU_{\hbar}^{\KM}(\g), \calU_{\hbar}^{\KM}(\Loop \bar{\g})$ respectively. By theorems of Ding-Frenkel and Beck (later on clarified by Frenkel and Mukhin), we know that there are algebra isomorphisms:
                $$\calU_{\hbar}^{\calR}(\g) \xrightarrow[]{\cong} \calU_{\hbar}^{\current}(\g) \xrightarrow[]{\cong} \calU_{\hbar}^{\KM}(\g)$$
            which then get restricted down to isomorphisms:
                $$\calU_{\hbar}^{\calR}(\Loop \bar{\g}) \xrightarrow[]{\cong} \calU_{\hbar}^{\current}(\Loop \bar{\g}) \xrightarrow[]{\cong} \calU_{\hbar}^{\KM}(\Loop \bar{\g})$$
                
            Next, suppose that $\calK_{\trigonometric}(z)$ and $\calK_{\rational}(u)$ respectively are solutions to the boundary quantum Yang-Baxter equation (bQYBE) given using the R-matrices fixed previously, and then denote the reflection algebras arising from these K-matrices by:
                $$\calB_{\hbar}(\calK)$$
            Suppose that $\vartheta$ is the (pseudo-)involutive automorphism on $\g(\calr)$ such that the classical limit of the reflection algebra above is the fixed point subalgebra:
                $$(\Loop \bar{\g})^{\vartheta}$$
            Moreover, $\calB_{\hbar}(\calK)$ is \textit{a priori} a right-coideal subalgebra of the quantum loop algebra $\calU_{\hbar}^{\calR}(\Loop \bar{\g})$, hence also a right-coideal subalgebra of $\calU_{\hbar}^{\KM}(\Loop \bar{\g})$ in particular.

            It is also worth noting that $\calU^{\calR}(\Loop \bar{\g})$ comes equipped with representations $\pi_z^{\pm}: \calU^{\calR}(\Loop \bar{\g}) \to \End(\bbC^N)[\![z^{\pm 1}]\!]$

            At the same time, the automorphism $\vartheta$ induces an automorphism on the Dynkin diagram of type $\sfX$, and hence . Thus, there exists an induced automorphism on the set of Kac-Moody-style generators of $\calU_{\hbar}^{\KM}(\Loop \bar{\g})$, and therefore there exists an $\i$quantum group associated to the aforementioned automorphism $\vartheta$, which we denote by:
                $$\calU_{\hbar}^{\KM}(\Loop \bar{\g}, \vartheta)$$
            
        \subsection{Goal}
            Ideally, we would like to be able to show that the algebras:
                $$\calB_{\hbar}(\calK) \quad \text{and} \quad \calU_{\hbar}^{\KM}(\Loop \bar{\g}, \vartheta)$$
            are isomorphic for \textit{all} affine types $\sfX$, even the twisted types (in which case the loop algebra $\Loop \bar{\g}$ will have to be replaced by an twisted version, but we will worry about this later). However, we will focus on the case when:
                $$\bar{\g} = \sl_2$$
            for now, as a proof of concept. In particular, we will be focusing on the cases when the K-matrix is of type $\sfA.1, \sfA.2, \sfA.3$ and thus are diagonal, and the K-matrix of the $q$-Onsager algebra (corresponding to when $\vartheta$ is the Chevalley involution), which is the simplest non-diagonal K-matrix. None of these K-matrices are dynamical, but this is not so much of a technical barrier as non-diagonality.

    \section{Strategy}
        From now on, we focus on the case of the untwisted affine Kac-Moody algebra $\g = \hat{\sl}_2$ with underlying finite-type Lie algebra $\bar{\g} = \sl_2$.

        \subsection{Free boson realisations for affine quantum algebras}
            It is well-known that the generators of $\calU_{\hbar}^{\current}(\hat{\sl}_2)$ - and hence $\calU_{\hbar}^{\current}(\Loop \sl_2)$ too - admits a realisation in terms of so-called free bosonic\footnote{As far as I understand it, non-super symmetry algebras are realised in terms of bosonic fields, while supersymmetry algebras are realised in terms of fermionic fields.} fields; see \cite{awata_odake_shiraishi_free_boson_realisation_of_quantum_affine_sl_N} and \cite[Section 7]{frenkel_reshetikhin_affine_QUEs_and_deformed_virasoro_and_finite_W_algebras}, and in essence, this means that we are constructing an embedding:
                $$\freeboson_{\hbar, \level}: \calU_{\hbar}^{\current}(\hat{\sl}_2)_{\level} \to \calA_{\hbar}^{\current}(\hat{\sl}_2)_{\level}$$
            of the quantum Kac-Moody algebra into a \say{deformed oscillator algebra} (in a presentation by Drinfeld-style currents), both at a \say{level} $\level \in \bbC$; more on this level parameter shortly. This deformed oscillator algebra $\calA_{\hbar}^{\current}(\hat{\sl}_2)_{\level}$ is the grading-completion of the the associative $\bbC[\![\hbar]\!]$-algebra generated by the elements:
                $$\lambda[n], b[n], c[n] \quad, \quad n \in \Z \setminus \{0\}$$
                $$e^{\pm \frac{\lambda_0}{2}}, e^{\pm \frac{(q - q^{-1}) b_0}{2}}, e^{\pm \frac{(q - q^{-1}) c_0}{2}}$$
                $$\beta, \gamma$$
                $$\del$$
            which are subjected to the following relations, given for all $m, n \in \Z \setminus \{0\}$:
                \begin{equation} \label{equation: q_deformed_affine_sl_2_oscillator_lambda_commutators}
                    [ \lambda[m], \lambda[n] ] = \frac1n \frac{ [(\level + \dualcoxeter) n]_q [n]_q^2 }{ [2n]_q } \delta_{m + n, 0}
                \end{equation}
                
                \begin{equation} \label{equation: q_deformed_affine_sl_2_oscillator_bc_commutators}
                    [ b[m], b[n] ] = -\frac1n [n]_q^2 \delta_{m + n, 0} \quad, \quad [ c[m], c[n] ] = \frac1n [n]_q^2 \delta_{m + n, 0}
                \end{equation}
                
                \begin{equation} \label{equation: q_deformed_affine_sl_2_oscillator_beta_gamma_commutators}
                    [ b[0], \beta ] = -\frac{q - q^{-1}}{2\hbar} \quad, \quad [ c[0], \gamma ] = \frac{q - q^{-1}}{2\hbar}
                \end{equation}
            in which $\dualcoxeter = -2$ is the dual Coxter number. Usually, one packages the generators above into series:
                $$\lambda(z) = \sum_{n \in \Z} \lambda[n] z^{-n - 1}$$
                $$
                    b(z) = \sum_{n \in \Z} b[n] z^{-n - 1} \quad, \quad c(z) = \sum_{n \in \Z} c[n] z^{-n - 1}
                $$
            using which we form the following series generating series for the algebra $\calA_{\hbar}^{\current}(\hat{\sl}_2)_{\level}$:
                \begin{equation} \label{equation: q_deformed_affine_sl_2_oscillator_Lambda_generating_series}
                    \Lambda^{\pm}(z) := e^{ \mp \frac{\lambda_0}{2} } e^{ \pm \lambda(z)^{\pm} }
                \end{equation}
                
                \begin{equation} \label{equation: q_deformed_affine_sl_2_oscillator_BC_generating_series}
                    B^{\pm}(z) := \pm (q - q^{-1}) \left( -\frac12 b[0] + b(z)^{\pm} \right)
                    \quad, \quad
                    C^{\pm}(z) := \pm (q - q^{-1}) \left( -\frac12 c[0] + c(z)^{\pm} \right)
                \end{equation}
                
                \begin{equation} \label{equation: quatum_affine_sl_2_oscillator_Theta_Gamma_generating_series}
                    \Beta(z) := \beta + \frac{(q - q^{-1}) b_0}{2\hbar} \log(z) + \sum_{n \in \Z \setminus \{0\}} \frac{b[n]}{[n]_q} z^{-n - 1}
                    \quad, \quad
                    \Gamma(z) := \gamma + \frac{(q - q^{-1}) c_0}{2\hbar} \log(z) + \sum_{n \in \Z \setminus \{0\}} \frac{c[n]}{[n]_q} z^{-n - 1}
                \end{equation}
            \begin{remark}[Regarding notations from \cite{frenkel_reshetikhin_affine_QUEs_and_deformed_virasoro_and_finite_W_algebras} and \cite{awata_odake_shiraishi_free_boson_realisation_of_quantum_affine_sl_N}]
                In \cite{frenkel_reshetikhin_affine_QUEs_and_deformed_virasoro_and_finite_W_algebras}, the elements $\beta, \gamma$ are denoted by $p_b, p_c$ respectively, and the series $B^{\pm}(z), C^{\pm}(z), \Beta(z)$, and $\Gamma(z)$ are denoted by $b^{\pm}(z), c^{\pm}(z), b(z)$, and $c(z)$ respectively. Also, in comparison with the generators $a[n]$ from \cite{awata_odake_shiraishi_free_boson_realisation_of_quantum_affine_sl_N}, the generators $\lambda[n]$ differ by $\lambda[n] = \frac{q - q^{-1}}{q^n + q^{-n}} a[n]$. This leads to $\Lambda^{\pm}(z) \Lambda^{\pm}(z q^{\pm 2}) = \exp\left( a^{\pm}( z q^{\pm 1} ) \right)$. 
            \end{remark}
            
            In the classical limit $\hbar \to 0$, the deformed oscillator algebra $\calA_{\hbar}(\hat{\sl}_2)_{\level}$ becomes the \say{classical oscillator algebra} (at the same level $\level \in \bbC$), which is the associative $\bbC$-algebra generated by the coefficients of the series:
                $$\chi(z) := \sum_{n \in \Z} \chi[n] z^{-n - 1}$$
                $$a(z) := \sum_{n \in \Z} a[n] z^{-n - 1} \quad, \quad a^{\dagger}(z) := \sum_{n \in \Z} a^{\dagger}[n] z^{-n - 1}$$
            which obeys the classical Heisenberg OPE (cf. \cite[Subsection 2.3.2]{frenkel_ben_zvi_vertex_algebras_and_algebraic_curves}):
                \begin{equation} \label{equation: classical_heisenberg_OPE}
                    [\chi(z), \chi(w)] \sim 2 \del_w \1(z - w)
                \end{equation}
            while the series $b(z)$ and $c(z)$ will become $a(z)$ and $z^{\dagger}(z)$ respectively, between which there are the following OPE relations:
                \begin{equation} \label{equation: classical_weyl_OPE}
                    [a(z), a(w)] = [a^{\dagger}(z), a^{\dagger}(w)] = 0, \quad, \quad [a(z), a^{\dagger}(w)] \sim \level \1(zw^{-1})
                \end{equation}
            Thus, $\lambda(z)^{\pm}$ should appear in the free boson realisations of the Cartan-like currents $K^{\pm}(z)$ of $\calU_{\hbar}(\hat{\sl}_2)_{\level}$, while $b(z)$ and $c(z)$ should appear in the free boson realisations of the \say{upper and lower triangular} currents $E(z), F(z)$. The idea behind the construction of the free boson realisation $\freeboson_{\hbar, \level}$ is thus, that $\calU_{\hbar}(\hat{\sl}_2)_{\level}$ can be embedded into $\calA_{\hbar}^{\current}(\hat{\sl}_2)_{\level}$ because the latter is generated by the same number of similarly behaving generating series, but those are subjected to fewer relations than the generating series of the former, and thus $\calA_{\hbar}^{\current}(\hat{\sl}_2)_{\level}$ is \say{sufficiently large}.

            \begin{remark}
                There are two Cartan-like currents $K^{\pm}(z)$ amongst the generating series for $\calU_{\hbar}(\hat{\sl}_2)_{\level}$, and hence two similarly behaving generating series $\lambda(z)^{\pm}$ for the deformed oscillator algebra $\calA_{\hbar}^{\current}(\hat{\sl}_2)_{\level}$, because $\calU_{\hbar}(\hat{\sl}_2)$ arises as a quantum double. In particular, in order to construct the underlying Lie bialgebra structure on the affine Kac-Moody algebra $\hat{\sl}_2$, it is necessary to \say{double-up} the Cartan subalgebra so that a non-degenerate invariant bilinear form could be constructed on the resulting enlargement of $\hat{\sl}_2$, and thus a Manin triple can be constructed. See \cite{etingof_kazhdan_quantisation_6} for more details. 
            \end{remark}

            \begin{remark}
                The factor of $2$ appears in equation \eqref{equation: classical_heisenberg_OPE} because we can choose a symmetric, non-degenerate, and invariant bilinear form $(\cdot, \cdot)_{\sl_2}$ on $\sl_2$ so that $( \chi[m], \chi[n] )_{\sl_2} = 2 \delta_{m + n, 0}$ for all $m, n \in \Z$. In general, it is important to keep track of such a bilinear form on $\bar{\g}$, since it determines the Cartan matrix of $\bar{\g}$, and hence of its affinisastion $\g$ (see \cite[Chapter 7]{kac_infinite_dimensional_lie_algebras}).
            \end{remark}

            To be a bit more precise, and also to explain the need for keeping track of the level parameter $\level \in \bbC$, let us first introduce the \say{Fock space} of $\calA_{\hbar}^{\current}(\hat{\sl}_2)_{\level}$. This is an instance of the general vacuum module construction, and is the left-$\calA_{\hbar}^{\current}(\hat{\sl}_2)_{\level}$-module whose underlying $\bbC[\![\hbar]\!]$-module is isomorphic to that of the subalgebra of $\calA_{\hbar}^{\current}(\hat{\sl}_2)_{\level}$ spanned by monomials in the coefficients $\lambda[n], b[n], c[n]$ with $n $
    
        \subsection{Free boson realisations for QSPs}
            It is not straightforward to apply this kind of thinking to $\calB_{\hbar}(\calK)$, however. The reason is that the Ding-Frenkel isomorphism $\calU_{\hbar}^{\calR}(\Loop \sl_2) \xrightarrow[]{\cong} \calU_{\hbar}^{\current}(\Loop \sl_2)$ relies on us knowing an explicit Gauss decomposition for the monodromy matrices $L^{\pm}(z) := \sum_{1 \leq i, j \leq N} \sum_{r \geq 0} L^{\pm}_{i, j}[r] z^{\pm r}$ generating $\calU_{\hbar}^{\calR}(\Loop \sl_2)$, while we do not know of such a formula for the monodromy matrix $B(z) := \sum_{1 \leq i, j \leq N} \sum_{r \geq 0} B_{i, j}[r] z^{-r}$ generating $\calB_{\hbar}(z)$. 

        \subsection{Technical difficulties}

    \section{Future works}
    
    \addcontentsline{toc}{section}{References}
    \printbibliography

\end{document}