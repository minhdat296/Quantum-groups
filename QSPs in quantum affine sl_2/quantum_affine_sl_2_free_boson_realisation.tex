\section{Free field realisations for affine quantum algebras}
    \subsection{Quantum algebras}
        Quantum algebras that arise as (quotients of) quantum doubles, such as quantum Kac-Moody algebras, quantum affine algebras, quantum loop algebras, (extended) double Yangians, will sometimes be referred to collectively as \textbf{affine quantum algebras} or something of the sort; this is because there are constructions that will work identically between the trigonometric and rational cases and will only depend qualitatively on the R-matrix. Confusion will hopefully be avoided with context.
    
        For us, $\hbar$ shall be a formal variable for the moment. We write $q := e^{\hbar}$, and when $\hbar$ is specialised to numerical values, we assume that $\hbar \in \bbC \setminus 2\pi i \Q$ so that $q$ is not a root of unity. We will also be needing the following $q$-number expressions:
            $$[n]_q := \frac{q^n - q^{-n}}{q - q^{-1}} \quad, \quad [n]_q! := \prod_{k = 1}^n [k]_q$$
            $$\exp_q(z) := \sum_{n \geq 0} \frac{1}{[n]_q!} z^n$$
        They are to be interpreted as elements of $\bbC[q^{\pm 1}]$.

        Trigonometric R-matrices (and K-matrices\footnote{More on these so-called \say{K-matrices} later.}) will usually depend multiplicatively on spectral parameters $z, w$, while rational ones will depend additively on spectral parameters $u := \log(z), v := \log(w)$. \textit{All R-matrices are assumed to be unitary}, which is to say that:
            \begin{equation} \label{equation: R_matrix_unitarity}
                \calR(z, w) \calR(w, z)_{2, 1} = 1
            \end{equation}
        and hence also invertible in particular (and likewise for K-matrices). We shall usually assume that trigonometric R-matrices depend only on $zw^{-1}$ (and that rational R-matrices depend only on $u - v$), and consequently, swapping variables has the following effect on the R-matrices:
            \begin{equation} \label{equation: swapping_spectral_parameters_in_R_matrices}
                \calR_{\trigonometric}(wz^{-1}) = \calR_{\trigonometric}(zw^{-1})^{-1}_{2, 1}
                \quad, \quad
                \calR_{\rational}(v - u) = \calR_{\rational}(u - v)^{-1}_{2, 1}
            \end{equation}
    
        Suppose that $\calR_{\trigonometric}(zw^{-1})$ is a trigonometric solution to the quantum Yang-Baxter equation (QYBE) with rational degeneration $\calR_{\rational}(u - v)$, and for their classical limits, let us write $\calr_{\trigonometric}(zw^{-1})$ and $\calr_{\rational}(u - v)$ respectively. Next, let us denote the corresponding quasi-triangular (topological) Hopf algebras as in \cite{etingof_kazhdan_quantisation_1} (see also \cite{etingof_kazhdan_quantisation_1} and \cite{etingof_kazhdan_quantisation_6}) by:
            $$\calU_{\hbar}(\calR)$$
        and their classical limits by:
            $$\g := \g(\calr)$$
        \begin{itemize}
            \item In the trigonometric case, $\g(\calr_{\trigonometric})$ is a Kac-Moody algebra (with derivation), but let us assume further that it is of a untwisted affine type $\sfX$ in the classification of \cite[Chapter 4]{kac_infinite_dimensional_lie_algebras}. Consequently, there exists a finite-dimensional simple Lie algebra $\bar{\g}$ such that:
                $$\g \cong ( \Loop \bar{\g} \oplus \bbC \level ) \rtimes \bbC \del_{\trigonometric}$$
            wherein $\Loop \bar{\g} := \g \tensor \bbC[t^{\pm 1}]$, on which $\del_{\trigonometric}$ acts by $\id \tensor t\frac{d}{dt}$, and $\level$ is central. In particular, this means that the loop algebra $\Loop \bar{\g}$ is a Lie sub-bialgebra of $\g(\calr_{\trigonometric})$, and hence the quantum Kac-Moody algebra $\calU_{\hbar}^{\calR}(\g) := \calU_{\hbar}(\calR_{\trigonometric})$ admits the quantum loop algebra $\calU_{\hbar}^{\calR}(\Loop \bar{\g})$ (both in the R-matrix presentation) as a Hopf subalgebra.

            Moreover, by fixing a non-degenerate, symmetric, and invariant bilinear form $(\cdot, \cdot)_{\bar{\g}}$ on the finite-type Lie algebra $\bar{\g}$, one obtains a similar bilinear form $(\cdot, \cdot)_{\g(\calr_{\trigonometric})}$ on the Kac-Moody algebra $\g := \g(\calr_{\trigonometric})$, determined by:
                $$(x \tensor t^m, y \tensor t^n)_{\g(\calr_{\trigonometric})} = (x, y)_{\bar{\g}} \delta_{m + n, 0} \quad, \quad (\level, \del_{\trigonometric})_{\g(\calr_{\trigonometric})} = 1$$
            Note that when regarded as an element of $\Hom(\Sym^2(\g), \bbC)^{\g}$, the bilinear form $(\cdot, \cdot)_{\g(\calr_{\trigonometric})}$ is of total degree $0$ with respect to the canonical $\Z$-grading. 
            \item In the rational case, we still have that $\g(\calr_{\rational}) \cong ( \Loop \bar{\g} \oplus \bbC \level ) \rtimes \bbC \del_{\rational}$, but now $\del_{\rational}$ acts on $\Loop \bar{\g}$ by $\id \tensor \frac{d}{dt}$. As such, there exists instead an invariant bilinear form $(\cdot, \cdot)_{\g(\calr_{\rational})}$ of total degree $-1$, given by:
                $$(x \tensor t^m, y \tensor t^n)_{\g(\calr_{\rational})} = x, y)_{\bar{\g}} \delta_{m + n, -1} \quad, \quad (\level, \del_{\rational})_{\g(\calr_{\rational})} = 1$$
        \end{itemize}

        $\calU_{\hbar}^{\calR}(\g)$ comes equipped with representations $\pi_z^{\pm}: \calU_{\hbar}^{\calR}(\g) \to \End(\bbV_N)[\![z^{\pm 1}]\!]$, using which we can write down the defining relations for $\calU_{\hbar}^{\calR}(\g)$, using coefficients of the matrix entries of the following so-called \textbf{monodromy matrices}:
            $$L^-(z) := (\pi_z^- \tensor \id)( \calR_{1, 2} ) \in \End(\bbV_N \tensor \bbV_N) \tensor \calU_{\hbar}^{\calR}(\g)[\![z^{-1}]\!]$$
            $$L^+(z) := (\pi_z^+ \tensor \id)( \calR_{2, 1}^{-1} ) \in \End(\bbV_N \tensor \bbV_N) \tensor \calU_{\hbar}^{\calR}(\g)[\![z]\!]$$
        along with a central generator $\level$ and a derivation $\del$. The aforementioned relations take place in $\End(\bbV_N \tensor \bbV_N) \tensor \calU_{\hbar}^{\calR}(\g)[\![z^{\pm 1}, w^{\pm 1}]\!]$ and take the following form:
            $$[\level, L^{\pm}(z)] = [\level, \del] = 0 \quad, \quad [\del, L^{\pm}(z)] = \frac{d}{dz} L^{\pm}(z)$$
            \begin{equation} \label{equation: FRT_relations}
                \begin{gathered}
                    \calR(zw^{-1})_{1, 2} L^{\pm}(z)_{1, 3} L^{\pm}(w)_{2, 3} = L^{\pm}(w)_{2, 3} L^{\pm}(z)_{1, 3} \calR(zw^{-1})_{1, 2}
                    \\
                    \calR(zw^{-1})_{1, 2} L^+(z)_{1, 3} L^-(w)_{2, 3} = L^-(w)_{2, 3} L^+(z)_{1, 3} \calR(zw^{-1})_{1, 2}
                \end{gathered}
            \end{equation}
        
        Recall also, that there are two other presentations for the quantum Kac-Moody and quantum loop algebras, namely by Drinfeld currents and Kac-Moody generators; let us denote these algebras by $\calU_{\hbar}^{\current}(\g), \calU_{\hbar}^{\current}(\Loop \bar{\g})$ and $\calU_{\hbar}^{\KM}(\g), \calU_{\hbar}^{\KM}(\Loop \bar{\g})$ respectively. For our purposes here, it is sufficient to know only that $\calU_{\hbar}^{\current}(\g)$ is generated a central generator $\level$ (acting as $q^{\level}$ on representations) and a derivation $\del$, along with the coefficients of generating series:
            $$E_i(z) := \sum_{n \in \Z} E_i[n] z^{-n} \quad, \quad F_i(z) := \sum_{n \in \Z} F_i[n] z^{-n} \quad, \quad K^{\pm}_i(z) := \sum_{n \geq 0} K^{\pm}_i[\mp n] z^{\pm n} \quad, \quad i \in \simpleroots$$
        while $\calU_{\hbar}^{\KM}(\g)$ is generated by the same central and derivation elements, along with generators:
            $$E_i, F_i, q^{\pm H_i} \quad, \quad i \in \simpleroots$$
        with $\simpleroots$ denoting a set of simple root for the Kac-Moody algebra $\g$; the quantum loop subalgebras are generated likewise. For more information on the commutation relations between these generators, see e.g. \cite[Section 2]{ding_pakuliak_khoroshkin_integral_formula_for_R_matrices_of_affine_QUEs} or \cite[Section 3]{frenkel_reshetikhin_affine_QUEs_and_deformed_virasoro_and_finite_W_algebras}. 
        
        The three presentations turn out to be equivalent to one another. By theorems of Ding-I. Frenkel and Beck (later on clarified by E. Frenkel and Mukhin), we know that there are algebra isomorphisms:
            $$\calU_{\hbar}^{\calR}(\g) \xrightarrow[]{\DF} \calU_{\hbar}^{\current}(\g) \xrightarrow[]{\Beck} \calU_{\hbar}^{\KM}(\g)$$
        which then get restricted down to isomorphisms:
            $$\calU_{\hbar}^{\calR}(\Loop \bar{\g}) \xrightarrow[]{\cong} \calU_{\hbar}^{\current}(\Loop \bar{\g}) \xrightarrow[]{\cong} \calU_{\hbar}^{\KM}(\Loop \bar{\g})$$
        \begin{remark} \label{remark: gaussian_decompositions_of_monodromy_matrices}
            The existence (and uniqueness) of the first isomorphism $\DF$ relies on the existence (and again, uniqueness) of the Gaussian decomposition of the monodromy matrices $L^{\pm}(z)$, so for the purpose of writing down an explicit formula for $\DF$, \textit{having an explicit formula for the aforementioned Gaussian decomposition is a prerequisite}. For instance, in the case wherein $\g = \hat{\sl}_2$, the following Gaussian decompositions exist in the algebra $\calU_{\hbar}^{\calR}(\hat{\sl}_2)$:
                \begin{equation} \label{equation: quantum_affine_sl_2_monodromy_matrices_gaussian_decompositions}
                    L^{\pm}(z)
                    =
                    \begin{pmatrix}
                        1 & 0
                        \\
                        e^{\pm}(z) & 1
                    \end{pmatrix}
                    \begin{pmatrix}
                        k_0^{\pm}(z) & 0
                        \\
                        0 & k_1^{\pm}(z)
                    \end{pmatrix}
                    \begin{pmatrix}
                        1 & f^{\pm}(z)
                        \\
                        0 & 1
                    \end{pmatrix}
                    =
                    \begin{pmatrix}
                        k_0^{\pm}(z) & k_0^{\pm}(z) f^{\pm}(z)
                        \\
                        e^{\pm}(z) k_0^{\pm}(z) & k_1^{\pm}(z) + e^{\pm}(z) k_0^{\pm}(z) f^{\pm}(z)
                    \end{pmatrix}
                \end{equation}
            and consequently, there is an algebra isomorphism:
            \begin{equation} \label{equation: ding_frenkel_isomorphism_quantum_affine_sl_2}
                \DF: \calU_{\hbar}^{\calR}(\hat{\sl}_2)_{\level} \xrightarrow[]{\cong} \calU_{\hbar}^{\current}(\hat{\sl}_2)_{\level}
            \end{equation}
            determined by:
                \begin{equation}
                    E(z) = \DF( e^+(z q^{\frac12 \level}) - e^-(z q^{-\frac12 \level}) ) \quad, \quad F(z) = \DF( f^+(z q^{-\frac12 \level}) - f^-(z q^{\frac12 \level}) )
                \end{equation}
                \begin{equation}
                    K_i^{\pm}(z) = \DF( k_i^{\pm}(z) )
                \end{equation}
        \end{remark}

    \subsection{Free field realisations for affine quantum algebras in the R-matrix presentation}

    \subsection{Free field realisations for affine quantum algebras in Drinfeld's current presentation}