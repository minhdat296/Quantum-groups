\section{\texorpdfstring{Free boson realisation for trigonometric and rational quantum affine $\sl_2$-algebras}{}}
    \subsection{\texorpdfstring{The quantum Kac-Moody algebra $\calU_{\hbar}(\hat{\sl}_2)$ and its universal R-matrix}{}}
        We write:
            $$\hat{\sl}_2$$
        to mean the untwisted affine Kac-Moody algebra associated to $\sl_2$, i.e. the Kac-Moody algebra whose Dynkin diagram is $\alpha_0 \Rightarrow \alpha_1$ (cf. \cite[Chapters 4 and 7]{kac_infinite_dimensional_lie_algebras}); we recall that the corresponding Cartan matrix is $\hat{C} = \begin{pmatrix} 2 & -2 \\ -2 & 2 \end{pmatrix}$. Denote the corresponding root system by $\rootsystem$ and fix a subset of simple roots $\simpleroots := \{\alpha_0, \alpha_1\}$ therein.
        
        From \cite[Chapter 7]{kac_infinite_dimensional_lie_algebras}, we know that the affine Kac-Moody algebra $\hat{\sl}_2$ has a so-called \say{loop realisation} in the form:
            $$\hat{\sl}_2 \cong ( \Loop \sl_2 \oplus \bbC \level ) \rtimes \bbC \del_{\aff}$$
        wherein $\Loop \sl_2 := \sl_2[t^{\pm 1}]$, with respect to which the defining commutation relations are:
            $$[J f(t), J'g(t)]_{\hat{\sl}_2} = [J, J']_{\sl_2} f(t) g(t) + (J, J')_{\sl_2} \Res_{t = 0} g(t) df(t)$$
            $$[\level, \hat{\sl}_2]_{\hat{\sl}_2} = 0$$
            $$[\del_{\aff}, J f(t)]_{\hat{\sl}_2} = J t\frac{d}{dt} f(t) \quad$$
        given for all $J, J' \in \sl_2$ and $f(t), g(t) \in \bbC[t^{\pm 1}]$. It is easily seen that the derived subalgebra of $\hat{\sl}_2$ is the Lie algebra:
            $$\tilde{\sl}_2 \cong \Loop \sl_2 \oplus \bbC \level$$
        which also happens to be isomorphic to the universal central extension of $\Loop \sl_2$ (see \cite{kassel_universal_central_extensions_of_lie_algebras}); this latter Lie algebra will sometimes be referred to as the \say{affinisation} of $\sl_2$.
    
        We begin by recalling a construction of a Hopf algebra that quantises\footnote{In the sense of \cite{etingof_kazhdan_quantisation_1}; see also \cite{etingof_kazhdan_quantisation_6}.} the standard Lie bialgebra structure on the affine Kac-Moody algebra $\hat{\sl}_2$ due to Fadeev-Reshetkhin-Takhtajan (FRT), also known as the RLL construction, so-called due to the form of the defining relations. For this, we need as input an R-matrix $\calR(z, w)$ depending on two spectral parameters $z, w$.

        For us, $\hbar$ shall be a formal variable for the moment. We write $q := e^{\hbar}$, and when $\hbar$ is specialised to numerical values, we assume that this quantity is not a root of unity, i.e. $\hbar \in \bbC \setminus 2\pi i \Q$.
        
        \begin{definition}[Quantum $\hat{\sl}_2$: R-matrix presentation] \label{def: quantum_affine_sl_2_R_matrix_presentation}
            The algebra:
                $$\calU_{\hbar}^{\calR}(\hat{\sl}_2)_{\level}$$
        \end{definition}

        At the same time, there are two other realisations of supposedly the same Hopf algebra that quantises the standard Lie bialgebra structure on $\hat{\sl}_2$. They shall be referred to as the presentations by Kac-Moody\footnote{... due to its similarity the Chevalley-Serre-style presentation for $\hat{\sl}_2$ in terms of simple root vectors given by Kac and Moody; for historical reasons, this is also known as the Drinfeld-Jimbo presentation.} and Drinfeld's current generators, respectively. See \cite[Subsection 2.1]{ding_pakuliak_khoroshkin_factorisation_of_the_R_matrix_of_quantum_affine_sl_2} for more details on these two realisations. To be able to write down these two presentations, we will be needing the following $q$-number expressions:
            $$[n]_q := \frac{q^n - q^{-n}}{q - q^{-1}} \quad, \quad [n]_q! := \prod_{k = 1}^n [k]_q$$
                $$\exp_q(z) := \sum_{n \geq 0} \frac{1}{[n]_q!} z^n$$
        They are to be interpreted as elements of $\bbC[q^{\pm 1}]$.
        \begin{definition}[Quantum $\hat{\sl}_2$: Kac-Moody presentation] \label{def: quantum_affine_sl_2_kac_moody_presentation}
            The algebra:
                $$\calU_{\hbar}^{\KM}(\hat{\sl}_2)_{\level}$$
            is the associative algebra generated by the set:
                $$$$
        \end{definition}
        \begin{definition}[Quantum $\hat{\sl}_2$: Drinfeld's current presentation] \label{def: quantum_affine_sl_2_drinfeld_current_presentation}
            The algebra:
                $$\calU_{\hbar}^{\current}(\hat{\sl}_2)_{\level}$$
            is the associative algebra generated by the set:
                $$\{ E[n], F[n] \}_{n \in \Z} \cup \{ K_i^{\pm}[n] \}_{(i, n) \in \simpleroots \x \Z_{\geq 0}} \cup \{ \del \}$$
            whose elements are subjected to the following relations:
                \todo[inline]{Drinfeld current relations}
            wherein we have written $E(z) := \sum_{n \in \Z} E[n] z^{-n}, F(z) := \sum_{n \in \Z} F[n] z^{-n}, K_i^{\pm}(z) := \sum_{n \geq 0} K_i^{\pm}[\mp n] z^{\pm n}$. It is commonly referred to as \textbf{quantum $\hat{\sl}_2$} at level $\level \in \bbC$ in \textbf{Drinfeld's current presentation}.
        \end{definition}

        The two realisations of quantum $\hat{\sl}_2$ are known to be equivalent, in the sense that they give rise to two isomorphic algebras. This is a special case of a result of Ding and Frenkel, whose main insight into the matter was that the matrices $L^{\pm}(z)$ admit a Gaussian decomposition\footnote{Note that this is not a trivial fact, for the entries of these matrices come from the noncommutative algebra $\calU_{\hbar}^{\calR}(\hat{\sl}_2)$.} in which the entries of the lower-triangular/diagonal/upper-triangular factors can be used to recover the series $E(z), F(z), K_i^{\pm}(z)$ from definition \ref{def: quantum_affine_sl_2_drinfeld_current_presentation}.
        \begin{lemma}[The Ding-Frenkel Isomorphism: the quantum $\hat{\sl}_2$ case] \label{lemma: ding_frenkel_isomorphism_quantum_affine_sl_2}
            At any level $\level \in \bbC$, the following Gaussian decomposition egammasts in the algebra $\calU_{\hbar}^{\calR}(\hat{\sl}_2)_{\level}$:
                \begin{equation} \label{equation: quantum_affine_sl_2_monodromy_matrices_gaussian_decompositions}
                    \begin{aligned}
                        L^{\pm}(z)
                        & =
                            \begin{pmatrix}
                                1 & 0
                                \\
                                e^{\pm}(z) & 1
                            \end{pmatrix}
                            \begin{pmatrix}
                                k_0^{\pm}(z) & 0
                                \\
                                0 & k_1^{\pm}(z)
                            \end{pmatrix}
                            \begin{pmatrix}
                                1 & f^{\pm}(z)
                                \\
                                0 & 1
                            \end{pmatrix}
                        \\
                        & =
                            \begin{pmatrix}
                                k_0^{\pm}(z) & k_0^{\pm}(z) f^{\pm}(z)
                                \\
                                e^{\pm}(z) k_0^{\pm}(z) & k_1^{\pm}(z) + e^{\pm}(z) k_0^{\pm}(z) f^{\pm}(z)
                            \end{pmatrix}
                    \end{aligned}
                \end{equation}
            and consequently, there is an algebra isomorphism:
                \begin{equation} \label{equation: ding_frenkel_isomorphism_quantum_affine_sl_2}
                    \DF: \calU_{\hbar}^{\calR}(\hat{\sl}_2)_{\level} \xrightarrow[]{\cong} \calU_{\hbar}^{\current}(\hat{\sl}_2)_{\level}
                \end{equation}
            determined by:
                \begin{equation}
                    E(z) = \DF( e^+(z q^{\frac12 \level}) - e^-(z q^{-\frac12 \level}) ) \quad, \quad F(z) = \DF( f^+(z q^{-\frac12 \level}) - f^-(z q^{\frac12 \level}) )
                \end{equation}
                \begin{equation}
                    K_i^{\pm}(z) = \DF( k_i^{\pm}(z) )
                \end{equation}
        \end{lemma}
            \begin{proof}
                
            \end{proof}
        Thanks to this result, we are able to regard $\calR(z, w)$ as \textit{the} universal R-matrix of $\calU_{\hbar}^{\current}(\hat{\sl}_2)$.

    \subsection{\texorpdfstring{Free boson realisation for $\calU_{\hbar}(\hat{\sl}_2)$ in Drinfeld's current presentation}{}}
        In \cite[Section 7]{frenkel_reshetikhin_affine_QUEs_and_deformed_virasoro_and_finite_W_algebras}, the following $q$-deformed analogue of the oscillator algebra was given. We hesitate to use the descriptor \say{quantum} at this point, for the role of the universal R-matrix of $\calU_{\hbar}(\hat{\sl}_2)$ is not yet clear.
        \begin{definition}[Deformed oscillator algebra of $\hat{\sl}_2$: Drinfeld's current presentation] \label{def: q_deformed_affine_sl_2_oscillator_algebra_drinfeld_current_presentation}
            In Drinfeld's current presentation, the \textbf{deformed oscillator algebra} at a level $\level \in \bbC$ of the affine Kac-Moody algebra $\hat{\sl}_2$ is the associative $\bbC[\![\hbar]\!]$-algebra:
                $$A_{\hbar}^{\current}(\hat{\sl}_2)_{\level}$$
            generated by the elements:
                $$\lambda[n], b[n], c[n] \quad, \quad n \in \Z \setminus \{0\}$$
                $$e^{\pm \frac{\lambda_0}{2}}, e^{\pm \frac{(q - q^{-1}) b_0}{2}}, e^{\pm \frac{(q - q^{-1}) c_0}{2}}$$
                $$\beta, \gamma$$
                $$\del$$
            which are subjected to the following relations, given for all $m, n \in \Z \setminus \{0\}$:
                \begin{equation} \label{equation: q_deformed_affine_sl_2_oscillator_lambda_commutators}
                    [ \lambda[m], \lambda[n] ] = \frac1n \frac{ [(\level + \dualcoxeter) n]_q [n]_q^2 }{ [2n]_q } \delta_{m + n, 0}
                \end{equation}
                
                \begin{equation} \label{equation: q_deformed_affine_sl_2_oscillator_bc_commutators}
                    [ b[m], b[n] ] = -\frac1n [n]_q^2 \delta_{m + n, 0} \quad, \quad [ c[m], c[n] ] = \frac1n [n]_q^2 \delta_{m + n, 0}
                \end{equation}
                
                \begin{equation} \label{equation: q_deformed_affine_sl_2_oscillator_beta_gamma_commutators}
                    [ b[0], \beta ] = -\frac{q - q^{-1}}{2\hbar} \quad, \quad [ c[0], \gamma ] = \frac{q - q^{-1}}{2\hbar}
                \end{equation}
            Here, $\dualcoxeter = -2$ is the dual Coxter number.
        \end{definition}
        
        We note right away that the algebra $A_{\hbar}^{\current}(\hat{\sl}_2)_{\level}$ has a natural $\Z$-grading given by:
            $$\deg \lambda[n] = \deg b[n] = \deg c[n] = n \quad, \quad n \in \Z$$
            $$\deg \beta = \deg \gamma = 0, \quad \deg \del = 0$$
            $$\deg \hbar = 1$$
        With respect to this grading, we can define left-$A_{\hbar}^{\current}(\hat{\sl}_2)_{\level}$-ideals $I_n$ spanned by monomials in $\lambda[n], b[n], c[n]$ which are of total degrees $\geq n$. This defines a cofiltered diagram of algebras $\left\{ A_{\hbar}^{\current}(\hat{\sl}_2)_{\level}/I_n \right\}_{n \geq 1}$, whose limit in the category of associative algebras shall be denoted by:
            $$\calA_{\hbar}^{\current}(\hat{\sl}_2)_{\level} := \projlim_{n \geq 1} A_{\hbar}^{\current}(\hat{\sl}_2)_{\level}/I_n$$

        To fix terminologies, let us also write down the following.
        \begin{definition}[Classical oscillator algebra of $\hat{\sl}_2$] \label{def: classical_affine_sl_2_oscillator_algebra_drinfeld_current_presentation}
            The \textbf{classical oscillator algebra} at a level $\level \in \bbC$ of the affine Kac-Moody algebra $\hat{\sl}_2$ is the associative $\bbC$-algebra:
                $$A_0(\hat{\sl}_2)_{\level}$$
            generated by the elements:
                $$\chi[n], a[n], a^{\dagger}[n] \quad, \quad n \in \Z$$
                $$\level$$
            which are subjected to the following relations, given for all $m, n \in \Z$:
                \begin{equation}
                    [ a[m], a[n] ] = [ a^{\dagger}[m], a^{\dagger}[n] ] = 0 \quad, \quad [ a[m], a^{\dagger}[n] ] = \delta_{m + n} \level
                \end{equation}
                \begin{equation}
                    [ \chi[m], \chi[n] ] = 2m \delta_{m + n, 0} \level
                \end{equation}
        \end{definition}

        \begin{remark}[Deformed oscillator algebra at the critical level] \label{remark: q_deformed_affine_sl_2_oscillator_algebra_drinfeld_current_presentation_at_critical_level}
            It is worth noting that the following exceptional phenomenon occurs at the so-called \textbf{critical level} $\level = -\dualcoxeter = -2$. There, relation \eqref{equation: q_deformed_affine_sl_2_oscillator_lambda_commutators} becomes:
                $$[\lambda[m], \lambda[n]] = 0$$
            for all $m, n \in \Z \setminus \{0\}$, and thus the subalgebra of $\calA_{\hbar}^{\current}(\hat{\sl}_2)_{\level}$ generated by the elements of the set $\{ \lambda[n] \}_{n \in \Z \setminus \{0\}}$ becomes commutative at $\level = -2$.
        \end{remark}
        \begin{convention}
            We will usually denote the critical level more descriptively by $\level_{\crit}$.
        \end{convention}
        
        Usually, one packages the generators above into series:
            $$\lambda(z)^{\pm} := \sum_{n \geq 1} \lambda[\pm n] z^{\mp n}$$
            $$
                b(z)^{\pm} := \sum_{n \geq 1} b[\pm n] z^{\mp n}
                \quad, \quad
                c(z) := \sum_{n \geq 0} c[\pm n] z^{\mp n}
            $$
        using which we form the following series generating series for the algebra $A_{\hbar}^{\current}(\hat{\sl}_2)_{\level}$ and its completion $\calA_{\hbar}^{\current}(\hat{\sl}_2)_{\level}$:
            \begin{equation} \label{equation: q_deformed_affine_sl_2_oscillator_Lambda_generating_series}
                \Lambda^{\pm}(z) := e^{ \pm \frac{\lambda_0}{2} } e^{ \pm \lambda(z)^{\pm} }
            \end{equation}
            
            \begin{equation} \label{equation: q_deformed_affine_sl_2_oscillator_BC_generating_series}
                B^{\pm}(z) := \pm (q - q^{-1}) \left( \frac12 b[0] + b(z)^{\pm} \right)
                \quad, \quad
                C^{\pm}(z) := \pm (q - q^{-1}) \left( \frac12 c[0] + c(z)^{\pm} \right)
            \end{equation}
            
            \begin{equation} \label{equation: quatum_affine_sl_2_oscillator_Theta_Gamma_generating_series}
                \Beta(z) := \beta + \frac{(q - q^{-1}) b_0}{2\hbar} \log(z) + \sum_{n \in \Z \setminus \{0\}} \frac{b[n]}{[n]_q} z^{-n}
                \quad, \quad
                \Gamma(z) := \gamma + \frac{(q - q^{-1}) c_0}{2\hbar} \log(z) + \sum_{n \in \Z \setminus \{0\}} \frac{c[n]}{[n]_q} z^{-n}
            \end{equation}
        \begin{remark}[Regarding notations from \cite{frenkel_reshetikhin_affine_QUEs_and_deformed_virasoro_and_finite_W_algebras} and \cite{awata_odake_shiraishi_free_boson_realisation_of_quantum_affine_sl_N}]
            In \cite{frenkel_reshetikhin_affine_QUEs_and_deformed_virasoro_and_finite_W_algebras}, the elements $\beta, \gamma$ are denoted by $p_b, p_c$ respectively, and the series $B^{\pm}(z), C^{\pm}(z), \Beta(z)$, and $\Gamma(z)$ are denoted by $b^{\pm}(z), c^{\pm}(z), b(z)$, and $c(z)$ respectively. Also, in comparison with the generators $a[n]$ from \cite{awata_odake_shiraishi_free_boson_realisation_of_quantum_affine_sl_N}, the generators $\lambda[n]$ differ by $\lambda[n] = \frac{q - q^{-1}}{q^n + q^{-n}} a[n]$. This leads to $\Lambda^{\pm}(z) \Lambda^{\pm}(z q^{\pm 2}) = \exp\left( a^{\pm}( z q^{\pm 1} ) \right)$. 
        \end{remark}

        It was noted in \cite{frenkel_reshetikhin_affine_QUEs_and_deformed_virasoro_and_finite_W_algebras} that the following relations between the generating series introduced above hold. There, it was mentioned that this can be seen via \cite{awata_odake_shiraishi_free_boson_realisation_of_quantum_affine_sl_N}, but no actual proof was given, so we supply one here.
        \begin{lemma} \label{lemma: q_deformed_affine_sl_2_oscillator_generating_series_commutation_relations}
            The series in \eqref{equation: q_deformed_affine_sl_2_oscillator_Lambda_generating_series}, \eqref{equation: q_deformed_affine_sl_2_oscillator_BC_generating_series}, and \eqref{equation: quatum_affine_sl_2_oscillator_Theta_Gamma_generating_series} satisfy the following commutation relation:
                \begin{equation} \label{equation: q_deformed_affine_sl_2_oscillator_Lambda_generating_series_commutators}
                    \Lambda^+(z) \Lambda^-(w) = \frac{ g\left( \frac{w}{z} q^{-k - 2} \right) }{ g\left( \frac{w}{z} q^{k + 2} \right) } \Lambda^-(w) \Lambda^+(z)
                \end{equation}
            (wherein $g$ is the normalising factor for the R-matrix $\calR$), as well as the relations:
                \begin{equation} \label{equation: q_deformed_affine_sl_2_oscillator_BC_generating_series_commutators}
                    e^{ B^+(z) } \order{ e^{\Beta(w)} } = \frac{ 1 - \frac{w}{z} q }{ q - \frac{w}{z} } \order{ e^{\Beta(w)} } e^{ B^+(z) }
                    \quad, \quad
                    \order{ e^{ \Beta(z) } } e^{ B^-(w) } = \frac{ 1 - \frac{w}{z} q }{ q - \frac{w}{z} } e^{B^-(w)} \order{ e^{ \Beta(z) } }
                \end{equation}
                
                \begin{equation} \label{equation: quatum_affine_sl_2_oscillator_Theta_Gamma_generating_series_commutators}
                    e^{B^+(z)} e^{B^-(w)} = \frac{ \left(1 - \frac{w}{z}\right)^2 }{ \left( 1 - \frac{w}{z} q^2 \right) \left( 1 - \frac{w}{z} q^{-2} \right) } e^{B^-(w)} e^{B^+(z)}
                    \quad, \quad
                    e^{C^+(z)} e^{C^-(w)} = \frac{ \left( 1 - \frac{w}{z} q^2 \right) \left( 1 - \frac{w}{z} q^{-2} \right) }{ \left(1 - \frac{w}{z}\right)^2 } e^{C^-(w)} e^{C^+(z)}
                \end{equation}
        \end{lemma}
            \begin{proof}
                
            \end{proof}

        \begin{proposition}[PBW basis for $A_{\hbar}^{\current}(\hat{\sl}_2)_{\level}$] \label{prop: q_deformed_affine_sl_2_oscillator_algebra_drinfeld_current_presentation_PBW}
            The deformed oscillator algebra $A_{\hbar}^{\current}(\hat{\sl}_2)_{\level}$ as above is a flat deformation over $\bbC[\![\hbar]\!]$ of the classical oscillator algebra $A_0(\hat{\sl}_2)_{\level}$.
        \end{proposition}
            \begin{proof}
                
            \end{proof}
            
        It was then stated in \cite[Proposition 4]{frenkel_reshetikhin_affine_QUEs_and_deformed_virasoro_and_finite_W_algebras} that the following version of the realisation of $\calU_{\hbar}^{\current}(\hat{\sl}_2)$ in terms of free bosonic fields. The formulae are as in \textit{loc. cit.}, but the authors made no mention of injectivity. 
        \begin{proposition}[Free boson realisation of $\calU_{\hbar}^{\current}(\hat{\sl}_2)$] \label{prop: free_boson_realisation_quantum_affine_sl_2_drinfeld_current_presentation}
            There egammasts an \textit{injective} algebra homomorphism:
                \begin{equation} \label{equation: free_boson_realisation_quantum_affine_sl_2_drinfeld_current_presentation}
                    \freeboson^{\current}_{\hbar, \level}: \calU_{\hbar}^{\current}(\hat{\sl}_2)_{\level} \to \calA_{\hbar}^{\current}(\hat{\sl}_2)_{\level}
                \end{equation}
            that is given on Drinfeld currents by:
                \begin{equation}
                    \freeboson^{\current}_{\hbar, \level}( E(z) ) = -\order{  }
                \end{equation}
                \begin{equation}
                    \freeboson^{\current}_{\hbar, \level}( F(z) ) =
                \end{equation}
                \begin{equation}
                    \freeboson^{\current}_{\hbar, \level}( K_i^+(z) ) = \quad, \quad \freeboson^{\current}_{\hbar, \level}( K_i^-(z) ) =
                \end{equation}
        \end{proposition}
            \begin{proof}
                
            \end{proof}

    \subsection{\texorpdfstring{Free boson realisation for $\calU_{\hbar}(\hat{\sl}_2)$ in the R-matrix presentation}{}}
        What follows work for general unitary solutions $\calR(z, w)$ to the QYBE in $\bbV_N^{\tensor 3}$, but in the end, we are most interested in the universal R-matrices of the quantum Kac-Moody algebra $\calU_{\hbar}(\hat{\sl}_2)$ and the double Yangian $\DY_{\hbar}(\sl_2)$, and for more details, we refer the reader to \cite{ragoucy_vertex_representations_of_R_matrix_quantum_groups}. That said, let us work with a general unitary solution for the moment.
    
        We begin by outlining the idea behind deformed oscillator algebras associated to R-matrices. Let:
            $$\calU_{\hbar}(\calR)$$
        be a quasi-triangular Hopf algebra over $\bbC[\![\hbar]\!]$ associated to $\calR(z, w)$. Examples include quantum Kac-Moody algebras, their quantum affine and quantum loop subalgebras, (extended) double Yangians, etc. Let us fix also a representation on a finite free $\bbC[\![\hbar]\!]$-module $V$ given by:
            $$\pi_z^{\pm}: \calU_{\hbar}(\calR) \to \End(V)[\![z^{\pm 1}]\!]$$
        By the FRT formalism, we know that such algebras are generated by the coefficients of the matrix entries of:
            $$L^+(z) := (\id_{\calU_{\hbar}(\calR)} \tensor \pi_z^{\pm})( \calR^{-1} )_{2, 1} \in \End(\bbV_N \tensor \bbV_N) \tensor \calU_{\hbar}(\calR)[\![z]\!]$$
            $$L^-(z) := (\pi_z^{\pm} \tensor \id_{\calU_{\hbar}(\calR)})( \calR )_{1, 2} \in \End(\bbV_N \tensor \bbV_N) \tensor \calU_{\hbar}(\calR)[\![z^{-1}]\!]$$
        called \textbf{monodromy matrices}, which are subjected to some RLL-type relations, along with perhaps some commutation relations identifying certain generators as being central or derivations (cf. e.g. definition \ref{def: quantum_affine_sl_2_R_matrix_presentation}). The comultiplication on $\calU_{\hbar}(\calR)$ is known to be such that:
            $$\Delta( L^{\pm}(z) ) = L^{\pm}(z) \tensor L^{\pm}(z)$$
        and thus arise as quantum doubles (such as in the case of quantum Kac-Moody algebras or extended double Yangians with derivations) or quotients thereof (e.g. quantum affine and loop algebras, ordinary double Yangians, etc.), in which the Hopf subalgebras:
            $$\calU_{\hbar}^-(\calR) \quad \text{and} \quad \calU_{\hbar}^+(\calR)$$
        generated by the series $L^-(z)$ and $L^+(z)$ respectively - and perhaps along with an additional central or derivation generator - are isotropic.
        \begin{example}
            In particular, when $\calR$ is rational and $\calU_{\hbar}(\calR)$ is either the double Yangian or the centrally extended double Yangian (i.e. the algebra generated by the generators of the double Yangian along with an additional central generator), the Hopf subalgebras $\calU_{\hbar}^{\pm}(\calR)$ will be nothing but the usual (dual) Yangians (cf. \cite[Chapter 1]{molev_yangians_and_classical_lie_algebras} and \cite[Chapter 10]{molev_sugawara_operators_for_classical_lie_algebras}).
        \end{example}

        In light of the above, the operators $L^-(z)$ and $L^+(z)$, respectively, can be reasonably thought of as annihilation and creation operators acting on some Fock space (whose underlying $\bbC[\![\hbar]\!]$-module we will see to be a quotient of $\calU_{\hbar}^+(\calR)$), and we thus are led to define the deformed oscillator algebra in the R-matrix presentation in the following manner. The generators are to satisfy sufficiently few relations so that eventually, one can embed the quasi-triangular Hopf algebra $\calU_{\hbar}(\calR)$ into this deformed oscillator algebra, thus giving us a free field realisation.
        \begin{definition}[Deformed oscillator algebras associated to R-matrices] \label{def: deformed_oscillator_algebras_R_matrix_presentation}
            Let $\calR := \calR(z, w)$ be a general unitary solution to the QYBE in $\bbV_N^{\tensor 3}$ and consider a system of generators:
                $$a^{\pm}_{i, j}[r] \quad, \quad 1 \leq i, j \leq N, r \in \Z_{\geq 0}$$
                $$\level, \del$$
            The \textbf{deformed oscillator algebra} associated to $\calR$ at level $\level \in \bbC$ is then the associative algebra:
                $$\calA_{\hbar}(\calR)$$
            generated by the generators above, equipped with representations $\pi_z^{\pm}: \calA_{\hbar}(\calR) \to \End(\bbV_N)[\![z^{\pm 1}]\!]$:
        \end{definition}
        \begin{convention}
            Suppose that $\g$ is a symmetrisable Kac-Moody associated to an indecomposable Cartan matrix and that $\calR$ is the universal R-matrix of the associated Etingof-Kazhdan quantisation $\calU_{\hbar}(\g)$ as in \cite{etingof_kazhdan_quantisation_6}. For this case, we shall denote the deformed oscillator algebra at level $\level \in \bbC$ by:
                $$\calA_{\hbar}^{\calR}(\g)_{\level}$$
        \end{convention}
        \begin{lemma} \label{lemma: ding_frenkel_isomorphism_oscillator_algebras}
            There is an isomorphism between the oscillator algebras from definitions \ref{def: q_deformed_affine_sl_2_oscillator_algebra_drinfeld_current_presentation} and \ref{def: deformed_oscillator_algebras_R_matrix_presentation}:
                $$\DF: \calA_{\hbar}^{\calR}(\hat{\sl}_2)_{\level} \xrightarrow[]{\cong} \calA_{\hbar}^{\current}(\hat{\sl}_2)_{\level}$$
            given by the same formulae that define the Ding-Frenkel isomorphism from lemma \ref{lemma: ding_frenkel_isomorphism_quantum_affine_sl_2}.
        \end{lemma}
            \begin{proof}
                
            \end{proof}
        \begin{proposition}[Free boson realisation of $\calU_{\hbar}^{\calR}(\hat{\sl}_2)$] \label{prop: free_boson_realisation_quantum_affine_sl_2_R_matrix_presentation}
            There egammasts a free boson realisation in the form of an injective algebra homomorphism:
                $$\freeboson_{\hbar, \level}^{\calR}: \calU_{\hbar}^{\calR}(\hat{\sl}_2)_{\level} \to \calA_{\hbar}^{\calR}(\hat{\sl}_2)_{\level}$$
            given by:
                $$\freeboson_{\hbar, \level}^{\calR}( L^{\pm}(z) ) := $$
            Moreover, the homomorphism $\freeboson_{\hbar, \level}^{\calR}$ as above fits into the following commutative diagram:
                $$
                    \begin{tikzcd}
                	{\calU_{\hbar}^{\calR}(\hat{\sl}_2)_{\level}} & {\calA_{\hbar}^{\calR}(\hat{\sl}_2)_{\level}} \\
                	{\calU_{\hbar}^{\current}(\hat{\sl}_2)_{\level}} & {\calA_{\hbar}^{\current}(\hat{\sl}_2)_{\level}}
                	\arrow["{\freeboson_{\hbar, \level}^{\calR}}", from=1-1, to=1-2]
                	\arrow["\DF"', from=1-1, to=2-1]
                	\arrow["\DF", from=1-2, to=2-2]
                	\arrow["{\freeboson_{\hbar, \level}^{\current}}", from=2-1, to=2-2]
                    \end{tikzcd}
                $$
        \end{proposition}
            \begin{proof}
                
            \end{proof}

    \subsection{\texorpdfstring{Degenerating to a free boson realisation for $\DY_{\hbar}(\sl_2)$}{}}