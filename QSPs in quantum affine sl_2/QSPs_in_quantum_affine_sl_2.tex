\section{\texorpdfstring{Free boson realisation for QSPs in trigonometric and rational quantum affine $\sl_2$ algebras}{}}
    \subsection{\texorpdfstring{Trigonometric and rational QSPs}{}}
        We begin by recalling the notion of reflection equations and coideal subalgebras of quasi-triangular complete topological Hopf algebra defined by those equations, so that we can explain in details the problem that we are trying to solve. All Lie algebras, Lie bialgebras, bialgebras, Hopf algebras, etc., if equipped with a topology, are assumed to be complete unless stated otherwise. Also, for our purposes, it suffices to work with (unitary) solutions $\calR(z, w)$ to the QYBE (in $\bbV_N^{\tensor 3}$) that depend only on the ratio $\frac{z}{w}$, i.e. $\calR(z, w) = \calR\left(\frac{z}{w}\right)$ (as all of the examples that we have in mind satisfy this condition), though the construction below, in principle, can be carried out with solutions that depend more generally on the spectral parameters $z, w$.
        
        The \textbf{reflection equations} mentioned above are equations in:
            $$\End( \bbV_N \tensor \bbV_N ) \tensor \calU_{\hbar}(\calR)[\![z, w]\!] \quad \text{or} \quad \End( \bbV_N \tensor \bbV_N ) \tensor \calU_{\hbar}(\calR)[\![z^{-1}, w^{-1}]\!]$$
        with solutions:
            \begin{equation} \label{equation: type_A_multiplicative_boundary_monodromy_matrices}
                B(z) \in \End(\bbV_N) \tensor \calU_{\hbar}(\calR)[\![z]\!] \quad \text{or} \quad B(z) \in \End(\bbV_N) \tensor \calU_{\hbar}(\calR)[\![z^{-1}]\!]
            \end{equation}
        respectively. These equations come in many flavours, with two amongst them being of particular interest to us, the so-called \textbf{untransposed and transposed reflection equations}, which read:
            \begin{equation} \label{equation: untransposed_reflection_equation_multiplicative}
                \calR\left(\frac{z}{w}\right)_{2, 1} B(z)_1 \calR(zw)_{1, 2} B(w)_2 = B(w)_2 \calR(zw)_{2, 1} B(z)_1 \calR\left(\frac{z}{w}\right)_{1, 2}
            \end{equation}
            \begin{equation} \label{equation: transposed_reflection_equation_multiplicative}
                \calR\left(\frac{z}{w}\right)_{1, 2} B(z)_1 \calR\left(\frac{1}{zw}\right)_{1, 2}^t B(w)_2 = B(w)_2 \calR\left(\frac{1}{zw}\right)_{1, 2}^t B(z)_1 \calR\left(\frac{z}{w}\right)_{1, 2}
            \end{equation}
        and take place in $\End( \bbV_N \tensor \bbV_N ) \tensor \calU_{\hbar}(\calR)[\![z, w]\!]$ and $\End( \bbV_N \tensor \bbV_N ) \tensor \calU_{\hbar}(\calR)[\![z^{-1}, w^{-1}]\!]$ respectively. Notice also, that equations \eqref{equation: untransposed_reflection_equation_multiplicative} and \eqref{equation: transposed_reflection_equation_multiplicative} are nothing but the images of bQYBEs of the same form under the induced map:
            $$
                \begin{aligned}
                    \pi_z^{\pm} \tensor \pi_w^{\pm} \tensor \id_{\calU_{\hbar}(\calR)[\![z^{\pm 1}, w^{\pm 1}]\!]} & : \calU_{\hbar}(\calR)^{\tensor 3}[\![z^{\pm 1}, w^{\pm 1}]\!]
                    \\
                    & \to \End( \bbV_N \tensor \bbV_N ) \tensor \calU_{\hbar}(\calR)[\![z^{\pm 1}, w^{\pm 1}]\!]
                \end{aligned}
            $$
        In particular, the image of solutions $\calK(z)$ to those bQYBEs get mapped to $B(z)$. In analogy with the series $L^{\pm}(z)$, we refer to solutions $B(z)$ as in \eqref{equation: type_A_multiplicative_boundary_monodromy_matrices} as \textbf{boundary monodromy matrices}.
        \begin{remark}
            As a side remark, we note that since universal K-matrices are required to satisfy unitarity, that is:
                $$\calK(z) \calK(z^{-1}) = 1$$
            one has also that:
                \begin{equation} \label{equation: type_A_multiplicative_boundary_monodromy_matrices_unitarity}
                    B(z) B(z^{-1}) = 1
                \end{equation}
            which is also referred to as \textbf{unitarity}.
        \end{remark}
        
        Next, by fixing a solution \eqref{equation: type_A_multiplicative_boundary_monodromy_matrices} to a reflection equation (or equivalently, fixing a universal K-matrix $\calK(z)$), one can define a subalgebra of $\calU_{\hbar}(\calR)$ generated by the coefficients of the matrix entries of the series $B(z)$, typically called a \textbf{reflection algebra} and denoted by:
            $$\calB_{\hbar}(\calK)$$
        \textit{A priori}, there are two possibilities:
        \begin{enumerate}
            \item In the first case, we have that:
                $$B(z) \in \End(\bbV_N) \tensor \calU_{\hbar}^-(\calR)[\![z^{-1}]\!] \quad \text{or} \quad B(z) \in \End(\bbV_N) \tensor \calU_{\hbar}^+(\calR)[\![z]\!]$$
            though with the roles of $z^{-1}$ and $z$ being interchangeable, and hence:
                $$\calB_{\hbar}(\calK) \subset \calU_{\hbar}^-(\calR) \quad \text{or} \quad \calB_{\hbar}(\calK) \subset \calU_{\hbar}^+(\calR)$$
            respectively. This occurs, for instance, in the case of the so-called \say{twisted Yangians} (of any classical type), such as in \cite[Chapter 2]{molev_yangians_and_classical_lie_algebras} and \cite{guay_regelskis_twisted_yangians_for_symmetric_pairs_of_types_BCD}, wherein the reflection algebra $\calB_{\hbar}(\calK)$ is a coideal subalgebra of the usual Yangian of Drinfeld. Another example is the so-called \say{$q$-Onsager algebra} (for $q := e^{\hbar}$), which is a subalgebra of the Yangian subalgebra of $\calU_{\hbar}^{\calR}(\Loop \sl_2)$ (or indeed, of $\calU_{\hbar}^{\calR}(\hat{\sl}_2)$). In light of the two examples above, we note that reflection algebras of this first kind can be constructed out of both rational and trigonometric K-matrices.
            \item In the second case, we have:
                $$B(z) \in \End(\bbV_N) \tensor \calU_{\hbar}(\calR)[\![z^{-1}]\!] \quad \text{or} \quad B(z) \in \End(\bbV_N) \tensor \calU_{\hbar}(\calR)[\![z]\!]$$
            with the roles of $z$ and $z^{-1}$ \textit{not} being interchangeable, and hence:
                $$\calB_{\hbar}(\calK) \subset \calU_{\hbar}(\calR)$$
            in both case, while not belonging properly to either of subalgebras $\calU_{\hbar}^{\pm}(\calR)$. Reflection algebras of this kind have been known to exist when $\calR$ is trigonometric and of type $\sfA$; a particular class of examples are the so-called \say{twisted $q$-Yangians} of types $\sfA.1$ and $\sfA.2$ of Molev-Ragoucy-Sorba from \cite{molev_ragoucy_sorba_twisted_q_yangians_type_A}, for which we have\footnote{The K-matrix is actually constant in both of these cases, but that is not very important for the present discussion.}:
                $$B(z) = L^-(z) \calK(z) L^+(z^{-1})^t \in \End(\bbV_N) \tensor \calU_{\hbar}(\calR)[\![z^{-1}]\!]$$
            Note that both the generating matrices $L^-(z), L^+(z^{-1})$ appear in the expression above for $B(z)$, but both depend on $z^{-1}$, $B(z)$ ultimately depends only on $z^{-1}$.
        \end{enumerate}
        For a lack of better terminologies, let us refer to the reflection algebras (as well as their defining universal K-matrices) in the two cases above as being of \textbf{type I} and \textbf{type II}, respectively. We would also like to be make it clear, that despite what the examples above may have suggested, whether the reflection algebra $\calB_{\hbar}(\calK)$ is of the first or second kind actually my not depend only on the type of the solution $\calR$ to the QYBE (and hence the type of the solution $\calK$ to the bQYBE). Rather, this is controlled by an automorphism on the classical limit of $\calU_{\hbar}(\calR)$. More curiously still, it is known that reflection algebras of type II can \say{degenerate} to reflection algebras of type I: for example, Conner and Guay demonstrated that by specialising $q \to 1$ in a sense, the twisted $q$-Yangians of types $\sfA.1$ and $\sfA.2$ from \cite{molev_ragoucy_sorba_twisted_q_yangians_type_A} (which are of type II) degenerates to the usual rational twisted Yangians of types $\sfA.1$ and $\sfA.2$ as in \cite[Chapter 2]{molev_yangians_and_classical_lie_algebras} (which are of type I).

    \subsection{\texorpdfstring{Free boson realisation for QSPs in $\calU_{\hbar}(\hat{\sl}_2)$}{}}

    \subsection{\texorpdfstring{Degenerating to free boson realisations for QSPs in $\DY_{\hbar}(\sl_2)$}{}}