\section{\texorpdfstring{Free boson realisation for QSPs in $\calU_{\hbar}(\hat{\sl}_2)$}{}}
    \subsection{\texorpdfstring{QSPs in $\calU_{\hbar}(\hat{\sl}_2)$}{}}
        We begin by recalling the notion of reflection equations and coideal subalgebras of quasi-triangular Hopf algebra defined by those equations.
        
        Let $\calU_{\hbar}(\calR)$ be a quasi-triangular Hopf algebra associated to a solution $\calR(z, w)$ of the QYBE; examples include quantum Kac-Moody algebras, their quantum affine and quantum loop subalgebras, (extended) double Yangians, etc; let us fix also a representation on a finite free $\bbC[\![\hbar]\!]$-module $V$ given by:
            $$\pi_z^{\pm}: \calU_{\hbar}(\calR) \to \End_{\bbC[\![\hbar]\!]}(V)[\![z^{\pm 1}]\!]$$
        By the FRT formalism, we know that such algebras are generated by the coefficients of the matrix entries of:
            $$L^+(z) := (\id_{\calU_{\hbar}(\calR)} \tensor \pi_z^{\pm})( \calR^{-1} )_{2, 1} \in \End_{\bbC[\![\hbar]\!]}(V \tensor_{\bbC[\![\hbar]\!]} V) \tensor_{\bbC[\![\hbar]\!]} \calU_{\hbar}(\calR)[\![z]\!]$$
            $$L^-(z) := (\pi_z^{\pm} \tensor \id_{\calU_{\hbar}(\calR)})( \calR )_{1, 2} \in \End_{\bbC[\![\hbar]\!]}(V \tensor_{\bbC[\![\hbar]\!]} V) \tensor_{\bbC[\![\hbar]\!]} \calU_{\hbar}(\calR)[\![z^{-1}]\!]$$
        called \textbf{monodromy matrices}, which are subjected to some RLL-type relations, along with perhaps some commutation relations identifying certain generators as being central or derivations (cf. e.g. definition \ref{def: quantum_affine_sl_2_R_matrix_presentation}). The comultiplication on $\calU_{\hbar}(\calR)$ is known to be such that:
            $$\Delta( L^{\pm}(z) ) = L^{\pm}(z) \tensor L^{\pm}(z)$$
        and thus arise as quantum doubles (such as in the case of quantum Kac-Moody algebras or extended double Yangians with derivations) or quotients thereof (e.g. quantum affine and loop algebras, ordinary double Yangians, etc.), in which the Hopf subalgebras $\calU_{\hbar}^{\pm}(\calR)$ generated by either the series $L^-(z)$ or $L^+(z)$ are isotropic. Due to the resemblance between these Hopf subalgebras of $\calU_{\hbar}(\calR)$ and the Yangian of Drinfeld in its R-matrix presentation (cf. \cite{drinfeld_original_yangian_paper} and \cite[Chapter 1]{molev_yangians_and_classical_lie_algebras}), let us refer to the Hopf subalgebras:
            $$\calU_{\hbar}^-(\calR) \quad \text{and} \quad \calU_{\hbar}^+(\calR)$$
        as the \textbf{Yangian} and \textbf{dual Yangian} subalgebras respectively.
        \begin{example}
            In particular, when $\calR$ is rational and $\calU_{\hbar}(\calR)$ is either the double Yangian or the centrally extended double Yangian (i.e. the algebra generated by the generators of the double Yangian along with an additional central generator), the Hopf subalgebras $\calU_{\hbar}^{\pm}(\calR)$ will be nothing but the usual (dual) Yangians (cf. \cite[Chapter 1]{molev_yangians_and_classical_lie_algebras} and \cite[Chapter 10]{molev_sugawara_operators_for_classical_lie_algebras}).
        \end{example}
        
        Also, for our purposes, it suffices to consider solutions $\calR(z, w) = \calR\left(\frac{z}{w}\right)$ (as all of the examples that we have in mind satisfy this condition), though the construction below, in principle, can be carried out with solutions that depend more generally on the spectral parameters $z, w$.
        
        The \textbf{reflection equations} mentioned at the beginning are equations in:
            $$\End_{\bbC[\![\hbar]\!]}( V \tensor_{\bbC[\![\hbar]\!]} V ) \tensor_{\bbC[\![\hbar]\!]} \calU_{\hbar}(\calR)[\![z, w]\!] \quad \text{or} \quad \End_{\bbC[\![\hbar]\!]}( V \tensor_{\bbC[\![\hbar]\!]} V ) \tensor_{\bbC[\![\hbar]\!]} \calU_{\hbar}(\calR)[\![z^{-1}, w^{-1}]\!]$$
        with solutions:
            \begin{equation} \label{equation: type_A_multiplicative_boundary_monodromy_matrices}
                B(z) \in \End_{\bbC[\![\hbar]\!]}(V) \tensor_{\bbC[\![\hbar]\!]} \calU_{\hbar}(\calR)[\![z]\!] \quad \text{or} \quad B(z) \in \End_{\bbC[\![\hbar]\!]}(V) \tensor_{\bbC[\![\hbar]\!]} \calU_{\hbar}(\calR)[\![z^{-1}]\!]
            \end{equation}
        respectively. These equations come in many flavours, with two amongst them being of particular interest to us, the so-called \textbf{untransposed and transposed reflection equations}, which read:
            \begin{equation} \label{equation: untransposed_reflection_equation_multiplicative}
                \calR\left(\frac{z}{w}\right)_{2, 1} B(z)_1 \calR(zw)_{1, 2} B(w)_2 = B(w)_2 \calR(zw)_{2, 1} B(z)_1 \calR\left(\frac{z}{w}\right)_{1, 2}
            \end{equation}
            \begin{equation} \label{equation: transposed_reflection_equation_multiplicative}
                \calR\left(\frac{z}{w}\right)_{1, 2} B(z)_1 \calR\left(\frac{1}{zw}\right)_{1, 2}^t B(w)_2 = B(w)_2 \calR\left(\frac{1}{zw}\right)_{1, 2}^t B(z)_1 \calR\left(\frac{z}{w}\right)_{1, 2}
            \end{equation}
        and take place in $\End_{\bbC[\![\hbar]\!]}( V \tensor_{\bbC[\![\hbar]\!]} V ) \tensor_{\bbC[\![\hbar]\!]} \calU_{\hbar}(\calR)[\![z, w]\!]$ and $\End_{\bbC[\![\hbar]\!]}( V \tensor_{\bbC[\![\hbar]\!]} V ) \tensor_{\bbC[\![\hbar]\!]} \calU_{\hbar}(\calR)[\![z^{-1}, w^{-1}]\!]$ respectively. Notice also, that equations \eqref{equation: untransposed_reflection_equation_multiplicative} and \eqref{equation: transposed_reflection_equation_multiplicative} are nothing but the images of bQYBEs of the same form under the induced map:
            $$
                \begin{aligned}
                    \pi_z^{\pm} \tensor \pi_w^{\pm} \tensor \id_{\calU_{\hbar}(\calR)[\![z^{\pm 1}, w^{\pm 1}]\!]} & : \calU_{\hbar}(\calR)^{\tensor 3}[\![z^{\pm 1}, w^{\pm 1}]\!]
                    \\
                    & \to \End_{\bbC[\![\hbar]\!]}( V \tensor_{\bbC[\![\hbar]\!]} V ) \tensor_{\bbC[\![\hbar]\!]} \calU_{\hbar}(\calR)[\![z^{\pm 1}, w^{\pm 1}]\!]
                \end{aligned}
            $$
        In particular, the image of solutions $\calK(z)$ to those bQYBEs get mapped to $B(z)$. In analogy with the series $L^{\pm}(z)$, we refer to solutions $B(z)$ as in \eqref{equation: type_A_multiplicative_boundary_monodromy_matrices} as \textbf{boundary monodromy matrices}.
        \begin{remark}
            As a side remark, we note that since universal K-matrices are required to satisfy unitarity, that is:
                $$\calK(z) \calK(z^{-1}) = 1$$
            one has also that:
                \begin{equation} \label{equation: type_A_multiplicative_boundary_monodromy_matrices_unitarity}
                    B(z) B(z^{-1}) = 1
                \end{equation}
            which is also referred to as \textbf{unitarity}.
        \end{remark}
        
        Next, by fixing a solution \eqref{equation: type_A_multiplicative_boundary_monodromy_matrices} to a reflection equation (or equivalently, fixing a universal K-matrix $\calK(z)$), one can define a subalgebra of $\calU_{\hbar}(\calR)$ generated by the coefficients of the matrix entries of the series $B(z)$, typically called a \textbf{reflection algebra} and denoted by:
            $$\calB_{\hbar}(\calK)$$
        \textit{A priori}, there are two possibilities:
        \begin{enumerate}
            \item In the first case, we have that:
                $$B(z) \in \calU_{\hbar}^-(\calR)[\![z^{-1}]\!] \quad \text{or} \quad B(z) \in \calU_{\hbar}^+(\calR)[\![z]\!]$$
            though with the roles of $z^{-1}$ and $z$ being interchangeable, and hence:
                $$\calB_{\hbar}(\calK) \subset \calU_{\hbar}^-(\calR) \quad \text{or} \quad \calB_{\hbar}(\calK) \subset \calU_{\hbar}^+(\calR)$$
            respectively. This occurs, for instance, in the case of the so-called \say{twisted Yangians} (of any classical type), such as in \cite[Chapter 2]{molev_yangians_and_classical_lie_algebras} and \cite{guay_regelskis_twisted_yangians_for_symmetric_pairs_of_types_BCD}, wherein the reflection algebra $\calB_{\hbar}(\calK)$ is a coideal subalgebra of the usual Yangian of Drinfeld. Another example is the so-called \say{$q$-Onsager algebra} (for $q := e^{\hbar}$), which is a subalgebra of the Yangian subalgebra of $\calU_{\hbar}^{\calR}(\Loop \sl_2)$ (or indeed, of $\calU_{\hbar}^{\calR}(\hat{\sl}_2)$). In light of the two examples above, we note that reflection algebras of this first kind can be constructed out of both rational and trigonometric K-matrices.
            \item In the second case, we have:
                $$B(z) \in \calU_{\hbar}(\calR)[\![z^{-1}]\!] \quad \text{or} \quad B(z) \in \calU_{\hbar}(\calR)[\![z]\!]$$
            with the roles of $z$ and $z^{-1}$ \textit{not} being interchangeable, and hence:
                $$\calB_{\hbar}(\calK) \subset \calU_{\hbar}(\calR)$$
            in both case, while not belonging properly to either of the Yangian or dual Yangian subalgebras $\calU_{\hbar}^{\pm}(\calR)$. Reflection algebras of this kind have been known to exist when $\calR$ is trigonometric and of type $\sfA$l; a particular class of examples are the so-called \say{twisted $q$-Yangians} of Molev-Ragoucy-Sorba from \cite{molev_ragoucy_sorba_twisted_q_yangians_type_A}.
        \end{enumerate}
        For a lack of better terminologies, let us refer to the reflection algebras (as well as their defining universal K-matrices) in the two cases above as being of \textbf{type I} and \textbf{type II}, respectively. We would also like to be make it clear, that despite what the examples above may have suggested, whether the reflection algebra $\calB_{\hbar}(\calK)$ is of the first or second kind actually has nothing to do with the type of the solution $\calR$ to the QYBE (and hence the type of the solution $\calK$ to the bQYBE). Rather, this is controlled by an automorphism on the classical limit of $\calU_{\hbar}(\calR)$.
        
        Recall from \cite{etingof_kazhdan_quantisation_1} that, because $\calU_{\hbar}(\calR)$ is assumed to be quasi-triangular, its classical limit is the Lie bialgebra given by:
            $$\g(\calr) := \prim( \calU_{\hbar}(\calR)/\hbar \calU_{\hbar}(\calR) )$$
        on which the Lie cobracket is given by $\delta(X) := [X \tensor 1 + 1 \tensor X, \calr]$, wherein $\calr \equiv \frac{\calR - 1}{\hbar} \pmod{\hbar^2}$ is the classical r-matrix, and one checks that $\delta(X) \equiv \frac{(\Delta - \Delta^{\cop})(\tilde{X})}{\hbar} \pmod{\hbar^2}$ for all lifts $\tilde{X}$ modulo $\hbar$ of $X$. Then, using the classical r-matrix $\calr$, one can construct a Manin triple:
            $$( \g(\calr), \g^+(\calr), \g^-(\calr) )$$
        such that $\g^{\pm}(\calr)$ are classical limits of the algebras $\calU_{\hbar}^{\pm}(\calR)$. In practice, the Lie (bi)algebras $\g(\calr)$ and $\g^{\pm}(\calr)$ are usually graded

    \subsection{\texorpdfstring{Associated deformed $\phi$-coordinated quasi-module}{}}