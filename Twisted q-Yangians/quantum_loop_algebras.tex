\section{Quantum (twisted) loop algebras in the R-matrix presentation}
    \subsection{Trigonometric-type universal R-matrices}
        To begin, let us recall some features of quantum R-matrices with spectral parameters, particularly those arising as Baxterisations of trigonometric-type constant quantum R-matrices. 
    
        Again, let:
            $$\g_N \subset \sl_N$$
        be a finite-dimensional simple Lie algebra of a classical type in the Cartan-Killing classification. Let:
            $$\calU_q(\g_N)$$
        be the\footnote{By a Hochschild-cohomological argument this is known to be unique up to isomorphisms as a Hopf algebra deforming the universal enveloping algebra $\calU(\g_N)$.} quantisation of the standard Lie bialgebra structure on $\g_N$. \textit{A priori}, this quantisation is quasi-triangular, thus possessing a (constant) quantum R-matrix:
            $$\bar{\calR}$$
        that is a \textit{constant} solution to the quantum Yang-Baxter equation (QYBE)
            \begin{equation} \label{equation: constant_QYBE}
                \calR_{1, 2} \calR_{1, 3} \calR_{2, 3} = \calR_{2, 3} \calR_{1, 3} \calR_{1, 2}
            \end{equation}
        Equivalently, by the uniqueness of quantisations of Lie bialgebra structures on finite-type Kac-Moody algebras, this is a solution of trigonometric type in the Belavin-Drinfeld classification. This quantum R-matrix is unique to the Hopf algebra $\calU_q(\g_N)$, for it is the Hochschild cohomological class representing this deformation of $\calU(\g_N)$, so we will instead write:
            $$\calU_q(\bar{\calR})$$
        instead of $\calU_q(\g_N)$ in order to put emphasis on the role of the R-matrix. More generally, if $\calr$ is a solution to the constant classical Yang-Baxter equation (CYBE) (set $u = v = 0$ in equation \eqref{equation: spectral_CYBE} to obtain its constant version) and $\calR \equiv \calr \pmod{\hbar}$ is a solution to the QYBE \eqref{equation: additive_spectral_QYBE} quantising $\calr$, then the corresponding quasi-triangular Hopf algebra obtained via the formalism of Faddeev-Reshetikhin-Takhtajan will be denoted by $\calU_q(\calR)$; this construction is well-known when $\calR$ is a solution to the QYBE with spectral parameter, and for the constant case, see \cite{gautam_rupert_wendlandt_R_matrix_presentation_for_finite_QUEs}).
    
        Next, suppose that $\sfX$ is a connected Dynkin diagram of affine type, of either an untwisted or twisted type, and let:
            $$\calR := \calR(w, z)$$
        be the corresponding universal R-matrix of the quantum Kac-Moody algebra $\calU_q(\sfX)$; technically speaking, this depends on the parameter $q$ as well as the Dynkin diagram $\sfX$ (or equivalently, its associated Kac-Moody algebra), but we omit these from the notation to avoid notational cluttering. We remind the reader that $\calR$ is a solution to the spectral QYBE, best written here in its multiplicative form:
            \begin{equation} \label{equation: multiplicative_spectral_QYBE}
                \calR_{1, 2}\left(\frac{w}{z}\right) \calR_{1, 3}(w) \calR_{2, 3}(z) = \calR_{2, 3}(z) \calR_{1, 3}(w) \calR_{1, 2}\left(\frac{w}{z}\right)
            \end{equation}
        (one obtains the equation above by making the change of variables $w := \exp(u)$ and $z := \exp(v)$ to \eqref{equation: additive_spectral_QYBE}). Moreover, it is a Baxterisation of the finite-type R-matrix $\bar{\calR}$ from before. 

    \subsection{Quantum (twisted) loop algebras of arbitrary classical affine types}
        Following \cite{guay_regelskis_wendlandt_R_matrix_presentation_of_quantum_loop_algebras}, we work with the following RLL-type construction of quantum (twisted) loop algebras. The defining relations in the following R-matrix presentation of (extended) quantum loop algebras are natural from the point of view of the FRT formalism, and let us also mention - for later reference - that it is useful to think of the (extended) quantum loop algebras defined below as trigonometric analogues of (centrally extended) double Yangians, a kind of quantum double for the usual Yangian. This is why the monodromy matrices are \say{doubled}, and later on, we will see that there is a natural trigonometric analogue of the Yangian (the so-called \say{$q$-Yangian}) that is realisable as the subalgebra of a quantum loop algebra generated by the non-negative Fourier modes of the aforementioned monodromy matrices. 
        \begin{definition}[Extended quantum loop algebras] \label{def: extended_quantum_loop_algebras_R_matrix_presentation}
            The \textbf{extended quantum loop algebra} associated to $\calR$ is the associative algebra:
                $$\calU_q^{\ext}(\Loop^{\sigma}\g_N, \calR)$$
            generated by the coefficients of the matrix entries of the elements:
                $$L^{\pm}(w) \in \Mat_N( \calU_q^{\ext}(\Loop^{\sigma}\g_N, \calR)[\![w^{-1}]\!] )$$
            which are subjected to the following relations:
                \todo[inline]{Upper/lower triangularity}
                $$L_{i, j}^-[0] = L_{j, i}^+[0] = 0, \quad 1 \leq i < j \leq N$$
                $$L_{i, i}^-[0] L_{i, i}^+[0] = L_{i, i}^+[0] L_{i, i}^-[0] = 1, \quad 1 \leq i \leq N$$
                $$
                    \begin{cases}
                        \calR\left(\frac{w}{z}\right) L_1^{\pm}(w) L_2^{\pm}(z) = L_2^{\pm}(z) L_1^{\pm}(w) \calR\left(\frac{w}{z}\right)
                        \\
                        \calR\left(\frac{w}{z}\right) L_1^+(z) L_2^-(w) = L_2^-(z) L_1^+(w) \calR\left(\frac{w}{z}\right)
                    \end{cases}
                $$
            As for the matrices $L^{\pm}(w)$, they shall be referred to as the \textbf{positive/negative monodromy matrices} respectively. 
        \end{definition}
        \begin{convention}
            Since we are letting $\calR$ be fixed once and for all as the universal R-matrix of $\calU_q(\sfX)$, we shall henceforth write only:
                $$\calU_q^{\ext}(\Loop^{\sigma}\g_N)$$
            instead of $\calU_q^{\ext}(\Loop^{\sigma}\g_N, \calR)$. Nevertheless, one must keep in mind the dependence of the construction of the extended quantum loop algebra - as well as anything that subsequently depends on it - on this particular choice of $\calR$.
        \end{convention}

        \begin{lemma}[Centres of extended quantum loop algebras]
            
        \end{lemma}
            \begin{proof}
                
            \end{proof}
        \begin{definition}[Quantum loop algebras] \label{def: quantum_loop_algebras_R_matrix_presentation}
            The \textbf{quantum loop algebra} associated to $\calR$ is the associative:
                $$\calU_q(\Loop^{\sigma}\g_N) := \calU_q^{\ext}(\Loop^{\sigma}\g_N)/\calZ_q(\Loop^{\sigma}\g_N)$$
        \end{definition}

        \begin{lemma}[Hopf structures on extended quantum loop algebras] \label{lemma: hopf_structure_on_extended_quantum_loop_algebras}
            
        \end{lemma}
            \begin{proof}
                
            \end{proof}
        \begin{corollary}[Hopf structures on quantum loop algebras] \label{coro: hopf_structure_on_quantum_loop_algebras}
            The centre $\calZ_q(\Loop^{\sigma}\g_N) \subset \calU_q^{\ext}(\Loop^{\sigma}\g_N)$ has the structure of a bi-ideal. Consequently, $\calU_q(\Loop^{\sigma}\g_N)$ has a natural Hopf algebra structure inherited from $\calU_q^{\ext}(\Loop^{\sigma}\g_N)$.
        \end{corollary}
            \begin{proof}
                
            \end{proof}
        \begin{example}[(Extended) $q$-Heisenberg algebras] \label{example: (extended)_q_heisenberg_algebras_R_matrix_presentation}
            When the underlying classical Lie algebra is $\gl_1$, the constructions from definitions \ref{def: extended_quantum_loop_algebras_R_matrix_presentation} and \ref{def: quantum_loop_algebras_R_matrix_presentation} specialises to the notion of \textbf{(extended) $q$-Heisenberg algebras}. In this case, the only possible diagram automorphism $\sigma$ is the identity, so we need not specify it, and we shall write:
                $$\calH_q$$
            
            The monodromy matrices $L^{\pm}(w)$ are $1 \x 1$, i.e. 
        \end{example}

        \begin{definition}[$q$-vacuum modules] \label{def: q_vacuum_modules}
            
        \end{definition}
        \begin{example}[$q$-Fock spaces] \label{example: q_fock_spaces}
            
        \end{example}
        \begin{lemma}[$q$-Wakimoto modules] \label{lemma: q_wakimoto_modules}
            
        \end{lemma}
            \begin{proof}
                
            \end{proof}
        \begin{proposition}[PBW bases for quantum loop algebras] \label{prop: PBW_bases_for_quantum_loop_algebras}
            
        \end{proposition}
            \begin{proof}
                
            \end{proof}

        \todo[inline]{Equivalence with the Drinfeld-Jimbo presentation was worked out Ding-Frenkel in type $\sfA_{\ell}^{(1)}$ and then by Jing-Liu-Molev for the $\sfB_{\ell}^{(1)}, \sfC_{\ell}^{(1)}, \sfD_{\ell}^{(1)}$ cases, and then for the $\sfA_{2\ell - 1}^{(2)}$ case by Jing-Zhang-Liu; there remains the $\sfA_{2\ell}^{(2)}$ and $\sfD_{\ell}^{(2)}$ and $\sfD_4^{(3)}$ cases. Regardless, the technique here is}