\section{Quantum (twisted) loop algebras in the R-matrix presentation}
    \subsection{Trigonometric-type universal R-matrices}
        To begin, let us recall some features of quantum R-matrices with spectral parameters, particularly those arising as Baxterisations of trigonometric-type constant quantum R-matrices. 
    
        Again, let:
            $$\g_N \subset \sl_N$$
        be a finite-dimensional simple Lie algebra of a classical type in the Cartan-Killing classification. Let:
            $$\calU_q(\g_N)$$
        be the\footnote{By a Hochschild-cohomological argument this is known to be unique up to isomorphisms as a Hopf algebra deforming the universal enveloping algebra $\calU(\g_N)$.} quantisation of the standard Lie bialgebra structure on $\g_N$. \textit{A priori}, this quantisation is quasi-triangular, thus possessing a (constant) quantum R-matrix:
            $$\bar{\calR}$$
        that is a \textit{constant} solution to the quantum Yang-Baxter equation (QYBE)
            \begin{equation} \label{equation: constant_QYBE}
                \calR_{1, 2} \calR_{1, 3} \calR_{2, 3} = \calR_{2, 3} \calR_{1, 3} \calR_{1, 2}
            \end{equation}
        Equivalently, by the uniqueness of quantisations of Lie bialgebra structures on finite-type Kac-Moody algebras, this is a solution of trigonometric type in the Belavin-Drinfeld classification. This quantum R-matrix is unique to the Hopf algebra $\calU_q(\g_N)$, for it is the Hochschild cohomological class representing this deformation of $\calU(\g_N)$, so we will instead write:
            $$\calU_q(\bar{\calR})$$
        instead of $\calU_q(\g_N)$ in order to put emphasis on the role of the R-matrix. More generally, if $\calr$ is a solution to the constant classical Yang-Baxter equation (CYBE) (set $u = v = 0$ in equation \eqref{equation: spectral_CYBE} to obtain its constant version) and $\calR \equiv \calr \pmod{\hbar}$ is a solution to the QYBE \eqref{equation: additive_spectral_QYBE} quantising $\calr$, then the corresponding quasi-triangular Hopf algebra obtained via the formalism of Faddeev-Reshetikhin-Takhtajan will be denoted by $\calU_q(\calR)$; this construction is well-known when $\calR$ is a solution to the QYBE with spectral parameter, and for the constant case, see \cite{gautam_rupert_wendlandt_R_matrix_presentation_for_finite_QUEs}).
    
        Next, suppose that $\sfX$ is a connected Dynkin diagram of affine type, of either an untwisted or twisted type, and let:
            $$\calR := \calR(w, z)$$
        be the corresponding universal R-matrix of the quantum Kac-Moody algebra $\calU_q(\sfX)$; technically speaking, this depends on the parameter $q$ as well as the Dynkin diagram $\sfX$ (or equivalently, its associated Kac-Moody algebra), but we omit these from the notation to avoid notational cluttering. We remind the reader that $\calR$ is a solution to the spectral QYBE, best written here in its multiplicative form:
            \begin{equation} \label{equation: multiplicative_spectral_QYBE}
                \calR_{1, 2}(w, z) \calR_{1, 3}(w) \calR_{2, 3}(z) = \calR_{2, 3}(z) \calR_{1, 3}(w) \calR_{1, 2}(w, z)
            \end{equation}
        (one obtains the equation above by making the change of variables $w := \exp(u)$ and $z := \exp(v)$ to \eqref{equation: additive_spectral_QYBE}). Moreover, it is a Baxterisation of the finite-type R-matrix $\bar{\calR}$ from before. 

    \subsection{Quantum (twisted) loop algebras associated to R-matrices of arbitrary classical affine types}
        Following \cite{guay_regelskis_wendlandt_R_matrix_presentation_of_loop_QUEs}, we begin with quantum (twisted) loop algebras given by RLL-type generators and relations. The defining relations in the following R-matrix presentation of (extended) quantum loop algebras are natural from the point of view of the FRT formalism, and let us also mention - for later reference - that it is useful to think of the (extended) quantum loop algebras defined below as trigonometric analogues of (centrally extended) double Yangians, a kind of quantum double for the usual Yangian. This is why the monodromy matrices are \say{doubled}, and later on, we will see that there is a natural trigonometric analogue of the Yangian (the so-called \say{$q$-Yangian}) that is realisable as the subalgebra of a quantum loop algebra generated by the non-negative Fourier modes of the aforementioned monodromy matrices.

        Because we will be employing quantum vertex algebras, it is more convenient to start with the formal versions of the (extended) quantum loop algebras, and then specialise the deformation parameter $q$ later on when necessary. Moreover, it will be convenient to define these algebras relative to a level parameter $\level \in \bbC$. Definition \ref{def: extended_loop_QUEs_R_matrix_presentation} is, strictly speaking, a slight generalisation of the notion of the same name from \cite{guay_regelskis_wendlandt_R_matrix_presentation_of_loop_QUEs}; there, the authors were concerned only with the level $\level = 0$ case. Working at an arbitrary level $\level \in \bbC$ introduces an extra dependence on $q^{\pm \level}$ of the quantum R-matrix in relation \eqref{equation: extended_loop_QUEs_mixed_relation}, which disappears as $\level \to 0$ (see \cite[Remark 3.6]{guay_regelskis_wendlandt_R_matrix_presentation_of_loop_QUEs}). 

        For what follows, let $\hbar$ be a formal variable, and then let:
            $$q := \exp \hbar$$
        At certain points, we will specialise $\hbar$ to some complex number, in which case it will take values in $\bbC \setminus 2\pi \sqrt{-1} \Q$ (i.e. we work only with generic values of $q$).
        \begin{definition}[Extended quantum loop algebras] \label{def: extended_loop_QUEs_R_matrix_presentation}
            The \textbf{extended formal quantum loop algebra} associated to $\calR$ at \textbf{level $\level \in \bbC$} is the associative algebra over $\bbC[\![\hbar]\!]$:
                $$\calU_q^{\ext}(\Loop^{\sigma}\g_N, \calR)_{\level}$$
            generated by the coefficients of the matrix entries of the elements:
                $$L^{\pm}(w) \in \Mat_N( \calU_q^{\ext}(\Loop^{\sigma}\g_N, \calR)[\![w^{-1}]\!] )$$
            which are subjected to the following relations:
                $$
                    \begin{cases}
                        L_{i, j}^-[0] = L_{j, i}^+[0] = 0
                        \\
                        L_{i, i}^+[0] L_{i, i}^-[0] = L_{i, i}^-[0], L_{i, i}^+[0]
                    \end{cases}, \quad 1 \leq i < j \leq N
                $$
                \begin{equation} \label{equation: extended_loop_QUE_RLL_relation}
                    \calR(w, z) L_1^{\pm}(w) L_2^{\pm}(z) = L_2^{\pm}(w) L_1^{\pm}(z) \calR(w, z)
                \end{equation}
                \begin{equation} \label{equation: extended_loop_QUEs_mixed_relation}
                    \calR(q^{\level}w, z) L_1^+(w) L_2^-(z) = L_2^-(w) L_1^+(z) \calR(q^{-\level}w, z)
                \end{equation}
            As for the matrices $L^{\pm}(w)$, they shall be referred to as the \textbf{upper/lower monodromy matrices} respectively. 
        \end{definition}
        \begin{definition}[Quantum loop algebras] \label{def: loop_QUEs_R_matrix_presentation}
            
        \end{definition}

        \begin{lemma}[Centres of extended quantum loop algebras] \label{lemma: centres_of_loop_QUEs}
            
        \end{lemma}
            \begin{proof}
                
            \end{proof}

        \begin{lemma}[Hopf structures on extended quantum loop algebras] \label{lemma: hopf_structure_on_extended_loop_QUEs}
            
        \end{lemma}
            \begin{proof}
                
            \end{proof}
        \begin{corollary}[Hopf structures on quantum loop algebras] \label{coro: hopf_structure_on_loop_QUEs}
            
        \end{corollary}
            \begin{proof}
                
            \end{proof}

    \subsection{Quantum Wakimoto modules}
        Our next objective is to give a free field realisation of the quantum loop algebra. The first step is to show that the original R-matrix presentation from definition \ref{def: loop_QUEs_R_matrix_presentation} is equivalent to Drinfeld's current presentation, with the purpose being that afterwards, we would like to package the Drinfeld generators into generating series, which is necessary for free field realisations. The strategy after this point shall be to somehow $q$-deform the Fock space associated to the root lattice of type $\bar{\sfX}$. 
        
        \begin{lemma}[Drinfeld's current presentation for quantum loop algebras] \label{lemma: drinfeld_current_presentation_for_loop_QUEs}
            \todo[inline]{Equivalence between Drinfeld's current presentation and R-matrix presentation}
        \end{lemma}
            \begin{proof}
                
            \end{proof}
        \begin{corollary}[Triangular decomposition] \label{coro: triangular_decomposition_for_loop_QUEs}
            
        \end{corollary}
            \begin{proof}
                
            \end{proof}
        \begin{corollary}[Kac-Moody coproduct fomrulae for Drinfeld generators] \label{coro: coproduct_formulae_for_drinfeld_generators}
            
        \end{corollary}
            \begin{proof}
                
            \end{proof}
        \begin{remark}
            The formulae from corollary \ref{coro: coproduct_formulae_for_drinfeld_generators} should not be confused with those of the Drinfeld coproduct, which gives rise to an alternate Hopf algebra structure on quantum loop algebras. This is conjecturally related to the standard Kac-Moody coproduct by a Drinfeld twisting process, but we will not touch upon this point. We make this comment instead to put emphasis on the fact that the coideal subalgebras arising from the K-matrices of Appel-Regelskis-Vlaar are with respect to the Kac-Moody coproduct. 
        \end{remark}
            
        \begin{definition}[The category O for quantum loop algebras] \label{def: category_O_loop_QUEs}
            
        \end{definition}
        \begin{lemma}[Tensor structures and semi-simplicity of the category O] \label{lemma: tensor_structure_and_semi_simplicity_of_category_O}
            \begin{enumerate}
                \item The weight module category $\calU_q(\Loop^{\sigma}\g_N, \calR)\mod^{\rootlattice}$ is a tensor category with respect to $\tensor := \tensor_{\bbC}$.
                \item The category $\calO( \calU_q(\Loop^{\sigma}\g_N, \calR) )$ is meromorphically braided via the universal R-matrix $\calR(w, z)$.
                \item The category $\calO^{\integrable}( \calU_q(\Loop^{\sigma}\g_N, \calR) )$ a semi-simple meromorphically braided subcategory of $\calO( \calU_q(\Loop^{\sigma}\g_N, \calR) )$.
            \end{enumerate}
        \end{lemma}
            \begin{proof}
                
            \end{proof}

        \begin{proposition}[Quantum bosonic Fock spaces for quantum loop algebras] \label{prop: quantum_bosonic_fock_spaces_for_loop_QUEs}
            \todo[inline]{
                The idea here is conceptually simple: just like how one realises Lie algebra actions in terms of vector fields, and then lift to actions of universal enveloping algebras, actions of quantum groups ought to be realised on algebras of "quantum" differential operators; to be honest, I'm a bit surprised this has not been attempted before (or at least, I have not seen it attempted).
                
                What I am proposing is the following. Suppose that $(\C, \tensor, \1, R)$ is a braided tensor category (e.g. the category O of $\calU_q(\Loop^{\sigma}\g_N, \calR)$). Then, an $R$-twisted derivation on an $R$-braided-commutative algebra object $A$ (with multiplication $\mu_A: A \tensor A \to A$) with values in a left-$A$-module $\rho_A: A \tensor M \to M$ will be a morphism $d: A \to M$ satisfying the $R$-braided Leibniz rule $d \circ \mu_A = \rho_A \circ R_{A, M}^{-1} \circ (d \tensor \id_A) + \rho_A \circ (\id_A \tensor d)$. We can then define $R$-braided differential operators in the style of Grothendieck.

                Now, for $\calU_q(\Loop^{\sigma}\g_N, \calR)$, the R-matrix appears in the braided-cocommutativity of the coproduct, not of the braided-commutativity of the product. However, we can consider the dual Hopf algebra, say $\calA(\calR)$ for now, thereof, which is $R$-braided-commutative by abstract nonsense, and then try to realise this algebra in terms of $R$-braided differential operators instead. 
            }
        \end{proposition}
            \begin{proof}
                
            \end{proof}

    \subsection{Quantum vertex algebras associated to trigonometric R-matrices}
        If $\scrV$ is a $\bbC[\![\hbar]\!]$-module then the submodule consisting of topologically nilpotent elements (i.e., elements on which the pseudo-uniformiser $\hbar$ of the underlying adic ring acts topologically nilpotently) will be denoted by $\scrV^{\circ}$. 
    
        We begin by recalling the definition of $\hbar$-adic quantum vertex algebras in the sense of \cite{etingof_kazhdan_quantisation_5}. 

        \begin{definition}[Formal quantum vertex algebras (qVAs)] \label{def: formal_quantum_vertex_algebras}
            Consider a topologically free $\bbC[\![\hbar]\!]$-module:
                $$\scrV$$
            with a distinguished element:
                $$\ket{0} \in \scrV$$
            identified (this is the so-called \textbf{vacuum vector}), and a \textbf{state-field correspondence}:
            \begin{equation} \label{equation: state_field_correspondence}
                Y(-, z): \scrV \hattensor_{\bbC[\![\hbar]\!]} \scrV \to \scrV^{\circ}(\!(z)\!)
            \end{equation}
            which is just a continuous $\bbC[\![\hbar]\!]$-linear map for now.
            \begin{enumerate}
                \item If the state-field correspondence \eqref{equation: state_field_correspondence} satisfies \:
                \begin{itemize}
                    \item the \textbf{vacuum condition}, which reads:
                    \item as well as \textbf{weak associativity}, which reads:
                \end{itemize}
                then we will say that $\scrV$ is a \textbf{non-local vertex algebra}.
                
                \todo[inline]{Non-local vertex algebras}
                
                \item If the weak associativity condition in the previous definition is strengthened to \textbf{braided associativity}\footnote{This is also called \say{S-locality}, but using this terminology will cause clashes later on.}, then we will obtain the notion of \textbf{braided vertex algebras}\footnote{Also called \say{weak quantum vertex algebras.}}. The condition reads:
                
                \todo[inline]{Braided vertex algebras}
                
                \item Finally, if the braiding satisfies the QYBE, then the resulting structure will be referred to as a \textbf{quantum vertex algebra}.
                
                \todo[inline]{Quantum vertex algebras}
                
            \end{enumerate}
        \end{definition}

        \begin{definition}[Vacuum modules] \label{def: loop_QUE_vacuum_modules}
            
        \end{definition}

        \begin{lemma}[qVOA structures on vacuum modules] \label{lemma: qVOA_structures_on_loop_QUE_vacuum_modules}
            
        \end{lemma}
            \begin{proof}
                
            \end{proof}

        \begin{proposition}[$q$-Wakimoto homomorphism] \label{prop: q_wakimoto_homomorphism}
            
        \end{proposition}
            \begin{proof}
                
            \end{proof}

    \subsection{PBW bases in terms of R-matrices}
        \todo[inline]{Equivalence between the R-matrix presentation and the Drinfeld current presentation was worked out Ding-Frenkel in type $\sfA_{\ell}^{(1)}$ and then by Jing-Liu-Molev for the $\sfB_{\ell}^{(1)}, \sfC_{\ell}^{(1)}, \sfD_{\ell}^{(1)}$ cases, and then for the $\sfA_{2\ell - 1}^{(2)}$ case by Jing-Zhang-Liu; there remains the $\sfA_{2\ell}^{(2)}$ and $\sfD_{\ell}^{(2)}$ and $\sfD_4^{(3)}$ cases. Regardless, the technique involves writing down a Gauss decomposition for the monodromy matrix $L(w)$ (this is the key insight of Ding-Frenkel), and then, using the lower/diagonal/upper diagonal portions to build corresponding quantum vertex operators that act as negative root/Cartan/positive root generators on the Fock space. Thus, we are able to recover the Drinfeld current presentation. Then, as a consequence, we obtain a PBW basis in terms of the monodromy matrix. \textbf{Side note: it seems that no PBW theorem purely in terms of the R-matrix presentation is known.}}