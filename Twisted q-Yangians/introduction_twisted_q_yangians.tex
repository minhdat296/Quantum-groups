\section{Introduction}
    \subsection{Notations}
        \subsubsection{Formal power series}
            All power series are understood in the formal sense. Moreover, transcendental expressions shall be used as abbreviations for their corresponding formal Taylor series around $0$; for instance, we have:
                $$e^{a(z)} := \sum_{r \geq 0} \frac{1}{r!} ( a(z) )^r \in \bbC[\![z^{\pm 1}]\!]$$
            for any $a(z) \in A[z^{\pm 1}]$ with coefficients in some associative algebra $A$.

            Given a formal power series $a(z) \in A[\![z^{\pm 1}]\!]$ with coefficients in some associative algebra $A$, we shall write $a[r]$ ($r \in \Z$) for the \textbf{Fourier modes} (i.e. coefficients) thereof, which are such that:
                $$a(z) = \sum_{r \in \Z} a[r] z^{-r - 1}$$
            
            Note also that while $\bbC[\![z]\!]$ and $\bbC(\!(z)\!)$ are commutative algebras (the latter is even a field), $\bbC[\![z^{\pm 1}]\!]$ is not even an associative algebra; the reason is that in general, an expression such as $\left( \sum_{r \in \Z} a[r] z^r \right) \left( \sum_{s \in \Z} b[s] z^s \right)$ may not be well-defined.
    
        \subsubsection{Affine (Kac-Moody) algebras and (twisted) loop algebras} 
            All throughout:
                $$\g_N \subseteq \gl_N$$
            will be used to denote a finite-dimensional simple Lie algebra over $\bbC$ that is of one of the classical types in the Cartan-Killing Classification; equivalently, suppose that:
                $$
                    \g_N =
                    \begin{cases}
                        \sl_N, \quad N \geq 2
                        \\
                        \o_{2n + 1}, \quad N = 2n + 1, \quad n \geq 1
                        \\
                        \sp_{2n}, \quad N = 2n, \quad n \geq 1
                        \\
                        \o_{2n}, \quad N = 2n, \quad n \geq 1
                    \end{cases}
                $$
            and respectively, we say that $\g_N$ is linear, odd-orthogonal, symplectic, or even-orthogonal. This Lie algebra has a natural action on $\bbV_N := \bbC^N$ (the so-called \say{vector representation}), which shall always be equipped with the standard basis:
                $$\{e_i\}_{1 \leq i \leq N}$$
            consisting of vectors with $1$ in the $i^{th}$ entry and $0$ elsewhere. By finite-dimensionality, we identify:
                $$\bbV_N^{\tensor 2} \xrightarrow[]{\cong} \End(\bbV_N)$$
            and then write:
                $$E_{i, j}$$
            for the image of $e_i \tensor e_j$ under the identification above; explicitly, $E_{i, j}$ is the $N \x N$ matrix with $1$ in the $(i, j)^{th}$ entry and $0$ elsewhere. Moreover, recall that $\bbV_N$ comes equipped with a natural $\g_N$-invariant and non-degenerate bilinear form that we shall denote by:
                $$\eta$$
            This form is symmetric for the linear type $\sfA$ and orthogonal types $\sfB, \sfD$, while skew-symmetric for the symplectic type $\sfC$.
        
            \begin{itemize}
                \item Let $\sfX$ be an affine Dynkin diagram - untwisted or twisted - of a classical type (see \cite[Chapter 4, Tables Aff 2 and 3, p. 55]{kac_infinite_dimensional_lie_algebras}), let $\g(\sfX)$ be the associated (affine) Kac-Moody algebra, let $\g_N \subset \sl_N$ be the underlying finite-dimensional simple Lie algebra, and let $\bar{\sfX}$ be the Dynkin diagram associated to $\g_N$.
     
                \item Additionally, let:
                    $$\Loop: \bbC\-\Lie\Alg \to \bbC\-\Lie\Alg$$
                be the looping functor, mapping Lie algebras $\a$ to their \textbf{loop algebras} given by\footnote{More generally, for any commutative algebra $A$, there exists a current algebra functor $- \tensor A: \bbC\-\Lie\Alg \to \bbC\-\Lie\Alg$, with the Lie bracket on $\a \tensor A$ given in the same manner for all Lie algebras $\a$. However, we are only interested in the case $A = \bbC[t^{\pm 1}]$ here.}:
                    $$\Loop \a := \a \tensor_{\bbC} \bbC[t^{\pm 1}]$$
                on which the Lie bracket is given by:
                    $$[x \tensor f, y \tensor g]_{\Loop \a} := [x, y]_{\a} \tensor fg$$
                for all $x, y \in \a$ and all $f, g \in \bbC[t^{\pm 1}]$; for notational simplicity, we will abbreviate $xf := x \tensor f$ for $x \in \a$ and $f \in \bbC[t^{\pm 1}]$.
    
                \item The \textit{untwisted} affine Kac-Moody algebra associated to $\g_N$ is denoted by $\hat{\g}_N$, while the affinisation of $\g_N$ is denoted by $\tilde{\g}_N$. \textit{A priori}, we have $\hat{\g}_N \cong \uce( \Loop \g_N ) \rtimes \bbC D_{\aff}$ and $\tilde{\g}_N \cong \uce( \Loop \g_N ) \cong \Loop \g_N \oplus \bbC c_{\aff} \cong \hat{\g}_N'$. The central generator $c_{\aff}$ is usually referred to as the central charge, while $D_{\aff} \in \der(\tilde{\g}_N)$ is a derivation given by $[D_{\aff}, xf]_{\hat{\g}_N} := x t \frac{df}{dt}$ for all $x \in \g_N$ and $f \in \bbC[t^{\pm 1}]$, and $[D_{\aff}, c_{\aff}] = 0$. For more details, see \cite[Chapter 7]{kac_infinite_dimensional_lie_algebras}.
                
                \item Next, let $\sigma$ be a finite-order automorphism of $\bar{\sfX}$. This gives rise to a Lie algebra automorphism of $\g_N$, which in turn gives rise to an automorphism of the loop algebra $\Loop\g_N$; in fact, this automorphism is involutive \textit{a priori}. Then, write:
                    $$\Loop^{\sigma}\g_N$$
                for the fixed-point subalgebra; this is the Lie algebra that we will be mainly working with. The involution $\sigma$ is sometimes referred to as the \textbf{twisting}.
            \end{itemize}
    
            Let us also set up some conventions around root systems of Kac-Moody algebras.
            \begin{itemize}
                \item Suppose that $\simpleroots$ is some finite set (to be thought of as indexing the simple roots), and then let $\g$ be the Kac-Moody algebra associated to a general indecomposable symmetrisable Cartan matrix $C := (C_{i, j})_{i, j \in \simpleroots}$ with set of Chevalley-Serre generators $\{h_i, e_i^{\pm}\}_{i \in \simpleroots}$ (as in \cite[Chapter 1]{kac_infinite_dimensional_lie_algebras}). Let $\h := \bigoplus_{i \in \simpleroots} \bbC h_i \subset \g$ be the standard Cartan subalgebra\footnote{By \cite{kac_peterson_infinite_flag_varieties_and_conjugacy_of_cartan_subalgebras}, we know that such subalgebras are conjugate to one another, so any choice is equally valid.} with respect to the aforementioned generators.
            
                \item Next, endow $\g$ with the $\Z$-grading\footnote{In \cite{kac_infinite_dimensional_lie_algebras}, this is referred to as the \say{principal grading}.} given by $\deg h_i := 0$ and $\deg e_i^{\pm} := \pm 1$, and then let $(-, -)_{\g}$ be the invariant and symmetric bilinear form on $\g$ of total degree $0$ with respect to said $\Z$-grading, normalised so that $(h_i, h_j) = 2\delta_{i, j}$; additionally, let $\n^{\pm} \subset \g$ be the Lie subalgebras generated by the sets $\{e_i^{\pm}\}_{i \in \simpleroots}$, and let $\b^{\pm} := \h \oplus \n^{\pm}$ be the corresponding Borel subalgebras.
                
                \item Let $\g' := [\g, \g]$ be the derived subalgebra, which \textit{a priori} is isomorphic to the Lie subalgebra of $\g$ generated by the set $\{e_i^-, e_i^+(\}_{i \in \simpleroots}$, and then let $\h' := \h \cap \g'$. 
            \end{itemize}

        \subsubsection{Trigonometric and rational quantum loop algebras}
            Suppose that $\calR(u, v)$ is a solution to the spectral quantum Yang-Baxter equation (QYBE) of type $\sfX$. According to the Belavin-Drinfeld classification, there are three possible types of solutions, namely the trigonometric, rational, and elliptic types. We are interested in the first two types. For more information, see subsection \ref{subsection: trigonometric_and_rational_loop_QUEs}.
            \begin{itemize}
                \item If this solution is of trigonometric type, then the corresponding formal quantum loop algebra over $\bbC[\![\hbar]\!]$ given via the R-matrix presentation will be denoted by:
                    $$\calU_{\hbar}(\calR)$$
                \item If the solution is of rational type, then the corresponding formal quantum group over $\bbC[\![\hbar]\!]$ will be denoted by:
                    $$\DY_{\hbar}(\calR)$$
                This is commonly referred to as the \textbf{(formal) double Yangian} associated to $\calR$.
            \end{itemize}

            When these algebras are given in the (equivalent) Drinfeld current or Drinfeld-Jimbo presentation (also known as the Kac-Moody presentation), we will instead write:
                $$\calU_{\hbar}^{\current}(\Loop^{\sigma}\g_N) \quad, \quad \calU_{\hbar}^{\KM}(\Loop^{\sigma}\g_N)$$
            and:
                $$\DY_{\hbar}^{\current}(\g_N) \quad, \quad \DY_{\hbar}^{\KM}(\g_N)$$
            respectively, if necessary.
            \begin{remark}
                In the rational case, the double Yangian actually does not depend on the twisting $\sigma$, which is why we do not need to specify it.
            \end{remark}

            A potentially confusing notation is:
                $$\DX_{\hbar}(\calR)$$
            which appears, for instance, in definitions \ref{def: extended_loop_QUEs_R_matrix_presentation} and \ref{def: vacuum_modules}. This is used in both the trigonometric and rational contexts, but as it is only an auxiliary algebra, we feel that it is not necessary to specify the type of the R-matrix in general when dealing with this algebra.

        \subsubsection{Quantum Kac-Moody algebras and quantum affine algebras}

        \subsubsection{Opposite (co)algebras}
            If $A$ is an algebra (not even necessarily associative or unital) with multiplication $\mu: A \tensor A \to A$, then its \textbf{opposite algebra} shall be denoted by $A^{\op}$. This object has the same underlying vector space, but the opposite multiplication is now given by $\mu^{\op} := \mu \circ \tau_{1, 2}$, wherein $\tau_{1, 2}: A \tensor A \xrightarrow[]{\cong} A \tensor A$ is the isomorphism given by $\tau_{1, 2}(x \tensor y) := y \tensor x$. Likewise, if $C$ is a coalgebra with comultiplication $\Delta: C \to C \tensor C$, then its \textbf{(co-)opposite} shall be denoted by $C^{\cop}$; the co-opposite multiplication is given by $\Delta^{\cop} := \tau_{1, 2} \circ \Delta$. These notions generalise in a straightforward manner to the setting of topological (co)algebras.

        \subsubsection{Tensor notations}
            \begin{itemize}
                \item Given vector spaces $\{V_i\}_{1 \leq i \leq n}$ and an operator with spectral parameters:
                    $$T(z_1, ..., z_n) \in \End(V_1 \tensor ... \tensor V_n)(\!(z_1^{-1}, ..., z_n^{-1})\!)$$
                along with an auxiliary space $V_0$, then for:
                    $$\{i_0\} := \{0, ..., n\} \setminus \{i_1, ..., i_n\}$$
                we will write:
                    \begin{equation} \label{equation: tensor_factor_notation}
                        T(z_1, ..., z_n)_{i_1, ..., i_n} \in \End\left(V_0 \tensor (V_1 \tensor ... \tensor V_n)\right)(\!(z_1^{-1}, ..., z_n^{-1})\!)
                    \end{equation}
                for the operator with spectral parameter on $\bigotimes_{i = 0}^n V_i$ which acts as $\id_{V_0}$ on the $i_0^{th}$ tensor factor and as $T(z_1, ..., z_n)$ on the remaining ones. In particular, we will usually be concerned with operators:
                    $$R(z_1, z_2) \in \End(V_1 \tensor V_2)(\!(z_1^{-1}, z_2^{-1})\!)$$
                    $$K(z) \in \End(V)(\!(z^{-1})\!)$$
                appearing in say, the quantum Yang-Baxter equation \eqref{equation: additive_spectral_QYBE} or the boundary quantum Yang-Baxter equation \eqref{equation: additive_spectral_bQYBE}.
                \item Given an algebra $A$ and an implicitly given representation $A \to \End(V)$, the operator on $V$ corresponding to an element $x \in A$ will be denoted by:
                    $$x_V$$
            \end{itemize}

    \subsection{Context}
        In statistical mechanics, the R-matrix is a device that encodes nearby interactions of particles, in such a way that intrinsic properties can be kept tracked of. More specifically, if we would like to keep track of $k$ particles, each with $N$ internal degrees of freedom, also known as \say{(quantum) states} (e.g. magnetic charge, spin, etc.), then first of all, the space of states will be $(\bbC^N)^{\tensor k}$; such a system is usually referred to as a \say{spin chain}, and if the $k^{th}$ particle does not interact with the $1^{st}$ one, then the chain will be called \say{open}, while it will be referred to as being \say{closed} if there is interaction between the last and first particle. We are interested in spin chains in which particles only interact with their immediate neighbours, i.e. particle $i$ interacts only with particle $i - 1$ and $i + 1$, with $2 \leq i \leq N - 1$ in the open case and $i \in \Z/k\Z$ in the closed case. In particular, this means that for all intents and purposes, we need to only deal with three-particle systems, or mathematically speaking, $\bbC^N \tensor \bbC^N \tensor \bbC^N$ as the space of states. 

Now, we assume that during interactions, particles keep their momentum while having their internal quantum states modified by said interactions. Since the $N$-tuple of internal quantum states of each particle is to be thought of as a vector in $\bbC^N$, the assumption above can be re-interpreted in more mathematical terms in the following manner. First of all, whatever is controlling these pairwise particle interactions, as it conserves momentum, is a linear operator on $\bbC^N \tensor \bbC^N$ (the operator acts not on $\bbC^N$, for this represents a single particle); we wish to also take into account the fact that the specificities of interactions may vary as the particles are placed at different points of an ambient space, e.g. some algebraic variety:
    $$\Sigma$$
so we shall be concerned with an element:
    $$\calR_{\hbar} \in \End( \bbC^N \tensor \bbC^N ) \tensor \bbC(\Sigma)[\![\hbar]\!]$$
wherein $\bbC(\Sigma)$ is the algebra of globally rational functions on $\Sigma$. We are interested only in the case when $\Sigma$ is a smooth algebraic curve, usually thought of as an embedded plane curve inside either $\A^2$ or $\P^2$; near $(0, 0)$, we can fix a local coordinate $u \in \Sigma$ and obtain a (non-canonical) embedding $\bbC(\Sigma) \subset \bbC(\!(u)\!)$; thus, we are interested in elements:
    $$\calR_{\hbar}(u) \in \End( \bbC^N \tensor \bbC^N ) \tensor \bbC(\!(u)\!)[\![\hbar]\!]$$
\begin{remark}
    When $\Sigma$ is a projective curve, such as $\P^1$, the coordinate $u$ can be written homogeneously as $u = \frac{z}{w}$. 
\end{remark}
\begin{convention}
    When it is not necessary to specify, we will omit the $\hbar$ subscript and only write $\calR$.
\end{convention}

\todo[inline]{Not done}

        In the event that a spin chain exists within a so-called boundary condition, i.e. an external physical presence with which the particles in the spin chain may also interact (e.g. an external magnetic field), then the algebra of symmetries must also be set up in a way that takes into account particle-boundary interactions. Again, we assume that momentum is conserved while the internal quantum states are modified during interactions, so the device controlling these interactions with the boundary is first and foremost an element:
    $$\calK(u) \in \End(\bbC^N)(\!(u^{-1})\!)$$

\todo[inline]{Not done}
        
    \subsection{Overview}
        Again, let $\g_N$ be a classical Lie algebra, and let $\calR := \calR(u, v)$ be a solution to the QYBE \eqref{equation: additive_spectral_QYBE}.
        
        Our goal shall be to prove, that the bottom arrow in the following diagram is an isomorphism that makes the whole diagram commute:
            \begin{equation} \label{diagram: twisted_q_yangians}
                \begin{tikzcd}
            	{\calU_{\hbar}^{\calR}(\tilde{\g}_N)_{\level}} & {\calU_{\hbar}^{\current}(\tilde{\g}_N)_{\level}} & {\calU_{\hbar}^{\KM}(\tilde{\g}_N)_{\level}} \\
            	{\calU_{\hbar}(\calR)_{\level}} & {\calU_{\hbar}^{\current}(\Loop^{\sigma}\g_N)_{\level}} & {\calU_{\hbar}^{\KM}(\Loop^{\sigma}\g_N)_{\level}} \\
            	{\calU^-_{\hbar}(\calR)} && {\calU_{\hbar}^{-, \KM}(\Loop^{\sigma}\g_N)} \\
            	{\calB_{\hbar}(\calK)} && {\calB_{\hbar}^{\KM}(\Loop^{\sigma} \g_N)}
            	\arrow["\cong", from=1-1, to=1-2]
            	\arrow["\cong", from=1-2, to=1-3]
            	\arrow[from=2-1, to=1-1]
            	\arrow["\cong", dashed, from=2-1, to=2-2]
            	\arrow[from=2-2, to=1-2]
            	\arrow["\cong", dashed, from=2-2, to=2-3]
            	\arrow[from=2-3, to=1-3]
            	\arrow[from=3-1, to=2-1]
            	\arrow["\cong", dashed, from=3-1, to=3-3]
            	\arrow[from=3-3, to=2-3]
            	\arrow[from=4-1, to=3-1]
            	\arrow["{\overset{?}{\cong}}", dashed, from=4-1, to=4-3]
            	\arrow[from=4-3, to=3-3]
                \end{tikzcd}
            \end{equation}
        First, let us briefly explain what terms in the diagram represent, apart from the notations that have been fixed.
        
        First of all, $\calU_{\hbar}^-(\calR)$ denotes an abstraction of the Yangian of Drinfeld in the RTT presentation, so we refer to it as the \say{Yangian associated to $\calR$}; namely, it is the subalgebra generated by the coefficients of the matrix entries of:
            $$L^-(u) := \sum_{r \geq 0} L^-[r] u^{-r - 1} \in \Mat_N( \calU_{\hbar}(\calR)_{\level} )[\![u^{-1}]\!]$$
        (cf. definition \ref{def: untwisted_yangians}); due to the form of the defininig relations, we will sometimes refer to this subalgebra as the \say{RLL Yangian}, especially when $\calR$ is of trigonometric type. Secondly, $\calU_{\hbar}^{-, \KM}(\Loop^{\sigma} \g_N)$ denotes the image of the aforementioned Yangian associated to $\calR$ under the composite isomorphism:
            $$\calU_{\hbar}(\calR)_{\level} \xrightarrow[]{\cong} \calU_{\hbar}^{\current}(\Loop^{\sigma} \g_N)_{\level} \xrightarrow[]{\cong} \calU_{\hbar}^{\KM}(\Loop^{\sigma} \g_N)_{\level}$$
        identifying the R-matrix and Kac-Moody presentations quantum loop algebra with one another.
        
        Inside these algebras, there are certain subalgebras $\calB_{\hbar}(\calK)$ and $\calB_{\hbar}^{\KM}(\Loop^{\sigma} \g_N)$, respectively, that are of primary interest to us. The former is an algebra given abstractly by generators that form an $N \x N$ matrix:
            $$B(u)$$
        obeying a (transposed) reflection equation (cf. equation \eqref{equation: reflection_equations}), and hence depending in particular on the K-matrix $\calK(u)$. It was then shown in \cite[Subsection 10.2]{regelskis_vlaar_reflection_matrices_coideal_subalgebras} that the reflection algebra $\calB_{\hbar}(\calK)$ can be embedded as a right-coideal subalgebra into RLL Yangian $\calU_{\hbar}^-(\calR)$, under which embedding the image of $B(u)$ is some $N \x N$ matrix:
            $$S(u)$$
        that depends on $L^{\pm}(u)$ and $\calK(u)$. The latter algebra $\calB_{\hbar}^{\KM}(\Loop^{\sigma} \g_N)$, on the other hand, is generated by certain products of the the Kac-Moody generators of $\calU^{-, \KM}_{\hbar}(\Loop^{\sigma} \g_N)$ (cf. \cite[Definition 5.1]{kolb_kac_moody_QSPs} or \cite[Definition 3.5]{regelskis_vlaar_reflection_matrices_coideal_subalgebras}), and hence is automatically a subalgebra. Kolb showed in \cite[Proposition 5.2]{kolb_kac_moody_QSPs} (see also \cite[Proposition 3.6]{regelskis_vlaar_reflection_matrices_coideal_subalgebras}) that $\calB_{\hbar}^{\KM}(\Loop^{\sigma} \g_N)$ is a right-coideal subalgebra of $\calU^{-, \KM}_{\hbar}(\Loop^{\sigma} \g_N)$ as well.
        
        Later on, in \cite{regelskis_vlaar_reflection_matrices_coideal_subalgebras}, Regelskis and Vlaar suggested a method by which $\calB_{\hbar}^{\KM}(\Loop^{\sigma} \g_N)$ can be identified with so-called \say{pseudo-symmetric pairs} given via automorphisms of a certain sort of the affine Kac-Moody algebra of type $\sfX$. Their main idea was that trigonometric K-matrices can be constructed out of the algebras $\calB_{\hbar}^{\KM}(\Loop^{\sigma} \g_N)$, and then by classifying these K-matrices in terms of some combinatorial data known as \say{generalised Satake diagrams} - which also define the aforementioned pseudo-symmetric pairs - one can identify $\calB_{\hbar}^{\KM}(\Loop^{\sigma} \g_N)$ and $\calB_{\hbar}(\calK)$. However, the classification of K-matrices has only been successfully accomplished using \url{Wolfram Mathematica} when the rank of the root system $\sfX$ is small, namely $| \simpleroots | \leq 15$, while the general case has remained conjectural thus far. This has to do with the fact that in classifying the K-matrices, one must write down explicit expressions thereof, but this task becomes prohibitively complicated once the rank of the root system $\sfX$ is sufficiently large. 

        We therefore propose a more qualitative approach towards the problem of comparing the algebras $\calB_{\hbar}(\calK)$ and $\calB_{\hbar}^{\KM}(\Loop^{\sigma}\g_N)$, based on free field realisations. 