\section{Free field realisations for quantum loop algebras}
    \subsection{Quantum Heisenberg-Weyl algebra}

    \subsection{Quantum affine vertex algebras associated to trigonometric and rational R-matrices}
        Having recalled certain features of the quantum loop algebras associated to a given trigonometric R-matrix, we next recall the construction of a quantum vertex algebra structure on the so-called \say{vacuum module}\footnote{Also known as \say{Weyl modules}, such as in \cite{etingof_kazhdan_quantisation_5}, but we would like to avoid this terminology as it is also used to refer to other constructions.}. This construction also takes as input the aforementioned trigonometric R-matrix, which for us is $\calR$ coming from the quantum Kac-Moody algebra $\calU_{\hbar}(\sfX)$, but is also applicable to the rational and elliptic cases. We are not interested in the last case, but it is important to us that it is possible to endow vacuum modules with structures of quantum vertex algebras in both the rational and trigonometric cases. This is because eventually, we would like it to be so that as $q \to 1$, the (twisted) $q$-Yangians will degenerate back to the usual (twisted) Yangians, and this compatibility must start at the level of quantum vertex algebras (and modules over them; more on this later).

        We begin with the following intermediary notion, which is necessary for defining vacuum modules.
        \begin{definition}[(Dual) Yangians] \label{def: (dual)_yangians}
            We define the \textbf{(formal) Yangian} (respectively, the \textbf{(formal) dual Yangian}) as the associative subalgebra of $\calU_{\hbar}(\calR)$ (for any $\level \in \bbC$) generated by the coefficients of the matrix entries of $L^-(u)$ (respectively, of $L^+(u)$) from equation \eqref{equation: RLL_creation_annihilation_operators}. Respectively, we denote them by:
                $$\calU^-_{\hbar}(\calR) \quad, \quad \calU^+_{\hbar}(\calR)$$
        \end{definition}
        \begin{example}[Trigonometric and rational (dual) Yangians] \label{example: trigonometric_and_rational_(dual)_yangians}
            \begin{itemize}
                \item When $\calR$ is a trigonometric R-matrix, one obtains via definition \ref{def: (dual)_yangians} the notions of $\hbar$-Yangians, also known as $q$-Yangians for $q := e^{\hbar}$.  
                \item When $\calR$ is a rational R-matrix, one obtains an isomorphism:
                    $$\calU^-_{\hbar}(\calR) \cong \calY_{\hbar}(\calR)$$
                with Drinfeld's Yangian from \cite{drinfeld_original_yangian_paper}, which we recalled in definition \ref{def: untwisted_yangians} (see also \cite[Chapter 1]{molev_yangians_and_classical_lie_algebras} for a more detailed discussion of these algebras, particularly in the R-matrix presentation). It is worth noting that the Yangian that one obtains in this case is independent of the diagram automorphism $\sigma$, thus relying only on the underlying finite-type Lie algebra $\g_N$. Also, as a side note, we remark that the matrices $L^{\pm}(u)$ are usually denoted by $T^{\pm}(u)$ in the rational case (cf. e.g. equation \eqref{equation: spectral_RTT_relation}).
            \end{itemize}
        \end{example}
        \begin{definition}[Vacuum modules] \label{def: vacuum_modules}
            The \textbf{(formal) vacuum module} associated to $\calR$ at level $\level \in \bbC$ is the left-$\DX_{\hbar}(\calR)_{\level}$-module given by:
                $$\vacuum_{\hbar}(\calR)_{\level} := \DX_{\hbar}(\calR)_{\level}/\DX_{\hbar}(\calR)_{\level} \cdot \< \{ T^-[r] \}_{r \geq 1} \>$$
            on which the matrices of generators $T^-[0], T^+[r] \in \Mat_N( \DX_{\hbar}(\calR)_{\level} )$ (with $r \geq 1$) act by left-multiplication.
        \end{definition}

        For the definition of general (formal) quantum vertex algebras, we refer the reader to section \ref{section: quantum_vertex_algebras}.
        \begin{proposition}[Trigonometric and rational affine qVAs] \label{prop: trigonometric_and_rational_affine_qVAs}
            The vacuum module $\vacuum_{\hbar}(\calR)_{\level}$ carries a $\hbar$-adic qVA structure $( \vacuum_{\hbar}(\calR)_{\level}, \ket{0}, Y_{\hbar}, D_{\hbar}, \calS )$, in which $\ket{0} \in \vacuum_{\hbar}(\calR)_{\level}$ is the vacuum vector, the state-field correspondence $Y_{\hbar}: \vacuum_{\hbar}(\calR)_{\level} \to \End_{\bbC[\![\hbar]\!]}( \vacuum_{\hbar}(\calR)_{\level} )[\![z^{\pm 1}]\!]$ is given by:
                \begin{equation} \label{equation: trigonometric_and_rational_qVAs_state_field_correpsondence}
                    Y_{\hbar}\left( L^+_{[n]}(\vec{u}) \cdot \ket{0}, z \right) := 
                \end{equation}
            and the braiding $\calS(z): \vacuum_{\hbar}(\calR)_{\level}^{\tensor 2} \to \vacuum_{\hbar}(\calR)_{\level}^{\tensor 2}(\!(z)\!)$ is given by:
                \begin{equation} \label{equation: trigonometric_and_rational_qVAs_braiding}
                    \calS() := 
                \end{equation}
        \end{proposition}
            \begin{proof}
                
            \end{proof}

        Next, we note the following relationship between modules over the affine qVA $\vacuum_{\hbar}(\calR)_{\level}$ and so-called \say{restricted modules} over the quantum loop algebra associated to $\calR$, in both the trigonometric and rational cases.
        \begin{definition}[Restricted modules over quantum loop algebras] \label{def: restricted_modules_over_loop_QUEs}
            
        \end{definition}
        \begin{proposition}[Modules over affine qVAs and loop QUEs] \label{prop: qVOA_quasi_modules_and_loop_QUE_restricted_modules}
            
        \end{proposition}
            \begin{proof}
                
            \end{proof}

    \subsection{Quantum Wakimoto modules associated to trigonometric and rational R-matrices}