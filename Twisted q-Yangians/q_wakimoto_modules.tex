\section{Free field realisations for quantum loop algebras}
    \subsection{Quantum affine vertex algebras associated to trigonometric and rational R-matrices}
        Having recalled certain features of the quantum loop algebras associated to a given trigonometric R-matrix, we next recall the construction of a quantum vertex algebra structure on the so-called \say{vacuum module}\footnote{Also known as \say{Weyl modules}, such as in \cite{etingof_kazhdan_quantisation_5}, but we would like to avoid this terminology as it is also used to refer to other constructions.}. This construction also takes as input the aforementioned trigonometric R-matrix, which for us is $\calR$ coming from the quantum Kac-Moody algebra $\calU_{\hbar}(\sfX)$, but is also applicable to the rational and elliptic cases. We are not interested in the last case, but it is important to us that it is possible to endow vacuum modules with structures of quantum vertex algebras in both the rational and trigonometric cases. This is because eventually, we would like it to be so that as $q \to 1$, the (twisted) $q$-Yangians will degenerate back to the usual (twisted) Yangians, and this compatibility must start at the level of quantum vertex algebras (and modules over them; more on this later).

        We begin with the following intermediary notion, which is necessary for defining vacuum modules.
        \begin{definition}[(Dual) Yangians] \label{def: (dual)_yangians}
            We define the \textbf{(formal) Yangian} (respectively, the \textbf{(formal) dual Yangian}) as the associative subalgebra of $\calU_{\hbar}(\calR)$ (for any $\level \in \bbC$) generated by the coefficients of the matrix entries of $L^-(u)$ (respectively, of $L^+(u)$) from equation \eqref{equation: RLL_creation_annihilation_operators}. Respectively, we denote them by:
                $$\calU^-_{\hbar}(\calR) \quad, \quad \calU^+_{\hbar}(\calR)$$
        \end{definition}
        \begin{remark}[Centrally extended (dual) Yangians] \label{remark: centrally_extended_(dual)_yangians}
            The algebras $\calU^{\pm}_{\hbar}(\calR)$ as in definition \ref{def: (dual)_yangians} above are isomorphic to the abstract associative algebras generated by the coefficients of the matrix entries of $L^{\pm}(u)$ (respectively) which are subjected only to the relation:
                $$L^-_{i, j}[0] = L^+_{j, i}[0] = 0, \quad 1 \leq i < j \leq N$$
            as well as the RLL relation \eqref{equation: extended_loop_QUE_RLL_relation} (cf. equation \eqref{equation: spectral_RTT_relation}).

            Equivalently, one can construct the algebras $\calU^{\pm}_{\hbar}(\calR)$ by first considering the subalgebras:
                $$\calX^{\pm}_{\hbar}(\calR)$$
            of $\DX_{\hbar}(\calR)_{\level}$, called \textbf{extended (dual) Yangians}, respectively generated by the coefficients of the matrix entries of $L^{\pm}(u)$. Then, by appealing to lemma \ref{lemma: centres_of_extended_loop_QUEs}, we set:
                $$\calU^{\pm}_{\hbar}(\calR) := \calX^{\pm}_{\hbar}(\calR)/\< \qdet L^{\pm}(u) - 1 \>$$

            One can also recover the extended (dual) Yangian $\calX^{\pm}_{\hbar}(\calR)$ from the \say{unextended} (dual) Yangian $\calU^{\pm}_{\hbar}(\calR)$ in the following manner, following \cite[Subsection 3.3]{etingof_kazhdan_quantisation_4}.

            Let $\del \in \Der_{\bbC[\![\hbar]\!]}( \calU^{\pm}_{\hbar}(\calR) )$ be the derivation given by:
                $$\del L^{\pm}(u) := \frac{d}{du} L^{\pm}(u)$$
            One can then enlarge each of the algebras $\calU^{\pm}_{\hbar}(\calR)$ by a central element $\level_{\calR}^{\pm}$ by forming the following $\bbC[\![\hbar]\!]$-algebra, which we note to be $\hbar$-adically complete by construction:
                $$\tilde{\calU}^{\pm}_{\hbar}(\calR)_{\level} := \left( \calU^{\pm}_{\hbar}(\calR) \hattensor \bbC[\level_{\calR}^{\pm}] \right)/\< [L^{\pm}(u), \level_{\calR}^{\pm}] = 0 \>$$
            One can then extend the Hopf structure on $\calU^{\pm}_{\hbar}(\calR)$ to the central extension $\tilde{\calU}^{\pm}_{\hbar}(\calR)_{\level}$ by:
                \begin{equation} \label{equation: centrally_extended_loop_QUEs_coproducts}
                    \tilde{\Delta}_{\hbar}(\level_{\calR}^{\pm}) := \level_{\calR}^{\pm} \tensor 1 + 1 \tensor \level_{\calR}^{\pm} \quad, \quad \tilde{\Delta}_{\hbar}( L^{\pm}(u) ) := e^{\frac12 \hbar (\level \tensor \del + \del \tensor \level)} \Delta_{\hbar}( L^{\pm}(u) )
                \end{equation}
                \begin{equation} \label{equation: centrally_extended_loop_QUEs_counit}
                    \tilde{\e}_{\hbar}( L^{\pm}(u) ) := 1 \quad, \quad \tilde{\e}_{\hbar}( \level_{\calR}^{\pm}) := 0
                \end{equation}
                \begin{equation} \label{equation: centrally_extended_loop_QUEs_antipode}
                    \tilde{\sigma}_{\hbar}( \level_{\calR}^{\pm} ) := -\level_{\calR}^{\pm}
                \end{equation}
        \end{remark}
        \begin{example}[Trigonometric and rational (dual) Yangians] \label{example: trigonometric_and_rational_(dual)_yangians}
            \begin{itemize}
                \item When $\calR$ is a trigonometric R-matrix, one obtains via definition \ref{def: (dual)_yangians} the notions of $\hbar$-Yangians, also known as $q$-Yangians for $q := e^{\hbar}$.  
                \item When $\calR$ is a rational R-matrix, one obtains an isomorphism:
                    $$\calU^-_{\hbar}(\calR) \cong \calY_{\hbar}(\calR)$$
                with Drinfeld's Yangian from \cite{drinfeld_original_yangian_paper} (see also \cite[Chapter 1]{molev_yangians_and_classical_lie_algebras} for a more detailed discussion of these algebras, particularly in the R-matrix presentation). It is worth noting that the Yangian that one obtains in this case is independent of the diagram automorphism $\sigma$, thus relying only on the underlying finite-type Lie algebra $\g_N$.

                Also, as a side note, the matrices $L^{\pm}(u)$ are usually denoted by $T^{\pm}(u)$ in the rational case (cf. e.g. equation \eqref{equation: spectral_RTT_relation}).
            \end{itemize}
        \end{example}
        \begin{definition}[Vacuum modules] \label{def: vacuum_modules}
            The \textbf{(formal) vacuum module} associated to $\calR$ at level $\level \in \bbC$ is the left-$\DX_{\hbar}(\calR)_{\level}$-module given by:
                $$\vacuum_{\hbar}(\calR)_{\level} := \DX_{\hbar}(\calR)_{\level}/\DX_{\hbar}(\calR)_{\level} \cdot \< \{ T^-[r] \}_{r \geq 1} \>$$
            on which the matrices of generators $T^-[0], T^+[r] \in \Mat_N( \DX_{\hbar}(\calR)_{\level} )$ (with $r \geq 1$) act by left-multiplication.
        \end{definition}

        For the definition of general (formal) quantum vertex algebras, we refer the reader to section \ref{section: quantum_vertex_algebras}.
        \begin{proposition}[Trigonometric and rational affine qVAs] \label{prop: trigonometric_and_rational_affine_qVAs}
            The vacuum module $\vacuum_{\hbar}(\calR)_{\level}$ carries a $\hbar$-adic qVA structure $( \vacuum_{\hbar}(\calR)_{\level}, \ket{0}, Y_{\hbar}, D_{\hbar}, \calS )$, in which $\ket{0} \in \vacuum_{\hbar}(\calR)_{\level}$ is the vacuum vector, the state-field correspondence $Y_{\hbar}: \vacuum_{\hbar}(\calR)_{\level} \to \End_{\bbC[\![\hbar]\!]}( \vacuum_{\hbar}(\calR)_{\level} )[\![z^{\pm 1}]\!]$ is given by:
                \begin{equation} \label{equation: trigonometric_and_rational_qVAs_state_field_correpsondence}
                    Y_{\hbar}\left( L^+_{[n]}(\vec{u}) \cdot \ket{0}, z \right) := 
                \end{equation}
            and the braiding $\calS(z): \vacuum_{\hbar}(\calR)_{\level}^{\tensor 2} \to \vacuum_{\hbar}(\calR)_{\level}^{\tensor 2}(\!(z)\!)$ is given by:
                \begin{equation} \label{equation: trigonometric_and_rational_qVAs_braiding}
                    \calS() := 
                \end{equation}
        \end{proposition}
            \begin{proof}
                
            \end{proof}

        Next, we note the following relationship between modules over the affine qVA $\vacuum_{\hbar}(\calR)_{\level}$ and so-called \say{restricted modules} over the quantum loop algebra associated to $\calR$, in both the trigonometric and rational cases.
        \begin{definition}[Restricted modules over quantum loop algebras] \label{def: restricted_modules_over_loop_QUEs}
            
        \end{definition}
        \begin{proposition}[Modules over affine qVAs and loop QUEs] \label{prop: qVOA_quasi_modules_and_loop_QUE_restricted_modules}
            
        \end{proposition}
            \begin{proof}
                
            \end{proof}

    \subsection{Quantum bosonic Heisenberg vertex algebras and quantum Wakimoto modules associated to trigonometric and rational R-matrices}
        We would now like to give a realisation of the quantum loop algebras defined in the previous subsection in terms of quantum vertex operators of free bosons. We expect such a realisation to exist since the R-matrix is unitary, signifying that the corresponding physical theory is that of free fields; their corresponding vertex operators are deformed in order to accommodate the $q$-deformed nature of the R-matrix at play. Setting up such a realisation first of all involves writing down Fock spaces on which creation and annihilation operators shall act; physically, one thinks of this stage of the process as setting up a theory obeying no non-trivial Lie-type symmetries, i.e. a so-called \say{quantum harmonic oscillator}. Such Fock spaces, as usual, shall be vacuum modules of a Heisenberg-type algebra. In light of the definition of general abstract vacuum modules (definition \ref{def: vacuum_modules}), one sees that the difficulty here lies in how one would quantise the usual affine Heisenberg algebra in terms of the R-matrix being considered. When the R-matrix at play is rational and of type $\sfA$ (i.e. that which gives rise to the Yangian $\calY(\sl_N)$), such a quantum Heisenberg algebra has already been studied by Butorac and Kozic in \cite{butorac_kozic_quantum_heisenberg_algebras_associated_with_type_A_rational_R_matrices}, so we shall rely on their work as a schematic for constructing quantum Heisenberg algebras associated to the trigonometric R-matrix $\calR$. Following their approach, the quantum Heisenberg algebra that we will obtain at the end of this subsection will be generated by the coefficients of certain quantum vertex algebras.

        Now, onto the technical details. We recall first of all that when regarded merely as an associative $\bbC[\![\hbar]\!]$-algebra, the quantum loop algebra $\calU_{\hbar}(\calR)$ is isomorphic to $\calU(\Loop^{\sigma}\g_N)[\![\hbar]\!]$. For the Lie algebra $\Loop^{\sigma}\g_N$, the appropriate choice of Fock space is:
            $$\boson := \Sym( \Loop^- \h_N ) \cdot \ket{0}$$
        with $\h_N \subset \g_N$ being the Cartan subalgebra and $\Loop^- \h_N := t^{-1} \h_N[t^{-1}]$, so let us begin by taking the underlying $\bbC[\![\hbar]\!]$-module of the quantum bosonic Fock space of $\calU_{\hbar}(\calR)$ to be:
            $$\boson_{\hbar} := \Sym( \Loop^- \h_N )[\![\hbar]\!] \cdot \ket{0}$$
        We note right away that this $\bbC[\![\hbar]\!]$-module naturally carries the $\hbar$-adic topology, with respect to which it is separated and complete. Moreover, it is $\bbC[\![\hbar]\!]$-torsion-free, thus making it topologically free as a topological $\bbC[\![\hbar]\!]$-module. Additionally, note that we indeed have:
            $$\boson_{\hbar}/\hbar \boson_{\hbar} \cong \boson$$
        as $\bbC[\![\hbar]\!]$-modules; had this not been true, any quantum vertex algebra structure that one may endow $\boson_{\hbar}$ with would have no chance of degenerating to the usual vertex algebra structure on $\boson$.

        Next, in order to construct creation and annihilation operators on $\boson_{\hbar}$, we take inspiration from the following phenomenon that occurs in the classical case.
        \begin{remark}[Classical Heisenberg degeneration] \label{remark: classical_heisenberg_degeneration}
            Suppose for a moment that $\g$ is an arbitrary finite-dimensional simple Lie algebra, fix a non-degenerate, symmetric, and invariant bilinear form $(-, -)_{\g}$ thereon, along with a Cartan subalgebra $\h \subset \g$. Let us also pick a basis $\{x_a\}_{a = 1}^{\dim \g}$ that is orthonormal with respect to the bilinear form $(-, -)_{\g}$. Next, form the affinisation with respect to $(-, -)_{\g}$:
                $$\tilde{\g} := \Loop \g \oplus \bbC c_{\aff}$$
            (which is \textit{a priori} isomorphic to the universal central extension of $\Loop \g$; see \cite{kassel_universal_central_extensions_of_lie_algebras}). We think of this Lie algebra as being spanned by the symbol $c_{\aff}$, along with the coefficients of the series:
                $$x_a(z) := \sum_{r \in \Z} x_a[r] z^{-r - 1}$$
            and the commutation relations between them are:
                $$[x_a(w), x_b(z)]_{\tilde{\g}} = $$
        \end{remark}
        
        \begin{proposition}
            
        \end{proposition}
            \begin{proof}
                
            \end{proof}

        Let us now establish a quantised analogue of the so-called Wakimoto homomorphism. Classically, it is a vertex algebra homomorphism from the affine vertex algebra to the Heisenberg vertex algebra. \textit{A priori}, this homomorphism is injective, and thus it gives us a description of the vertex operators generating the affine vertex algebra in terms of the vertex operators generating the Heisenberg vertex algebra, and hence a description of the currents generating the underlying affine Lie algebra in terms of differential operators on the bosonic Fock space; see \cite{frenkel_ben_zvi_vertex_algebras_and_algebraic_curves} for more details. We would now like to make an attempt at quantising the classical Wakimoto homomorphism mentioned above - with respect to the previously fixed R-matrix $\calR$ - in such a way that the classical limit coincides with said vertex algebra homomorphism. The purpose of doing this is the same as in the classical case, namely to describe the quantum vertex operators generating the quantum affine vertex algebra arising from $\calR$ in terms of the quantum vertex operators that generate the quantum Heisenberg algebra associated to the same R-matrix.
        \begin{proposition}[Quantum Wakimoto homomorphism] \label{prop: quantum_wakimoto_homomorphism}
            
        \end{proposition}
            \begin{proof}
                
            \end{proof}
        \begin{corollary}[Quantum Wakimoto modules] \label{coro: quantum_wakimoto_modules}
            
        \end{corollary}
            \begin{proof}
                
            \end{proof}