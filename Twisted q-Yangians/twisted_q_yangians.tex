\section{\texorpdfstring{Twisted $q$-Yangians}{}}
    \subsection{\texorpdfstring{Untwisted $q$-Yangians}{}}
        \begin{definition}[Untwisted $q$-Yangians]
            
        \end{definition}
    
        \todo[inline]{Classical limit of untwisted $q$-Yangians.}

    \subsection{\texorpdfstring{Twisted $q$-Yangians as coideal subalgebras of quantum (twisted) loop algebras}{}}
        \todo[inline]{Define the twisted $q$-Yangian $\calU_q^{\tw}(\calK)$ associated to $\calK$ as the images of the embedding $\calB^+(\calK) \hookrightarrow \calU_q(\sfX)$ given by $B(w) \mapsto S(w)$. By now, we should have shown that $\calU_q^{\tw}(\calK)$ is isomorphic to the Drinfeld-Jimbo QSP $\calU_q(\Loop^{\sigma}\g_N, \vartheta)$. The coideal structure of $\calU_q^{\tw}(\calK)$ can then be checked by computing $\Delta_q( S(w) )$.}

        \todo[inline]{Classical limit of twisted $q$-Yangians.}

    \subsection{Evaluation homomorphism}
        \todo[inline]{Evaluation homomorphism to $\calU_q(\g_N)$}

    \subsection{Specialisation}
        \todo[inline]{Does taking $q \to 1$ yield the usual twisted Yangians ?}

        \todo[inline]{It seems that the existence of and explicit formulae for universal K-matrices of twisted Yangians are not yet known. Therefore, one possible strategy is to take $q \to 1$ in the trigonometric universal K-matrices, and then check on representations to see if this yields intertwiners for representations of twisted Yangians.}