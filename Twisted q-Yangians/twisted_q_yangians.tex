\section{\texorpdfstring{Twisted $q$-Yangians}{}}
    \subsection{\texorpdfstring{Untwisted $q$-Yangians}{}}
        \begin{definition}[Untwisted $q$-Yangians]
            \todo[inline]{Subalgebra $\calY_q(\g_N)$ of $\calU_q(\Loop^{\sigma}\g_N)$ generated by the non-negative Fourier modes of the monodromy matrix. Probably will have to define an extended version $\calX_q(\g_N)$ first, and then identify the central elements of that extended algebra with $1$ like in \cite{guay_regelskis_twisted_yangians_for_symmetric_pairs_of_types_BCD}.}
        \end{definition}

        \todo[inline]{From what I've checked, I think the twisted $q$-Yangians are actually coideal subalgebras of the $q$-Yangians, not merely of the quantum loop algebras, and this should also be compatible with taking $q \to 1$.}

    \subsection{\texorpdfstring{Twisted $q$-Yangians as coideal subalgebras of quantum (twisted) loop algebras}{}}
        \todo[inline]{Define the twisted $q$-Yangian $\calY_q^{\tw}(\calK)$ associated to $\calK$ as the images of the embedding $\calB^+(\calK) \hookrightarrow \calU_q(\sfX)$ given by $B(w) \mapsto S(w)$. By now, we should have shown that $\calY_q^{\tw}(\calK)$ is isomorphic to the Drinfeld-Jimbo QSP $\calU_q(\Loop^{\sigma}\g_N, \vartheta)$. The coideal structure of $\calY_q^{\tw}(\calK)$ can then be checked by computing $\Delta_q( S(w) )$.}
        
        \todo[inline]{Existence of evaluation homomorphism to $\calU_q(\g_N)$.}

    \subsection{\texorpdfstring{Specialising $q \to 1$}{}}
        \todo[inline]{What is the right notion of classical limits for (twisted) $q$-Yangians ? Should these constructions be interpreted as $2$-parameter quantisations ? Taking $q \to 1$ ought to yield the usual (twisted) Yangian from \cite{guay_regelskis_twisted_yangians_for_symmetric_pairs_of_types_BCD}, and then those (twisted) Yangians have classical limits of their own, obtained as associated graded algebras (or as quotients modulo $\hbar$ if we consider the formal quantisations).}

        \todo[inline]{It seems that the existence of and explicit formulae for universal K-matrices of twisted Yangians are not yet known. Therefore, one possible strategy is to take $q \to 1$ in the trigonometric universal K-matrices, and then check on representations to see if this yields intertwiners for representations of twisted Yangians.}