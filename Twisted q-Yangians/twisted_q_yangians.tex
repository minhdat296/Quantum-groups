\section{\texorpdfstring{Twisted $q$-Yangians}{}}
    \subsection{\texorpdfstring{Untwisted $q$-Yangians}{}}
        \begin{definition}[Untwisted $q$-Yangians]
            \todo[inline]{Subalgebra $\calY_q(\g_N)$ of $\calU_{\hbar}(\Loop^{\sigma}\g_N)$ generated by the non-negative Fourier modes of the monodromy matrix. Probably will have to define an extended version $\calX_q(\g_N)$ first, and then identify the central elements of that extended algebra with $1$ like in \cite{guay_regelskis_twisted_yangians_for_symmetric_pairs_of_types_BCD}. From what I've checked, I think the twisted $q$-Yangians are actually coideal subalgebras of the $q$-Yangians, not merely of the quantum loop algebras, and this should also be compatible with taking $q \to 1$.}
        \end{definition}

    \subsection{\texorpdfstring{Twisted $q$-Yangians as coideal subalgebras of quantum (twisted) loop algebras}{}}

        \todo[inline]{Centre of twisted $q$-Yangians}

    \subsection{\texorpdfstring{Specialising $q \to 1$}{}}
        \todo[inline]{It seems that the existence of and explicit formulae for universal K-matrices of twisted Yangians are not yet known. Therefore, one possible strategy is to take $q \to 1$ in the trigonometric universal K-matrices, and then check on representations to see if this yields intertwiners for representations of twisted Yangians.}