\section{Quantum pseudo-symmetric pairs in the R-matrix presentation}
    \subsection{Pseudo-involutions and generalised Satake diagram}
        For now, let $\simpleroots$ be some finite set (to be thought of as indexing the simple roots), and then let $\g$ be the Kac-Moody algebra associated to a general indecomposable symmetrisable Cartan matrix $C := (C_{i, j})_{i, j \in \simpleroots}$ with set of Chevalley-Serre generators $\{h_i, e_i^{\pm}\}_{i \in \simpleroots}$ (as in \cite[Chapter 1]{kac_infinite_dimensional_lie_algebras}). Let $\h := \bigoplus_{i \in \simpleroots} \bbC h_i \subset \g$ be the standard Cartan subalgebra\footnote{By \cite{kac_peterson_infinite_flag_varieties_and_conjugacy_of_cartan_subalgebras}, we know that such subalgebras are conjugate to one another, so any choice is equally valid.} with respect to the aforementioned generators. Next, endow $\g$ with the $\Z$-grading given by $\deg h_i := 0$ and $\deg e_i^{\pm} := \pm 1$, and then let $(-, -)_{\g}$ be the invariant and symmetric bilinear form on $\g$ of total degree $0$ with respect to said $\Z$-grading, normalised so that $(h_i, h_j) = 2\delta_{i, j}$; additionally, let $\n^{\pm} \subset \g$ be the Lie subalgebras generated by the sets $\{e_i^{\pm}\}_{i \in \simpleroots}$, and let $\b^{\pm} := \h \oplus \n^{\pm}$ be the corresponding Borel subalgebras.
        
        Let $\g' := [\g, \g]$ be the derived subalgebra, which \textit{a priori} is isomorphic to the Lie subalgebra of $\g$ generated by the set $\{e_i^-, e_i^+(\}_{i \in \simpleroots}$, and then let $\h' := \h \cap \g'$. 

        Now, recall that given a vector space $V$ and two vector subspaces $V_1, V_2 \subset V$ thereof, those two subspaces are \textbf{commensurable} if $\codim(V_1 \cap V_2, V_1 + V_2) < +\infty$. We say that a Lie algebra automorphism:
            $$\vartheta \in \Aut_{\Lie\Alg}(\g)$$
        is of the \textbf{first kind} (respectively, of the \textbf{second kind}) if $\vartheta(\b^+)$ is commensurable with $\b^+$ (respectively, with $\b^-$) inside $\g$. \textit{A priori}, if $\vartheta$ is of the second kind, then up to conjugation of $\vartheta$ by some inner automorphism of $\g$, we have that:
            $$\vartheta(\h) = \h$$
        and moreover, any $\vartheta$-stable root space is in fact fixed by $\vartheta$. When $\vartheta$ itself possesses these properties, we shall write:
            $$\vartheta \in \Aut_{\Lie\Alg}(\g, \h)$$
        or merely $\vartheta \in \Aut(\g, \h)$. We note that this is a natural notion, seeing that the Chevalley involution, given by $\vartheta(e_i^{\pm}) := -e_i^{\mp}$ and $\vartheta(h_i) := -h_i$, is an example of an automorphism of the second kind. 
        \begin{definition}[Pseudo-involutions] \label{def: pseudo_involutions}
            A \textbf{pseudo-involution} of the symmetrisable Kac-Moody algbera $\g$ is an automorphism:
                $$\vartheta \in \Aut_{\Lie\Alg}(\g, \h)$$
            such that its restriction $\vartheta|_{\h} \in \Aut_{\Lie\Alg}(\h)$ down onto any Cartan subalgebra $\h \subset \g$ is involutive. 
        \end{definition}

    \subsection{Quantum pseudo-symmetric pairs in the Drinfeld-Jimbo presentation}

    \subsection{Trigonometric-type universal K-matrices}
        Again, let $\calR(w, z)$ be the universal R-matrix of the quantum Kac-Moody algebra $\calU_q(\sfX)$.
        \begin{definition}[Spectral bQYBE (multiplicative form)] \label{def: multiplicative_spectral_bQYBEs}
            By applying the following changes of variables:
                $$w := \exp(u), z := \exp(v)$$
            to the additive form of the boundary quantum Yang-Baxter equation (bQYBE) (equation \eqref{equation: additive_spectral_bQYBE}), one obtains the following, which shall be referred to as the \textbf{multiplicative form} of the bQYBE with spectral parameters (namely $w$ and $z$):
                \begin{equation} \label{equation: multiplicative_spectral_bQYBE}
                    \calR_{1, 2}\left(\frac{w}{z}\right) \calK_1(w) \calR_{1, 2}(wz) \calK_2(z) = \calK_2(z) \calR_{1, 2}(wz) \calK_1(w) \calR_{1, 2}\left(\frac{w}{z}\right) 
                \end{equation}
            Solutions:
                $$\calK(w)$$
            to equation \eqref{equation: multiplicative_spectral_bQYBE} shall be referred to collectively as \textbf{trigonometric-type universal K-matrices}.
        \end{definition}
        \begin{remark}
            In principle, the R-matrices that appear in the equation above can even be distinct from one another, but such cases are beyond the scope of our discussion here.
        \end{remark}

        \todo[inline]{Amongst the (generalised) reflection equations, there are two reflection equations of interested, \textbf{the untransposed and transposed ones} (see \cite[Section 6]{regelskis_vlaar_reflection_matrices_coideal_subalgebras}). Molev-Ragoucy-Sorba used the transposed one in their paper \cite{molev_ragoucy_sorba_twisted_q_yangians_type_A} initially, but then commented that the untransposed version would also give rise to the same twisted $q$-Yangians in type $\sfA$. In general, though, these two equations would give rise to different reflection algebras, but it's not clear to me if these reflection algebras are the same up to some kind of "Drinfeld twist". The ambiguity lies within the fact that the (transposed) reflection equations admit many "inequivalent" solutions, i.e. these different K-matrices - unlike the different R-matrices - may give rise to distinct coideal subalgebras of $\calU_q(\Loop^{\sigma}\g_N)$. It is also not clear, if after taking $q \to 1$, one would recover the same usual twisted Yangians. Regardless, \textit{loc. cit.} seems to indicate that it is possible to treat the two cases uniformly.}

        \todo[inline]{It seems that by performing a cylindrical analogue of Drinfeld twisting, one can change between the untransposed and transposed reflection equations (or rather, the K-matrices that solve these equations). This needs more investigation, but if it is indeed true, then it will suffice to only define twisted Yangians using the untransposed reflection equation.}

    \subsection{Reflection algebras}
        \todo[inline]{
            First of all, let $\calR$ be the universal R-matrix of $\calU_q(\sfX)$. Suppose that $\calB^+(\calK)$ is a reflection algebra defined by some solution $\calK(w)$ to a (generalised) reflection equation defined via $\calR$ as above. Moreover, to fix notations, suppose that $B(w)$ is the matrix wherein the coefficients $B_{i, j}^{(r)}$ (with $r \geq 0$) of the entries $B_{i, j}(w) := \sum_{r \geq 0} B_{i, j}^{(r)} w^{-r - 1}$ generate $\calB^+(\calK)$. In \cite[Subsection 10.2]{regelskis_vlaar_kac_moody_pseudo_symmetric_pairs}, it has already been shown that there is an algebra embedding of $\calB^+(\calK) \hookrightarrow \calU_q(\calR)$ given by $B(w) \mapsto S(u)$, with the boundary monodromy matrix $S(u)$ being given by some explicit formula. They conjectured that the images of the these embeddings should be isomorphic to the QSPs in the Drinfeld-Jimbo presentation; let us denote these by $\calU_q(\Loop^{\sigma}\g_N, \vartheta)$ in the meantime, with $\vartheta$ being the defining pseudo-involution. No proof was given, but it seems that we will need this equivalence between the two presentations, so we will have to supply a proof.
        }
        \todo[inline]{
            One way to do this is to begin by establishing a PBW basis for $\calB^+(\calK)$, and then compare it with the PBW basis that $\calU_q(\Loop^{\sigma}\g_N, \vartheta)$ inherits from $\calU_q(\Loop^{\sigma}\g_N)$. Such a PBW basis for $\calB^+(\calK)$ can be obtained via quantum vertex operators, and to this end we can adapt some works of Kozic. \textit{It is crucial that we have already obtained an algebra embedding $\calB^+(\calK) \hookrightarrow \calU_q(\calR)$ before attempting this.} 
        }
        \todo[inline]{
            Next, let $\DB(\calK)$ be the algebra generated by the set $\left\{ B_{i, j}^{(r)} \right\}_{r \in \Z}$, thus "doubling" $\calB^+(\calK)$ in a sense, and let $\calB^-(\calK)$ be the subalgebra thereof generated by the set $\left\{ B_{i, j}^{(-r)} \right\}_{r \geq 0}$.
        }