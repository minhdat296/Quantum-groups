\section{Free field realisations for reflection algebras}
    \subsection{Reflection algebras}
        \todo[inline]{
            First of all, let $\calR$ be the universal R-matrix of $\calU_q(\sfX)$. Suppose that $\calB^+(\calK)$ is a reflection algebra defined by some solution $\calK(w)$ to a (generalised) reflection equation defined via $\calR$ as above. Moreover, to fix notations, suppose that $B(w)$ is the matrix wherein the coefficients $B_{i, j}^{(r)}$ (with $r \geq 0$) of the entries $B_{i, j}(w) := \sum_{r \geq 0} B_{i, j}^{(r)} w^{-r - 1}$ generate $\calB^+(\calK)$. In \cite[Subsection 10.2]{regelskis_vlaar_kac_moody_pseudo_symmetric_pairs}, it has already been shown that there is an algebra embedding of $\calB^+(\calK) \hookrightarrow \calU_q(\calR)$ given by $B(w) \mapsto S(w)$, with the boundary monodromy matrix $S(w)$ being given by some explicit formula. They conjectured that the images of the these embeddings should be isomorphic to the QSPs in the Drinfeld-Jimbo presentation; let us denote these by $\calY_q^{\tw}(\calK)$. No proof was given, but it seems that we will need this equivalence between the two presentations, so we will have to supply a proof.
        }

    \subsection{\texorpdfstring{Boundary $q$-Wakimoto modules}{}}
        \todo[inline]{
            One way to do this is to begin by establishing a PBW basis for $\calB^+(\calK)$, and then compare it with the PBW basis that $\calY_q^{\tw}(\calK)$ inherits from $\calU_q(\Loop^{\sigma}\g_N)$. Such a PBW basis for $\calB^+(\calK)$ can be obtained via quantum vertex operators, more specifically a $q$-deformed free field realisation, and to this end we can adapt some works of Kozic. \textit{It is crucial that we have already obtained an algebra embedding $\calB^+(\calK) \hookrightarrow \calU_q(\calR)$ before attempting this}. The matrix $S(w)$ are given in terms of the matrice $L^{\pm}(w)$ and $\calK(w)$. For the former, we already have the Ding-Frenkel free field realisation, and for the latter, there are some formulas for its evaluation on integrable modules by Appel-Vlaar.
        }
        \todo[inline]{
            To do this, we will need some vacuum modules. Firstly, let $\vacuum^{\lambda}_{\kappa}(\calR)$ be the vacuum module for $\calU_q(\Loop^{\sigma}\g_N)$, with vacuum vector $\ket{\lambda}$ with weight $\lambda$; incidentally, this is why it is useful to think of the quantum loop algebra as a trigonometric analogue of the double Yangian $\DY(\g_N)$. Secondly, let $\DB(\calK)$ be the algebra generated by the set $\left\{ B_{i, j}^{(r)} \right\}_{r \in \Z}$, thus "doubling" $\calB^+(\calK)$ in a sense, and let $\calB^-(\calK)$ be the subalgebra thereof generated by the set $\left\{ B_{i, j}^{(-r)} \right\}_{r \geq 0}$. Then, define the vacuum module $\vacuum^{\lambda}_{\kappa}(\calK)$ of $\DB(\calK)$ as the $\calB^-(\calK)$-submodule of $\vacuum^{\lambda}_{\kappa}(\calR)$ generated by $\ket{\lambda}$; by PBW, this second vacuum module is isomorphic to $\calB^-(\calK)$. I need to double check my calculations, because my intuition here is still a bit fuzzy, but I believe that by mimicking Kozic's work in type $\sfA$, we can put a qVOA structure on $\vacuum^{\lambda}_{\kappa}(\calR)$, and then view $\vacuum^{\lambda}_{\kappa}(\calK)$ as a quasi-module over that qVOA.
        }